\documentclass[a4paper,twoside]{article}
\usepackage{fullpage}
\usepackage{graphics}
\usepackage{xspace}
\usepackage{alltt}
\usepackage[latin1]{inputenc}
\usepackage[T1]{fontenc}

\setcounter{secnumdepth}{3}  %% pour num�roter les subsubsections
\setcounter{tocdepth}{3}     %% profondeur dans la table des mati�res

\usepackage{times}

\title{Paj� User Manual}

\author{Benhur Stein}

\newenvironment{code}{\begin{alltt}}{\end{alltt}}

\begin{document}

\newcommand{\menu}[1]{\fbox{#1}}

\maketitle

\section{Starting the application}

To execute Paj�, the GNUstep environment must have been set
previously, with either
\begin{code}
source \emph{gnustep_path}/System/Makefiles/GNUstep.sh
\end{code}
or
\begin{code}
source \emph{gnustep_path}/System/Makefiles/GNUstep.csh
\end{code}
depending on your shell.
Then, the command
\begin{code}
openapp Paje
\end{code}
will execute it.
Once started, the \menu{Open...} menu
option can be used to open a file.
Alternatively, the command
\begin{code}
gopen \emph{file.trace}
\end{code}
will execute Paj� and make it open the trace file
\emph{file.trace}.

\section{Navegating the trace file}

The main visualization window is composed of three parts: 
\begin{itemize}
\item space-time diagram;
\item status area;
\item zooming buttons.
\end{itemize}

\subsection{Space-time diagram}
\subsection{Status area}
\subsection{Zooming buttons}

\section{Customizing the space-time view}

\section{Using Filters}


\end{document}
