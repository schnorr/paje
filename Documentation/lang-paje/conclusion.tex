%%%%%%%%%%%%%%%%%%%%%%%%%%%%%%%%%%%%%%%%%%%%%%%%%%%%%%%%%%%%%%%%%%%%%%%%%%%%%
%\chapter{Conclusion}
%%%%%%%%%%%%%%%%%%%%%%%%%%%%%%%%%%%%%%%%%%%%%%%%%%%%%%%%%%%%%%%%%%%%%%%%%%%%%

Paj� is a versatile visualization tool which can be used in a large
variety of contexts. This report describes the data format used by
Paj�. Paj� being trace-based, the data actually used for the
visualisation is to be presented as a set of execution events. In
addition, a description of the type hierarchy of the visual objects
needs to be included in the data (trace) file. Both the formats of the
type hierarchy description and of the events being self defined, there
also need to be a definition of these formats in the input data
(trace) file used by Paj�.

The versatility property has been used so far to visualize a
distributed Java application and the activity of the nodes of a
large-sized cluster of PCs. 

To enlarge the applicability of Paj�, a translator from traces
produced by Tau \cite{ShendeMCLBK:1998} into the Paj� format is
currently being implemented.


