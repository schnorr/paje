\documentclass[12pt]{article}
\usepackage{epsfig}
\usepackage{xspace}
\usepackage{times}
\usepackage[rubberchapters,clearempty,pagestyles]{titlesec}
\usepackage[utf8]{inputenc}
\usepackage[T1]{fontenc}
\usepackage{setspace}
\usepackage{hyperref}
\usepackage{url}
\usepackage{graphicx}
\usepackage{titletoc}
\usepackage{tabularx}
\usepackage{listings}
\usepackage{multicol}
\usepackage[margin=2cm,left=3cm,top=3cm]{geometry}

\newcommand{\Paje}{\textsc{Paj\'e}\xspace}
\newcommand{\PajeField}[1]{\emph{#1}\xspace}
\newcommand{\Time}{\PajeField{Time}}
\newcommand{\Name}{\PajeField{Name}}
\newcommand{\Type}{\PajeField{Type}}
\newcommand{\Container}{\PajeField{Container}}
\newcommand{\StartContainerType}{\PajeField{StartContainerType}}
\newcommand{\EndContainerType}{\PajeField{EndContainerType}}
\newcommand{\Value}{\PajeField{Value}}
\newcommand{\Alias}{\PajeField{Alias}}
\newcommand{\EndContainer}{\PajeField{EndContainer}}
\newcommand{\StartContainer}{\PajeField{StartContainer}}
\newcommand{\Key}{\PajeField{Key}}
%deprecated
\newcommand{\Shape}{\PajeField{Shape}}
\newcommand{\Height}{\PajeField{Height}}
\newcommand{\Width}{\PajeField{Width}}
\newcommand{\SourceContainer}{\PajeField{SourceContainer}}
\newcommand{\DestContainer}{\PajeField{DestContainer}}
\newcommand{\ContainerType}{\PajeField{ContainerType}}
\newcommand{\SourceContainerType}{\PajeField{SourceContainerType}}
\newcommand{\DestContainerType}{\PajeField{DestContainerType}}

\newcommand{\PajeEvent}[1]{\textsf{#1}\xspace}
\newcommand{\PajeDefineContainerType}{\PajeEvent{PajeDefineContainerType}}
\newcommand{\PajeDefineStateType}{\PajeEvent{PajeDefineStateType}}
\newcommand{\PajeDefineEventType}{\PajeEvent{PajeDefineEventType}}
\newcommand{\PajeDefineVariableType}{\PajeEvent{PajeDefineVariableType}}
\newcommand{\PajeDefineLinkType}{\PajeEvent{PajeDefineLinkType}}
\newcommand{\PajeCreateContainer}{\PajeEvent{PajeCreateContainer}}
\newcommand{\PajeDestroyContainer}{\PajeEvent{PajeDestroyContainer}}
\newcommand{\PajeSetState}{\PajeEvent{PajeSetState}}
\newcommand{\PajePushState}{\PajeEvent{PajePushState}}
\newcommand{\PajePopState}{\PajeEvent{PajePopState}}
\newcommand{\PajeNewEvent}{\PajeEvent{PajeNewEvent}}
\newcommand{\PajeSetVariable}{\PajeEvent{PajeSetVariable}}
\newcommand{\PajeAddVariable}{\PajeEvent{PajeAddVariable}}
\newcommand{\PajeSubVariable}{\PajeEvent{PajeSubVariable}}
\newcommand{\PajeStartLink}{\PajeEvent{PajeStartLink}}
\newcommand{\PajeEndLink}{\PajeEvent{PajeEndLink}}

\setlength{\parskip}{4pt}
\setlength{\parindent}{0px}

%limit table of contents to subsections
\setcounter{tocdepth}{2}


% itemize with minimal spacing
\newenvironment{itemize*}%
               {\vspace{-1em}
                 \begin{itemize}%
                   \setlength{\itemsep}{0pt}%
                   \setlength{\parskip}{0pt}}%
               {\end{itemize}}
\newenvironment{description*}%
               {\vspace{-1em}
                 \begin{description}%
                   \setlength{\itemsep}{0pt}%
                   \setlength{\parskip}{0pt}}%
               {\end{description}}

% page style (for footer and header)
\newpagestyle{paje}[\small\sffamily]
  {\headrule
    \sethead[\textsl{}][][]%
            {\textsl{\footnotesize \Paje \hfill Trace File Format }}{}{}
    \setfoot[][\usepage][]{}{-- \usepage~--}{}
  }


%Change the name of table of contents
\renewcommand{\contentsname}{\vspace{-2cm}}

%\contentsmargin{0px}
%\dottedcontents{section}[1em]{\bf}{1em}{1pc}
%\dottedcontents{subsection}[4em]{}{3.2em}{1pc}
%\dottedcontents{subsubsection}[5em]{}{3.2em}{1pc}

\titlecontents{section}
              [15pt]
              {\bf}
              {\contentslabel{15pt}}
              {}
              {\hfill\contentspage}
              [\vspace{-3pt}]
\titlecontents{subsection}
              [35pt]
              {}
              {\contentslabel{30pt}}
              {}
              {\titlerule*[3px]{.}\contentspage}
              [\vspace{-3pt}]

% section formatting
\titleformat{\section}[block]
{\normalfont\bfseries\sffamily}
{\Large{\thesection}.}{.2em}{\large\bfseries}

% subsection formatting
\titleformat{\subsection}[block]
{\normalfont\bfseries\sffamily}
{\thesubsection.}{.5em}{\bfseries}

% subsubsection formatting
\titleformat{\subsubsection}[block]
{\normalfont\bfseries\sffamily}
{\thesubsubsection}{.5em}{\normalsize\bfseries\ding{242}~}


\usepackage{amsmath}
\usepackage{pifont} % use of dingbats
\usepackage{array} % extension to the tabular env.
\usepackage{color} % for adding colors to things
\usepackage{colortbl} % for adding colors to tables

\newenvironment{captiontext}{\begin{quote} \footnotesize}{\end{quote}}
\newcommand{\ath}{\textsc{Atha\-pas\-can}\xspace}
\newcommand{\comment}[1]{}
\newlength{\figurewidth}\setlength{\figurewidth}{\textwidth}
%\addtolength{\figurewidth}{-\fboxsep}
%\addtolength{\figurewidth}{-\fboxsep}
\newsavebox{\figurebox}
\newcommand{\makefigure}[4]{%
   \begin{figure}[bt]\centering%
   \sbox{\figurebox}{#2}%
   \ifdim \wd\figurebox >\figurewidth%
      \resizebox{\figurewidth}{!}{\usebox{\figurebox}}%
   \else%
      \makebox[\figurewidth]{\usebox{\figurebox}}%
   \fi%
\vspace*{-3mm}
   \caption[#3]{\parbox[t]{11cm}{#3\newline\raggedright\protect\scriptsize #4}}%
\vspace*{2mm}
   \label{#1}%
   \end{figure}%
}
%% Tables
%
% Usage:
% \maketable{label}{figure command}{main caption}{secondary caption}
%
% there is a new column type for titles, T, to be used in \multicolumns,
% to put titles in reverse color, bold, sans serif, left aligned.
% there are also types R, L and C; they are like r, l, c but sans serif
%
\newlength{\tablewidth}\setlength{\tablewidth}{\textwidth}
\newcommand{\maketable}[4]{%
   \begin{table}[bt]\centering%
   \caption[#3]{#3\newline\footnotesize #4}\label{#1}%
   \sbox{\figurebox}{#2}%
   \ifdim \wd\figurebox >\tablewidth
      \resizebox{\tablewidth}{!}{\usebox{\figurebox}}%
   \else%
      \usebox{\figurebox}%
   \fi%
   \end{table}%
}
\newcolumntype{T}{>{\sffamily\bfseries\color{white}\columncolor[gray]{.2}}l}
\newcolumntype{R}{>{\sffamily}r}
\newcolumntype{L}{>{\sffamily}l}
\newcolumntype{C}{>{\sffamily}c}

\newcommand{\codefigurestart}{
  \begin{figure}[hbt]
    \begin{tabular}{|l|}
      \hline
      \begin{minipage}{\codewidth}
        \medskip\small
}

\newcommand{\codefigureend}[3]{
        \medskip
      \end{minipage}
      \\\hline
    \end{tabular}
%    \caption[#2]{#2\newline\footnotesize #3}%
    \caption[#2]{\parbox[t]{11cm}{#2\newline\raggedright\protect\scriptsize #3}}%
    \label{#1}%
  \end{figure}
}
\newlength{\codewidth}\setlength{\codewidth}{\figurewidth}
\addtolength{\codewidth}{-\fboxsep}
\addtolength{\codewidth}{-\fboxsep}
 % definitions

\begin{document}
\begin{onehalfspace}
{\Huge \Paje~-- trace file format} \\
{\Large Version {\bf 1.2} - March 14th, 2012} \\
\rule{\linewidth}{2pt}
\end{onehalfspace}

This report describes the \Paje trace file format, a self-defined,
textual, and generic format to described the behavior of computer
programs that are executed in parallel or distributed systems. A file
in this format includes three different kind of information: event
definition, type hierarchy declaration and timestamped events,
complying with the format definition, to be used to build
visualizations according to the type hierarchy.

{\bf Current maintainer:} \\
Lucas M. Schnorr (CNRS) -- schnorr@gmail.com

{\bf Original authors:} \\
Benhur de Oliveira Stein (UFSM) -- benhur@inf.ufsm.br \\
Jacques Chassin de Kergommeaux (INPG)

\vfill
{\bf \Large{Contents}}
\tableofcontents
\vfill

%{\bf ChangeLog} (see details in Section~\ref{ChangeLog}) \\
More information about \Paje at \url{http://paje.sf.net}.

\thispagestyle{empty}
\newpage
\pagestyle{paje}

\section{Introduction and overview}
\label{s.Introduction}

This report describes the \Paje trace file format, a self-defined,
textual, and generic format to described the behavior of computer
programs that are executed in parallel or distributed systems. A file
in this format includes three different kind of information: event
definition, type hierarchy definition and a set of recorded events,
complying with the format definition, to be used to build
visualizations according to the type hierarchy.

Before going into the details of the file format itself, it is
necessary to understand the five concepts of objects that exist in the
\Paje format. They are:

\begin{description}
\item [Container.] A container object can be any monitored entity, be
  it hardware or software. It represents anything which behavior can
  be described along time. A container can be a processor, an network
  link, a process, or a thread. {\bf Note:} Container is the only
  \Paje object that can hold other objects, including other
  containers. It is the main component to define the {\bf type
    hierarchy} necessary to \Paje trace files.  See
  Section~\ref{s.PajeDefineContainerType} to define specific container
  types.

\item [State.] A state is anything that has a beginning and ending
  timestamp. It is always associated to a container, and can be used
  to describe periods of time where a given container stays at the
  same state.  See Section~\ref{s.PajeDefineStateType} to define
  specific state types.

\item [Event.] An event is anything remarkable enough to be uniquely
  identified in a trace file, having only one timestamp (when it
  happened). It is always associated to a container.  See
  Section~\ref{s.PajeDefineEventType} to define specific event types.

\item [Variable.] A variable object represents the evolution of the
  value of a variable along time. It is always associated to a
  container and can be used to represent the numeric evolution of a
  variable along time. See Section~\ref{s.PajeDefineVariableType} to
  define specific variable types.

\item [Link.] A link is used to represent a directed relation between
  two containers, having a beginning and ending timestamps (which
  represent the start and end of the relation). It is always
  associated to a container. It can be used to represent
  point-to-point or collective communications between processes.  See
  Section~\ref{s.PajeDefineLinkType} to define specific link types.
\end{description}

These five concepts of objects in the \Paje data format must be
organized as a {\bf type hierarchy} in a \Paje trace file. The
\emph{containers} types are the nodes of the tree, while the other
four types (state, event, variable, and link) are the leaves of the
hierarchy. This hierarchy is specific to each trace file, although one
can use always the same type hierarchy for traces that are of the same
scenario with the same monitored objects. All MPI applications, for
example, might share the same type hierarchy. The type hierarchy
definition is often written in the beginning of a \Paje trace file, by
defining the specific types and the child/parent relation among
them. The Section~\ref{s.PajeEvents} details all the \Paje events that
can be used to define types.

After the type hierarchy definition, there is the list of timestamped
events that together describe the behavior of the monitored entities.
Section~\ref{s.PajeEvents} details all the \Paje events that can be
used in this point to describe behavior.

The text is organized as follows. Section~\ref{s.Specification}
presents the \Paje trace file format
specification. Section~\ref{s.PajeEvents} presents all the \Paje
events: those used to declare new types and those to describe the
behavior of monitored entities along time. Section~\ref{s.Technical}
presents technical details about the \Paje file format, such as the
notion of time, and use of aliases to reduce trace size.
Section~\ref{s.PajeHeaders} lists the two reference \Paje headers that
can be replicated. Section~\ref{ChangeLog} presents the modifications
of this specification.

\section{\Paje trace file format specification}
\label{s.Specification}

A trace file is composed of events.  An event can be seen as a table
composed of named fields, as shown in figure~\ref{f:event:table}.  The
first event in the figure can represent the sending of a message
containing 320 bytes by process 5 to process 3, containing 320 bytes
by process 5 to process 3, containing 320 bytes by process 5 to
process 3, containing 320 bytes by process 5 to process 3, 3.233222
seconds after the process started executing.  The second event shows
that process 5 unblocked at time 5.123002, and that this happened
while executing line 98 of file sync.c. 

\begin{figure}[!htb]
\begin{tabularx}{\textwidth}{XX}
\begin{tabular}{|>{\bf}rll|}
\hline
\textbf{Field Name} & \textbf{Field Type} & \textbf{Field Value} \\\hline
EventName     & string    & SendMessage \\
Time          & timestamp & 3.233222    \\
ProcessId     & integer   & 5           \\
Receiver      & integer   & 3           \\
Size          & integer   & 320         \\
\hline
\end{tabular}
&
\begin{tabular}{|>{\bf}rll|}
\hline
\textbf{Field Name} & \textbf{Field Type} & \textbf{Field Value} \\\hline
EventName     & string    & UnblockProcess \\
Time          & timestamp & 5.123002    \\
ProcessId     & integer   & 5           \\
FileName      & string    & sync.c      \\
LineNumber    & integer   & 98          \\
\hline
\end{tabular}
\end{tabularx}
\caption{Examples of events}
\label{f:event:table}
\end{figure}

Each event has some fields, each of them composed of a name, a type
and a value. Generally, there are lots of similar events in a trace
file (lots of ``SendMessage'' events, all with the same fields); a
typical trace file contains thousands of events of tens of different
types.  Usually, events of the same type have the same fields.  It is
therefore wise, in order to reduce the trace file size, not to put the
information that is common to many events in each of those events.
The most common solution is to put only the type of each event and the
values of its fields in the trace file. Information on what event
types exist and the fields that constitute each of these event types
being kept elsewhere.  In some trace file formats, this information is
hardcoded in the trace generator and trace reader, making the trace
structure hard to change in order to incorporate new types of events,
new data in existing events or to remove unused or unknown data from
those events.

A \Paje trace file is self defined, meaning that the event definition
information is put inside the trace file itself, much like the SDDF
file format used by the Pablo visualization tool\footnote{See ``The
  Pablo Self-Defining Data Format'', by Ruth A. Aydt (May 1994)}.  The
file is constituted of two parts: the definition of the events at the
beginning of the file followed by the events themselves.  The
definition of events contains the name of each event type and the
names and types of each field.  The second part of the trace file
contains the events, with the values associated to each field, in the
same order as in the definition.  The correspondence of an event with
its definition is made by means of a number, that must be unique for
each event description; this number appears in an event definition and
at the beginning of each event contained in the trace file.

\subsubsection*{First Part: Event Definition}

The event definition part of a \Paje trace file follows the following format:

\begin{itemize*}
\item All the lines start with a `\%' character.
\item Each event definition starts with a \%EventDef line and
  terminates with a \%EndEventDef line.
\item The \%EventDef line contains the name and the unique number of
  an event type.  The number (an integer) will be used to identify the
  event in the second part of the trace file. The choice of this
  number is left to the user. The numbers given in event definitions
  are thus \textbf{arbitrary}. The name of the event will be put in a
  field called \PajeField{PajeEventName}. There cannot be another
  field called so. The name is used to identify the type of an event.
\item the fields of an event are defined between the \%EventDef and
  the \%EndEventDef lines, one field per line, with the name
  of the field followed by its type (see below for details on its types).
\end{itemize*}

The type of a field can be one of the following:

\begin{description*}
  \item [date:] for fields that represent dates. It's a double
    precision floating-point number, usually meaning seconds since
    program start (See Section~\ref{s.NotionOfTime} for a discussion
    about the Notion of Time in \Paje)
  \item [int:] for fields containing integer numeric values;
  \item [double:] for fields containing floating-point values;
  \item [hex:] for fields that represent addresses, in hexadecimal;
  \item [string:] for strings of characters.
  \item [color:] for fields that represent colors. A color is a sequence of
                 three floating-point numbers between 0 and 1, inside double 
                 quotes (").
                 The three numbers are the values of red, green and blue
                 components.
\end{description*}

The structure of the two events of figure~\ref{f:event:table} are
shown below:

\begin{verbatim}
%EventDef SendMessage 21
%   Time       date
%   ProcessId  int
%   Receiver   int
%   Size       int
%EndEventDef
\end{verbatim}

\begin{verbatim}
%EventDef UnblockProcess 17
%   Time       date
%   ProcessId  int
%   LineNumber int
%   FileName   string
%EndEventDef
\end{verbatim}

\subsubsection*{Second Part: Events}

The second part of the trace file contains one event per line, whose
fields are separated by spaces or tabs, the first field being the
number that identifies the event type, followed by the other fields,
in the same order that they appear in the definition of the event.
Fields of type string must be inside double quotes (") if they contain
space or tab characters, or if they are empty.

For example, the two events of figure~\ref{f:event:table} are shown
below:

\begin{verbatim}
21 3.233222 5 3 320
17 5.123002 5 98 sync.c
\end{verbatim}

In \Paje, event numbers are used only as a means to find the
definition of an event; they are discarded as soon as an event is
read.  After being read, events are identified by their names.  Two
different definitions can have the same name (and different numbers),
making it possible to have, in the same trace file, two events of the
same type containing different fields.  We may use this feature to
optionally include the source file identification in some events. The
``UnblockProcess'' event in the examples above could also be defined
without the fields FileName and LineNumber, for use in places where
this information is not known or not necessary.

\section{\Paje events}
\label{s.PajeEvents}

\Paje includes a simulator module which builds the hierarchical data
structure from the elementary event records of the trace files.  \Paje
has no predefined containers or entities.  Before an entity can be
created and visualized, a hierarchy of container and entity types must
be defined, and containers must be instantiated. In this section all
the \Paje events to define and create entities are detailed.

\subsection{Types}
\label{s.Types}

New types can be defined at any moment of a \Paje trace file. The
events used to define these types share the fact that they have no
timestamp, and they are taken into account for the events that follow
the definition. In other words, a \Paje trace file is invalid if a
type is used before its definition.

This Section describes how new types are defined in a \Paje trace file
based on the five objects described in Section~\ref{s.Introduction}
(containers, states, events, links and variables). For each 

{\bf Note:} a figure is presented for each type definition, and they
show the obligatory and optional fields. Obligatory fields are
presented first. Optional fields can be safely omitted from a \Paje
trace file, without prejudice to the type definition.

%%%%%%%%%%%%%%%%%%%%%%%%%%%%%%%%%%%%%%%%%%%%%%%%%%%%%%%%%%%%%%%%%%%%%%%%%%%%%%%%
% PajeDefineContainerType
%%%%%%%%%%%%%%%%%%%%%%%%%%%%%%%%%%%%%%%%%%%%%%%%%%%%%%%%%%%%%%%%%%%%%%%%%%%%%%%%
\subsubsection{PajeDefineContainerType}
\label{s.PajeDefineContainerType}
The \PajeDefineContainerType event is used to define container
types. It must contain the fields \Name and \Type.  It defines a new
{\bf container} type called \Name, contained in a previously defined
container type of \Type (or the number zero -- {\bf 0} -- indicating
the root container type).

Optionally this event can contain a field \Alias with an
alias to identify this type. See Section~\ref{s.Alias} to correctly
use alias in a \Paje trace file.

\begin{tabular}{|LLL|}
\hline
\multicolumn{3}{|T|}{\textbf{\PajeDefineContainerType}}\\\hline
\textbf{Field Name} & \textbf{Field Type} & \textbf{Description}\\\hline
\Name          & string or integer & Name of new container type\\
\Type          & string or integer & Parent container type\\\hline
\Alias         & string or integer & Alternative name of new container type\\\hline
\end{tabular}

An invalid use of this event happens on the following cases:
\begin{itemize*}
\item \Name was already used in a previous definition
\item \Type is not a container type, or was not previously defined
\item \Alias was already used in a previous definition
\end{itemize*}

Related timestamped \Paje events:
\begin{itemize*}
\item \PajeCreateContainer (Section~\ref{s.PajeCreateContainer})
\item \PajeDestroyContainer (Section~\ref{s.PajeDestroyContainer})
\end{itemize*}

%%%%%%%%%%%%%%%%%%%%%%%%%%%%%%%%%%%%%%%%%%%%%%%%%%%%%%%%%%%%%%%%%%%%%%%%%%%%%%%%
% PajeDefineStateType
%%%%%%%%%%%%%%%%%%%%%%%%%%%%%%%%%%%%%%%%%%%%%%%%%%%%%%%%%%%%%%%%%%%%%%%%%%%%%%%%
\subsubsection{PajeDefineStateType}
\label{s.PajeDefineStateType}
The \PajeDefineStateType event is used to define new state types. It
must contain the fields \Name and \Type. It defines a new {\bf state}
entity type called \Name, contained in a previously defined container
type of \Type.

Optionally this event can contain a field \Alias with an alias to
identify this type. See Section~\ref{s.Alias} to correctly use alias
in a \Paje trace file.

\begin{tabular}{|LLL|}
\hline
\multicolumn{3}{|T|}{\textbf{\PajeDefineStateType}}\\\hline
\textbf{Field Name} & \textbf{Field Type} & \textbf{Description}\\\hline
\Name          & string or integer & Name of the new state type \\
\Type          & string or integer & Parent container type\\\hline
\Alias         & string or integer & Alternative name of new state type \\\hline
\end{tabular}

An invalid use of this event happens on the following cases:
\begin{itemize*}
\item \Name was already used in a previous definition
\item \Type is not a container type, or was not previously defined
\item \Alias was already used in a previous definition
\end{itemize*}

Related timestamped \Paje events:
\begin{itemize*}
\item \PajeSetState (Section~\ref{s.PajeSetState})
\item \PajePushState (Section~\ref{s.PajePushState})
\item \PajePopState (Section~\ref{s.PajePopState})
\end{itemize*}

%%%%%%%%%%%%%%%%%%%%%%%%%%%%%%%%%%%%%%%%%%%%%%%%%%%%%%%%%%%%%%%%%%%%%%%%%%%%%%%%
% PajeDefineEventType
%%%%%%%%%%%%%%%%%%%%%%%%%%%%%%%%%%%%%%%%%%%%%%%%%%%%%%%%%%%%%%%%%%%%%%%%%%%%%%%%
\subsubsection{PajeDefineEventType}
\label{s.PajeDefineEventType}
The \PajeDefineEventType event is used to define new event types. This
event must contain the fields \Name and \Type.  It defines a new {\bf
  event} entity type called \Name, contained in a previously defined
container type of \Type.

Optionally this event can contain a field \Alias with an alias to
identify this type. See Section~\ref{s.Alias} to correctly use alias
in a \Paje trace file.

\begin{tabular}{|LLL|}
\hline
\multicolumn{3}{|T|}{\textbf{\PajeDefineEventType}}\\\hline
\textbf{Field Name} & \textbf{Field Type} & \textbf{Description}\\\hline
\Name          & string or integer & Name of new event type \\
\Type          & string or integer & Parent container type\\\hline
\Alias         & string or integer & Alternative name of new event type \\\hline
\end{tabular}

An invalid use of this event happens on the following cases:
\begin{itemize*}
\item \Name was already used in a previous definition
\item \Type is not a container type, or was not previously defined
\item \Alias was already used in a previous definition
\end{itemize*}

Related timestamped \Paje events:
\begin{itemize*}
\item \PajeNewEvent (Section~\ref{s.PajeNewEvent})
\end{itemize*}

%%%%%%%%%%%%%%%%%%%%%%%%%%%%%%%%%%%%%%%%%%%%%%%%%%%%%%%%%%%%%%%%%%%%%%%%%%%%%%%%
% PajeDefineVariableType
%%%%%%%%%%%%%%%%%%%%%%%%%%%%%%%%%%%%%%%%%%%%%%%%%%%%%%%%%%%%%%%%%%%%%%%%%%%%%%%%
\subsubsection{PajeDefineVariableType}
\label{s.PajeDefineVariableType}
The \PajeDefineVariableType event is used to define new variable
types. This event must contain the fields \Name and \Type.  It defines
a new {\bf variable} entity type called \Name, contained in a
previously defined container type of \Type.

Optionally this event can contain a field \Alias with an alias to
identify this type. See Section~\ref{s.Alias} to correctly use alias
in a \Paje trace file.

\begin{tabular}{|LLL|}
\hline
\multicolumn{3}{|T|}{\textbf{\PajeDefineVariableType}}\\\hline
\textbf{Field Name} & \textbf{Field Type} & \textbf{Description}\\\hline
\Name          & string or integer & Name of new variable type \\
\Type          & string or integer & Parent container type\\\hline
\Alias         & string or integer & Alternative name of new variable type \\\hline
\end{tabular}

An invalid use of this event happens on the following cases:
\begin{itemize*}
\item \Name was already used in a previous definition
\item \Type is not a container type, or was not previously defined
\item \Alias was already used in a previous definition
\end{itemize*}

Related timestamped \Paje events:
\begin{itemize*}
\item \PajeSetVariable (Section~\ref{s.PajeSetVariable})
\item \PajeAddVariable (Section~\ref{s.PajeAddVariable})
\item \PajeSubVariable (Section~\ref{s.PajeSubVariable})
\end{itemize*}

%%%%%%%%%%%%%%%%%%%%%%%%%%%%%%%%%%%%%%%%%%%%%%%%%%%%%%%%%%%%%%%%%%%%%%%%%%%%%%%%
% PajeDefineLinkType
%%%%%%%%%%%%%%%%%%%%%%%%%%%%%%%%%%%%%%%%%%%%%%%%%%%%%%%%%%%%%%%%%%%%%%%%%%%%%%%%
\subsubsection{PajeDefineLinkType}
\label{s.PajeDefineLinkType}
The \PajeDefineLinkType event is used to define new variable
types. This event must contain the fields \Name and \Type,
\StartContainerType, and \EndContainerType. It defines a new {\bf
  link} entity type called \Name, contained in a previously defined
container type of \Type, that connects the previously defined
container type \StartContainerType the previously defined container
type \EndContainerType. Additionally, the container type provided in
\Type must be a common ancestral of containers types provided in
\StartContainerType and \EndContainerType.

Optionally this event can contain a field \Alias with an alias to
identify this type. See Section~\ref{s.Alias} to correctly use alias
in a \Paje trace file.

\begin{tabular}{|LLL|}
\hline
\multicolumn{3}{|T|}{\textbf{\PajeDefineLinkType}}\\\hline
\textbf{Field Name} & \textbf{Field Type} & \textbf{Description}\\\hline
\Name                & string or integer & Name of new link type \\
\Type                & string or integer & Common ancestral container type \\
\StartContainerType  & string or integer & Container type of the starting container\\
\EndContainerType    & string or integer & Container type of the ending container\\\hline
\Alias               & string or integer & Alternative name of new link type \\\hline
\end{tabular}

An invalid use of this event happens on the following cases:
\begin{itemize*}
\item \Name was already used in a previous definition
\item \Type is not a container type, or was not previously defined
\item \Type is not a common ancestral of \StartContainerType and \EndContainerType
\item \StartContainerType is not a container type, or was not previously defined
\item \EndContainerType is not a container type, or was not previously defined
\item \Alias was already used in a previous definition
\end{itemize*}

Related timestamped \Paje events:
\begin{itemize*}
\item \PajeStartLink (Section~\ref{s.PajeStartLink})
\item \PajeEndLink (Section~\ref{s.PajeEndLink})
\end{itemize*}

\subsection{Containers}
Instances of containers are created using the \PajeCreateContainer
event, and destroyed using the \PajeDestroyContainer event. If
instances of containers are not destroyed, they must exist until the
end of the trace file. These two events are detailed below.

%%%%%%%%%%%%%%%%%%%%%%%%%%%%%%%%%%%%%%%%%%%%%%%%%%%%%%%%%%%%%%%%%%%%%%%%%%%%%%%%
% PajeCreateContainer
%%%%%%%%%%%%%%%%%%%%%%%%%%%%%%%%%%%%%%%%%%%%%%%%%%%%%%%%%%%%%%%%%%%%%%%%%%%%%%%%
\subsubsection{PajeCreateContainer}
\label{s.PajeCreateContainer}
The \PajeCreateContainer event creates a container instance in a given
timestamp. This event must contain the fields \Time, \Name, \Type, and
\Container. It creates, at timestamp \Time, a container instance
uniquely identified by \Name of the container type \Type. This new
container instance is a new child of the previously created container
instance identified by \Container (the parent container instance).

Optionally this event can contain a field \Alias with an alias to
identify this type. See Section~\ref{s.Alias} to correctly use alias
in a \Paje trace file.

\begin{tabular}{|LLL|}
\hline
\multicolumn{3}{|T|}{\textbf{\PajeCreateContainer}}\\\hline
\textbf{Field Name} & \textbf{Field Type} & \textbf{Description}\\\hline
\Time          & date              & Time of creation of container \\
\Name          & string or integer & Name of new container \\
\Type          & string or integer & Container type of new container \\
\Container     & string or integer & Parent of new container \\\hline
\Alias         & string or integer & Alternative name of new container \\
\hline
\end{tabular}

An invalid use of this event happens on the following cases:
\begin{itemize*}
\item \Time is not present, or is not a timestamp (see Section~\ref{s.NotionOfTime} for time considerations)
\item \Name instance identifier was already used in a previous container creation
\item \Type is not a container type, or was not previously defined
\item \Type is not a child type of the container type of \Container
\item \Container instance was not previously created
\item \Alias instance identifier was already used in a previous container creation
\end{itemize*}

Related type definition \Paje events:
\begin{itemize*}
\item \PajeDefineContainerType (Section~\ref{s.PajeDefineContainerType})
\end{itemize*}

%%%%%%%%%%%%%%%%%%%%%%%%%%%%%%%%%%%%%%%%%%%%%%%%%%%%%%%%%%%%%%%%%%%%%%%%%%%%%%%%
% PajeDestroyContainer
%%%%%%%%%%%%%%%%%%%%%%%%%%%%%%%%%%%%%%%%%%%%%%%%%%%%%%%%%%%%%%%%%%%%%%%%%%%%%%%%
\subsubsection{PajeDestroyContainer}
\label{s.PajeDestroyContainer}
The \PajeDestroyContainer event destroys a container instance in a
given timestamp. This event must contain the fields \Time, \Name,
\Type. It destroys, at timestamp \Time, a container instance uniquely
identified by \Name of the container type \Type.

\begin{tabular}{|LLL|}
\hline
\multicolumn{3}{|T|}{\textbf{\PajeDestroyContainer}}\\\hline
\textbf{Field Name} & \textbf{Field Type} & \textbf{Description}\\
\hline
\Time          & date              & Time of destruction of container \\
\Name          & string or integer & Name of container \\
\Type          & string or integer & Type of container \\
\hline
\end{tabular}

An invalid use of this event happens on the following cases:
\begin{itemize*}
\item \Time is not present, or is not a timestamp (see Section~\ref{s.NotionOfTime} for time considerations)
\item \Name instance identifier does not exist (see Section~\ref{s.Alias} if alias are used in the trace file)
\item \Type is not a container type, or was not previously defined
\item \Type is not the container type of the container instance identified by \Name
\end{itemize*}

Related type definition \Paje events:
\begin{itemize*}
\item \PajeDefineContainerType (Section~\ref{s.PajeDefineContainerType})
\end{itemize*}

\subsection{States}
The \PajeSetState updates a state of a container instance to a new
value (and erase any previously saved values). The \PajePushState
pushes a new value of a state of a container instance, saving the old
state. The \PajePopState pops the previously saved value of a state of
a container instance.  These three events are detailed below.

%%%%%%%%%%%%%%%%%%%%%%%%%%%%%%%%%%%%%%%%%%%%%%%%%%%%%%%%%%%%%%%%%%%%%%%%%%%%%%%%
% PajeSetState
%%%%%%%%%%%%%%%%%%%%%%%%%%%%%%%%%%%%%%%%%%%%%%%%%%%%%%%%%%%%%%%%%%%%%%%%%%%%%%%%
\subsubsection{PajeSetState}
\label{s.PajeSetState}
The \PajeSetState event is used to update the value of a state of a
container instance. This event must contain the fields \Time, \Type,
\Container and \Value. It changes the state type \Type to the value
\Value of the container identified by \Container at time \Time.

\begin{tabular}{|LLL|}
\hline
\multicolumn{3}{|T|}{\textbf{\PajeSetState}}\\\hline
\textbf{Field Name} & \textbf{Field Type} & \textbf{Description}\\\hline
\Time          & date              & Time the state changed \\
\Type          & string or integer & State type \\
\Container     & string or integer & Container whose state changed \\
\Value         & string or integer & Value of new state of container \\\hline
\end{tabular}

An invalid use of this event happens on the following cases:
\begin{itemize*}
\item \Time is not present, or is not a timestamp (see Section~\ref{s.NotionOfTime} for time considerations)
\item \Type is not a state type, or it was not previously defined
\item \Type is not a child type of the container type of \Container
\item \Container instance was not previously created
\end{itemize*}

Related type definition \Paje events:
\begin{itemize*}
\item \PajeDefineStateType (Section~\ref{s.PajeDefineStateType})
\end{itemize*}

%%%%%%%%%%%%%%%%%%%%%%%%%%%%%%%%%%%%%%%%%%%%%%%%%%%%%%%%%%%%%%%%%%%%%%%%%%%%%%%%
% PajePushState
%%%%%%%%%%%%%%%%%%%%%%%%%%%%%%%%%%%%%%%%%%%%%%%%%%%%%%%%%%%%%%%%%%%%%%%%%%%%%%%%
\subsubsection{PajePushState}
\label{s.PajePushState}
The \PajePushState event is used to push the value of a state of a
container instance, saving the existing value of the same state. This
event must contain the fields \Time, \Type, \Container and \Value. It
pushes the value \Value of state type \Type in the container
identified by \Container at time \Time. 

\begin{tabular}{|LLL|}
\hline
\multicolumn{3}{|T|}{\textbf{\PajePushState}}\\\hline
\textbf{Field Name} & \textbf{Field Type} & \textbf{Description}\\\hline
\Time          & date              & Time the state changed \\
\Type          & string or integer & State type \\
\Container     & string or integer & Container whose state changed \\
\Value         & string or integer & Value of new state of container \\\hline
\end{tabular}

An invalid use of this event happens on the following cases:
\begin{itemize*}
\item \Time is not present, or is not a timestamp (see Section~\ref{s.NotionOfTime} for time considerations)
\item \Type is not a state type, or it was not previously defined
\item \Type is not a child type of the container type of \Container
\item \Container instance was not previously created
\item There is no existing value to be saved (see note below)
\end{itemize*}

Related type definition \Paje events:
\begin{itemize*}
\item \PajeDefineStateType (Section~\ref{s.PajeDefineStateType})
\end{itemize*}

{\bf Note:} Considering a state \Type in a \Container, when a
\PajePushState is read from the trace file, it is expected that a
previous \PajeSetState was already read for the same state type and
container.

%%%%%%%%%%%%%%%%%%%%%%%%%%%%%%%%%%%%%%%%%%%%%%%%%%%%%%%%%%%%%%%%%%%%%%%%%%%%%%%%
% PajePopState
%%%%%%%%%%%%%%%%%%%%%%%%%%%%%%%%%%%%%%%%%%%%%%%%%%%%%%%%%%%%%%%%%%%%%%%%%%%%%%%%
\subsubsection{PajePopState}
\label{s.PajePopState}
The \PajePopState event is used to pop the previously saved value
(with \PajePushState) of a state of a container instance. This event
must contain the fields \Time, \Type, and \Container. It pops the
state type \Type in the container identified by \Container at time
\Time.

\begin{tabular}{|LLL|}
\hline
\multicolumn{3}{|T|}{\textbf{\PajePopState}}\\\hline
\textbf{Field Name} & \textbf{Field Type} & \textbf{Description}\\\hline
\Time          & date              & Time the state changed \\
\Type          & string or integer & State type \\
\Container     & string or integer & Container whose state changed \\\hline
\end{tabular}

An invalid use of this event happens on the following cases:
\begin{itemize*}
\item \Time is not present, or is not a timestamp (see Section~\ref{s.NotionOfTime} for time considerations)
\item \Type is not a state type, or it was not previously defined
\item \Type is not a child type of the container type of \Container
\item \Container instance was not previously created
\item There is no saved value to pop (see note below)
\end{itemize*}

Related type definition \Paje events:
\begin{itemize*}
\item \PajeDefineStateType (Section~\ref{s.PajeDefineStateType})
\end{itemize*}

{\bf Note:} Considering a state \Type in a \Container, when a
\PajePopState is read from the trace file, it is expected that a
previous \PajePushState was already read for the same state type and
container.

\subsection{Events}
%%%%%%%%%%%%%%%%%%%%%%%%%%%%%%%%%%%%%%%%%%%%%%%%%%%%%%%%%%%%%%%%%%%%%%%%%%%%%%%%
% PajeNewEvent
%%%%%%%%%%%%%%%%%%%%%%%%%%%%%%%%%%%%%%%%%%%%%%%%%%%%%%%%%%%%%%%%%%%%%%%%%%%%%%%%
\subsubsection{PajeNewEvent}
\label{s.PajeNewEvent}
The \PajeNewEvent event instantiates a remarkable event with a unique
timestamp. This event must contain the fields \Time, \Type,
\Container, and \Value.

\begin{tabular}{|LLL|}
\hline
\multicolumn{3}{|T|}{\textbf{\PajeNewEvent}}\\\hline
\textbf{Field Name} & \textbf{Field Type} & \textbf{Description}\\\hline
\Time          & date              & Time the event happened \\
\Type          & string or integer & Type of event \\
\Container     & string or integer & Container that produced event \\
\Value         & string or integer & Value of new event \\\hline
\end{tabular}

An invalid use of this event happens on the following cases:
\begin{itemize*}
\item \Time is not present, or is not a timestamp (see Section~\ref{s.NotionOfTime} for time considerations)
\item \Type is not a event type, or it was not previously defined
\item \Type is not a child type of the container type of \Container
\item \Container instance was not previously created
\end{itemize*}

Related type definition \Paje events:
\begin{itemize*}
\item \PajeDefineEventType (Section~\ref{s.PajeDefineEventType})
\end{itemize*}

\subsection{Variables}
The \PajeSetVariable changes the value of a variable of a container
instance to a new value. The \PajeAddVariable adds the value provided
to the current value, while \PajeSubVariable subtracts the value
provided from the current value.  These three events are detailed
below.

%%%%%%%%%%%%%%%%%%%%%%%%%%%%%%%%%%%%%%%%%%%%%%%%%%%%%%%%%%%%%%%%%%%%%%%%%%%%%%%%
% PajeSetVariable
%%%%%%%%%%%%%%%%%%%%%%%%%%%%%%%%%%%%%%%%%%%%%%%%%%%%%%%%%%%%%%%%%%%%%%%%%%%%%%%%
\subsubsection{PajeSetVariable}
\label{s.PajeSetVariable}
The \PajeSetVariable event sets the value of a variable type of a
container instance to a new value. This event must contain the fields
\Time, \Type, \Container, and \Value.

\begin{tabular}{|LLL|}
\hline
\multicolumn{3}{|T|}{\textbf{\PajeSetVariable}}\\\hline
\textbf{Field Name} & \textbf{Field Type} & \textbf{Description}\\\hline
\Time          & date              & Time the variable changed value\\
\Type          & string or integer & Type of variable \\
\Container     & string or integer & Container whose value changed \\
\Value         & double            & New value of variable \\\hline
\end{tabular}

An invalid use of this event happens on the following cases:
\begin{itemize*}
\item \Time is not present, or is not a timestamp (see Section~\ref{s.NotionOfTime} for time considerations)
\item \Type is not a variable type, or it was not previously defined
\item \Type is not a child type of the container type of \Container
\item \Container instance was not previously created
\item \Value is not a double
\end{itemize*}

Related type definition \Paje events:
\begin{itemize*}
\item \PajeDefineVariableType (Section~\ref{s.PajeDefineVariableType})
\end{itemize*}

%%%%%%%%%%%%%%%%%%%%%%%%%%%%%%%%%%%%%%%%%%%%%%%%%%%%%%%%%%%%%%%%%%%%%%%%%%%%%%%%
% PajeAddVariable
%%%%%%%%%%%%%%%%%%%%%%%%%%%%%%%%%%%%%%%%%%%%%%%%%%%%%%%%%%%%%%%%%%%%%%%%%%%%%%%%
\subsubsection{PajeAddVariable}
\label{s.PajeAddVariable}
The \PajeAddVariable event adds a value to the existing value of a
variable type of a container instance. This event must contain the
fields \Time, \Type, \Container, and \Value.

\begin{tabular}{|LLL|}
\hline
\multicolumn{3}{|T|}{\textbf{\PajeAddVariable}}\\\hline
\textbf{Field Name} & \textbf{Field Type} & \textbf{Description}\\\hline
\Time          & date              & Time the variable changed value\\
\Type          & string or integer & Type of variable \\
\Container     & string or integer & Container whose value changed \\
\Value         & double            & Value to be added to variable \\\hline
\end{tabular}

An invalid use of this event happens on the following cases:
\begin{itemize*}
\item \Time is not present, or is not a timestamp (see Section~\ref{s.NotionOfTime} for time considerations)
\item \Type is not a variable type, or it was not previously defined
\item \Type is not a child type of the container type of \Container
\item \Container instance was not previously created
\item \Value is not a double
\item There is no previous defined value for the variable (see note below)
\end{itemize*}

Related type definition \Paje events:
\begin{itemize*}
\item \PajeDefineVariableType (Section~\ref{s.PajeDefineVariableType})
\end{itemize*}

{\bf Note:} Considering a variable \Type in a \Container, when a
\PajeAddVariable is read from the trace file, it is expected that a
previous \PajeSetVariable was already read for the same variable type
and container.

%%%%%%%%%%%%%%%%%%%%%%%%%%%%%%%%%%%%%%%%%%%%%%%%%%%%%%%%%%%%%%%%%%%%%%%%%%%%%%%%
% PajeSubVariable
%%%%%%%%%%%%%%%%%%%%%%%%%%%%%%%%%%%%%%%%%%%%%%%%%%%%%%%%%%%%%%%%%%%%%%%%%%%%%%%%
\subsubsection{PajeSubVariable}
\label{s.PajeSubVariable}
The \PajeSubVariable event subtracts a value from the existing value
of a variable type of a container instance. This event must contain
the fields \Time, \Type, \Container, and \Value.

\begin{tabular}{|LLL|}
\hline
\multicolumn{3}{|T|}{\textbf{\PajeSubVariable}}\\\hline
\textbf{Field Name} & \textbf{Field Type} & \textbf{Description}\\\hline
\Time          & date              & Time the variable changed value\\
\Type          & string or integer & Type of variable \\
\Container     & string or integer & Container whose value changed \\
\Value         & double            & Value to be subtracted from variable \\\hline
\end{tabular}

An invalid use of this event happens on the following cases:
\begin{itemize*}
\item \Time is not present, or is not a timestamp (see Section~\ref{s.NotionOfTime} for time considerations)
\item \Type is not a variable type, or it was not previously defined
\item \Type is not a child type of the container type of \Container
\item \Container instance was not previously created
\item \Value is not a double
\item There is no previous defined value for the variable (see note below)
\end{itemize*}

Related type definition \Paje events:
\begin{itemize*}
\item \PajeDefineVariableType (Section~\ref{s.PajeDefineVariableType})
\end{itemize*}

{\bf Note:} Considering a variable \Type in a \Container, when a
\PajeSubVariable is read from the trace file, it is expected that a
previous \PajeSetVariable was already read for the same variable type
and container.

\subsection{Links}
A \Paje link is defined by two events: a \PajeStartLink and a
\PajeEndLink. These two events are matched and considered to form a
link when their respective \Container, \Value and \Key fields, which
should always be the same.

%%%%%%%%%%%%%%%%%%%%%%%%%%%%%%%%%%%%%%%%%%%%%%%%%%%%%%%%%%%%%%%%%%%%%%%%%%%%%%%%
% PajeStartLink
%%%%%%%%%%%%%%%%%%%%%%%%%%%%%%%%%%%%%%%%%%%%%%%%%%%%%%%%%%%%%%%%%%%%%%%%%%%%%%%%
\subsubsection{PajeStartLink}
\label{s.PajeStartLink}
The \PajeStartLink event indicates the beginning of a link between two
containers. This event must contain the fields \Time, \Type,
\Container, \StartContainer, \Value and \Key.

\begin{tabular}{|LLL|}
\hline
\multicolumn{3}{|T|}{\textbf{\PajeStartLink}}\\\hline
\textbf{Field Name} & \textbf{Field Type} & \textbf{Description}\\\hline
\Time           & date              & Time the link started\\
\Type           & string or integer & Type of link \\
\Container      & string or integer & Container that has the link \\
\StartContainer & string or integer & Container where link started \\
\Value          & string or integer & Value of link \\
\Key            & string or integer & Used to match to link end \\\hline
\end{tabular}

An invalid use of this event happens on the following cases:
\begin{itemize*}
\item \Time is not present, or is not a timestamp (see Section~\ref{s.NotionOfTime} for time considerations)
\item \Type is not a link type, or it was not previously defined
\item \Type is not a child type of the container type of \Container
\item \Container instance was not previously created
\item \StartContainer instance was not previously created
\item \Key was already used by another \PajeStartLink
\end{itemize*}

Related type definition \Paje events:
\begin{itemize*}
\item \PajeDefineLinkType (Section~\ref{s.PajeDefineLinkType})
\end{itemize*}

%%%%%%%%%%%%%%%%%%%%%%%%%%%%%%%%%%%%%%%%%%%%%%%%%%%%%%%%%%%%%%%%%%%%%%%%%%%%%%%%
% PajeEndLink
%%%%%%%%%%%%%%%%%%%%%%%%%%%%%%%%%%%%%%%%%%%%%%%%%%%%%%%%%%%%%%%%%%%%%%%%%%%%%%%%
\subsubsection{PajeEndLink}
\label{s.PajeEndLink}
The \PajeEndLink event indicates the ending of a link between two
containers. This event must contain the fields \Time, \Type,
\Container, \EndContainer, \Value and \Key.

\begin{tabular}{|LLL|}
\hline
\multicolumn{3}{|T|}{\textbf{\PajeEndLink}}\\\hline
\textbf{Field Name} & \textbf{Field Type} & \textbf{Description}\\\hline
\Time          & date              & Time the link started\\
\Type          & string or integer & Type of link \\
\Container     & string or integer & Container that has the link \\
\EndContainer  & string or integer & Container where link ended \\
\Value         & string or integer & Value of link \\
\Key           & string or integer & Used to match to link start \\\hline
\end{tabular}

An invalid use of this event happens on the following cases:
\begin{itemize*}
\item \Time is not present, or is not a timestamp (see Section~\ref{s.NotionOfTime} for time considerations)
\item \Type is not a link type, or it was not previously defined
\item \Type is not a child type of the container type of \Container
\item \Container instance was not previously created
\item \EndContainer instance was not previously created
\item \Key was already used by another \PajeEndLink
\end{itemize*}

Related type definition \Paje events:
\begin{itemize*}
\item \PajeDefineLinkType (Section~\ref{s.PajeDefineLinkType})
\end{itemize*}

\section{Technical details}
\label{s.Technical}

\subsection{Comments and blank lines}

All lines starting by a {\bf \#} are ignored the same way blank lines
are. They can be anywhere in the trace file. Within the same line,
everything after a {\bf \#} is also ignored.

\subsection{Notion of time}
\label{s.NotionOfTime}

The notion of time accepted by \Paje is assumed to be in {\bf seconds}
since program start. So, the first timestamped event of a trace file
should have a time that equals to zero (or is close to zero). The
resolution of timestamps, in micro or nanoseconds, is left to the user
and handled by \Paje. The resolution is separated from the number of
seconds by a point.  Here are some examples of timestamps accepted by
\Paje:

\begin{verbatim}
0.0
10.45
12.345676
23.542831209
\end{verbatim}

Although the \Paje file format assumes timestamps in seconds, the user
is free to use another notion of time, such as logical
clocks. Analysis tools that read \Paje trace files must be configured
to accordingly.


\subsection{Using aliases}
\label{s.Alias}
The field \Alias is an optional argument in \Paje events that can be
used to decrease the size of trace files. When they are used, all
references to types (through \Type, \StartContainerType, and
\EndContainerType fields) and containers (through \Container,
\StartContainer, and \EndContainer fields) have to be done through the
value of their \Alias (the value provided when the type was defined or
the container created), and not through the \Name of types and
containers.

The use of aliases also enables for the definition of more than one
container with the same name, by using one different alias for each
container.

\subsection{String delimitation}
Fields of type string must be inside double quotes (") if they contain
space or tab characters, or if they are empty. Single quotes (') are
not used.

%% \section{Examples of \Paje trace files}
%% \label{s.Examples}

%% For example, to visualize the states of threads in a program, one must
%% first define the container types ``Program'' and ``Thread'' and the
%% entity type ``Thread State''.  One must also define the possible
%% values that the entities of type ``Thread State'' can assume (for
%% example, ``Executing'' and ``Blocked'').  Then, one must instantiate
%% the program creating a container of type ``Program'' (called ``Thread
%% Testing Program'', for example).  The threads of the program also have
%% to be instantiated; they are containers of type ``Thread'', called for
%% example ``Thread 1'' and ``Thread 2'', and contained in container
%% ``Thread Testing Program''.  Only then one is able to create
%% visualizable entities of type ``Thread State'', by means of events
%% that represent changes in state, contained either in ``Thread 1'' or
%% ``Thread 2''.

\section{Reference \Paje headers}
\label{s.PajeHeaders}

Two \Paje headers are provided, one without the \Alias field and one
with the \Alias field. They are the reference headers for \Paje trace
files.

\subsubsection*{The \Paje header without the \Alias field}
\begin{lstlisting}[basicstyle=\small,label=header1]
%EventDef PajeDefineContainerType 0
%       Type string
%       Name string
%EndEventDef
%EventDef PajeDefineVariableType 1
%       Type string
%       Name string
%       Color color
%EndEventDef
%EventDef PajeDefineStateType 2
%       Type string
%       Name string
%EndEventDef
%EventDef PajeDefineEventType 3
%       Type string
%       Name string
%       Color color
%EndEventDef
%EventDef PajeDefineLinkType 4
%       Type string
%       StartContainerType string
%       EndContainerType string
%       Name string
%EndEventDef
%EventDef PajeDefineEntityValue 5
%       EntityType string
%       Name string
%       Color color
%EndEventDef
%EventDef PajeCreateContainer 6
%       Time date
%       Type string
%       Container string
%       Name string
%EndEventDef
%EventDef PajeDestroyContainer 7
%       Time date
%       Type string
%       Name string
%EndEventDef
%EventDef PajeSetVariable 8
%       Time date
%       Type string
%       Container string
%       Value double
%EndEventDef
%EventDef PajeAddVariable 9
%       Time date
%       Type string
%       Container string
%       Value double
%EndEventDef
%EventDef PajeSubVariable 10
%       Time date
%       Type string
%       Container string
%       Value double
%EndEventDef
%EventDef PajeSetState 11
%       Time date
%       Container string
%       Type string
%       Value string
%EndEventDef
%EventDef PajePushState 12
%       Time date
%       Container string
%       Type string
%       Value string
%EndEventDef
%EventDef PajePopState 13
%       Time date
%       Container string
%       Type string
%EndEventDef
%EventDef PajeStartLink 14
%       Time date
%       Container string
%       Type string
%       StartContainer string
%       Value string
%       Key string
%EndEventDef
%EventDef PajeEndLink 15
%       Time date
%       Container string
%       Type string
%       EndContainer string
%       Value string
%       Key string
%EndEventDef
%EventDef PajeNewEvent 16
%       Time date
%       Container string
%       Type string
%       Value string
%EndEventDef
\end{lstlisting}

\subsubsection*{The \Paje header with the \Alias field}

\begin{lstlisting}[basicstyle=\small,label=header2]
%EventDef PajeDefineContainerType 0
%       Alias string
%       Type string
%       Name string
%EndEventDef
%EventDef PajeDefineVariableType 1
%       Alias string
%       Type string
%       Name string
%       Color color
%EndEventDef
%EventDef PajeDefineStateType 2
%       Alias string
%       Type string
%       Name string
%EndEventDef
%EventDef PajeDefineEventType 3
%       Alias string
%       Type string
%       Name string
%       Color color
%EndEventDef
%EventDef PajeDefineLinkType 4
%       Alias string
%       Type string
%       StartContainerType string
%       EndContainerType string
%       Name string
%EndEventDef
%EventDef PajeDefineEntityValue 5
%       Alias string
%       EntityType string
%       Name string
%       Color color
%EndEventDef
%EventDef PajeCreateContainer 6
%       Time date
%       Alias string
%       Type string
%       Container string
%       Name string
%EndEventDef
%EventDef PajeDestroyContainer 7
%       Time date
%       Type string
%       Name string
%EndEventDef
%EventDef PajeSetVariable 8
%       Time date
%       Type string
%       Container string
%       Value double
%EndEventDef
%EventDef PajeAddVariable 9
%       Time date
%       Type string
%       Container string
%       Value double
%EndEventDef
%EventDef PajeSubVariable 10
%       Time date
%       Type string
%       Container string
%       Value double
%EndEventDef
%EventDef PajeSetState 11
%       Time date
%       Container string
%       Type string
%       Value string
%EndEventDef
%EventDef PajePushState 12
%       Time date
%       Container string
%       Type string
%       Value string
%EndEventDef
%EventDef PajePopState 13
%       Time date
%       Container string
%       Type string
%EndEventDef
%EventDef PajeStartLink 14
%       Time date
%       Container string
%       Type string
%       StartContainer string
%       Value string
%       Key string
%EndEventDef
%EventDef PajeEndLink 15
%       Time date
%       Container string
%       Type string
%       EndContainer string
%       Value string
%       Key string
%EndEventDef
%EventDef PajeNewEvent 16
%       Time date
%       Container string
%       Type string
%       Value string
%EndEventDef
\end{lstlisting}

\section{ChangeLog}
\label{ChangeLog}

\subsection*{Version 1.2 (March 14th, 2012)}
\begin{itemize*}
\item Update to the latest format (as accepted by the \Paje
  visualization tool)
  \begin{itemize}
    \item the field \SourceContainer is deprecated, replaced by \StartContainer
    \item the field \DestContainer is deprecated, replaced by \EndContainer
    \item the field \ContainerType is deprecated, replaced by \Type
    \item the field \SourceContainerType is deprecated, replaced by \StartContainerType
    \item the field \DestContainerType is deprecated, replaced by \EndContainerType
    \item the fields \Shape, \Height and \Width (previously deprecated), are removed
  \end{itemize}
\end{itemize*}


\subsection*{Version 1.1 (February 24th, 2010)}
\begin{itemize*}
\item No longer accept interchangeable \Name and \Alias reference for types and containers
\end{itemize*}

\subsection*{Version 1.0 (March 22nd, 2003)}
\begin{itemize*}
\item Original proposal of the \Paje file format
\end{itemize*}

%\section{ToThink}
%Events are created with the event named
%``PajeNewEvent''\index{PajeNewEvent}.  Just like states, the values of
%events must be previously defined by ``PajeDefineEntityValue''.

\end{document}

Typically, the events of the first two classes are in the beginning of
a trace file, followed by events that instantiate containers, followed
by a large number of events creating entities.  The simulator does not
impose this order, events of these four classes can be mixed in the
trace file. The limitation is that an entity or a container cannot be
created before its type has been defined and its container created.

\subsubsection{PajeDefineEntityValue\index{PajeDefineEntityValue}}
\label{sec:entvaldef}

Contains fields ``Name'', ``EntityType'' and optionally ``Alias''.  Used to give
names to the possible values of an entity type.  ``Alias''
will represent the value ``Name'' that entities of type ``EntityType''
can have.  

\begin{figure}[htbp]
\begin{center}
\begin{tabular}{|LLL|}
\hline
\multicolumn{3}{|T|}{\textsf{\textbf{PajeDefineEntityValue}}}\\\hline
\textbf{Field Name} & \textbf{Field Type} & \textbf{Description}\\
\hline
Name          & string or integer & Value of entity \\
EntityType    & string or integer & Type of entity that can have this value \\
\hline
Alias         & string or integer & Alternative name of new value \\
Color         & color             & Color of entities of this value\\
\hline
\end{tabular}%
\end{center}%
\caption{Fields of PajeDefineEntityValue event}
\label{f:pajedefinevalue}
\end{figure}



\end{document}
