\documentclass[11pt,twoside]{report}
\usepackage{fullpage}
\usepackage{epsfig}
\usepackage{xspace}
\usepackage{alltt}

% \usepackage[francais]{babel}        % Pour Linux
\usepackage[latin1]{inputenc}
\usepackage[T1]{fontenc}

\setcounter{secnumdepth}{3}  %% pour num�roter les subsubsections
\setcounter{tocdepth}{3}     %% profondeur dans la table des mati�res

\usepackage{times}


\usepackage{amsmath}
\usepackage{pifont} % use of dingbats
\usepackage{array} % extension to the tabular env.
\usepackage{color} % for adding colors to things
\usepackage{colortbl} % for adding colors to tables

\newenvironment{captiontext}{\begin{quote} \footnotesize}{\end{quote}}
\newcommand{\ath}{\textsc{Atha\-pas\-can}\xspace}
\newcommand{\comment}[1]{}
\newlength{\figurewidth}\setlength{\figurewidth}{\textwidth}
%\addtolength{\figurewidth}{-\fboxsep}
%\addtolength{\figurewidth}{-\fboxsep}
\newsavebox{\figurebox}
\newcommand{\makefigure}[4]{%
   \begin{figure}[bt]\centering%
   \sbox{\figurebox}{#2}%
   \ifdim \wd\figurebox >\figurewidth%
      \resizebox{\figurewidth}{!}{\usebox{\figurebox}}%
   \else%
      \makebox[\figurewidth]{\usebox{\figurebox}}%
   \fi%
\vspace*{-3mm}
   \caption[#3]{\parbox[t]{11cm}{#3\newline\raggedright\protect\scriptsize #4}}%
\vspace*{2mm}
   \label{#1}%
   \end{figure}%
}
%% Tables
%
% Usage:
% \maketable{label}{figure command}{main caption}{secondary caption}
%
% there is a new column type for titles, T, to be used in \multicolumns,
% to put titles in reverse color, bold, sans serif, left aligned.
% there are also types R, L and C; they are like r, l, c but sans serif
%
\newlength{\tablewidth}\setlength{\tablewidth}{\textwidth}
\newcommand{\maketable}[4]{%
   \begin{table}[bt]\centering%
   \caption[#3]{#3\newline\footnotesize #4}\label{#1}%
   \sbox{\figurebox}{#2}%
   \ifdim \wd\figurebox >\tablewidth
      \resizebox{\tablewidth}{!}{\usebox{\figurebox}}%
   \else%
      \usebox{\figurebox}%
   \fi%
   \end{table}%
}
\newcolumntype{T}{>{\sffamily\bfseries\color{white}\columncolor[gray]{.2}}l}
\newcolumntype{R}{>{\sffamily}r}
\newcolumntype{L}{>{\sffamily}l}
\newcolumntype{C}{>{\sffamily}c}

\newcommand{\codefigurestart}{
  \begin{figure}[hbt]
    \begin{tabular}{|l|}
      \hline
      \begin{minipage}{\codewidth}
        \medskip\small
}

\newcommand{\codefigureend}[3]{
        \medskip
      \end{minipage}
      \\\hline
    \end{tabular}
%    \caption[#2]{#2\newline\footnotesize #3}%
    \caption[#2]{\parbox[t]{11cm}{#2\newline\raggedright\protect\scriptsize #3}}%
    \label{#1}%
  \end{figure}
}
\newlength{\codewidth}\setlength{\codewidth}{\figurewidth}
\addtolength{\codewidth}{-\fboxsep}
\addtolength{\codewidth}{-\fboxsep}
 % D�finitions de la th�se de Benhur

\makeindex

\title{Paj� trace file format}

\author{B. de Oliveira Stein\\ 
Departamento de Eletr\^onica e Computa\c{c}\~ao\\
Universidade Federal de Santa Maria - RS, Brazil.\\
Email: benhur@inf.UFSM.br\\
\and
J. Chassin de Kergommeaux\\
Laboratoire Informatique et Distribution (ID-IMAG)\\
ENSIMAG - antenne de Montbonnot,\\ ZIRST, 51, avenue Jean Kuntzmann\\
F-38330 Montbonnot Saint Martin, France \\ 
Email:Jacques.Chassin-de-Kergommeaux@imag.fr\\
http://www-apache.imag.fr/\~\/chassin
}

\begin{document}

\maketitle

\begin{abstract}
  
  Paj� is an interactive and scalable trace-based visualization tool
  which can be used for a large variety of visualizations including
  performance monitoring of parallel applications, monitoring the
  execution of processors in a large scale PC cluster or representing
  the behavior of distributed applications. Users of Paj� can tailor
  the visualization to their needs, without having to know any insight
  nor to modify any component of Paj�. This can be done by defining
  the type hierarchy of objects to be visualized as well as how these
  objects should be visualized. This feature allows the use of Paj�
  for a wide variety of visualizations such as the use of resources by
  applications in a large-size cluster or the behavior of distributed
  Java applications.  This report describes the trace format used by
  Paj�. Traces include three different kind of informations:
  definition of the formats of the event, definition of the type
  hierarchy of the objects to be visualized, definition of the formats
  of the events of the trace and a set of recorded events, complying
  with the format definition, to be used to build visualizations
  according to the type hierarchy.

 \textbf{Keywords:} performance debugging, visualization, MPI, pthread, 
parallel programming, self defined data format.

  
\end{abstract}

\tableofcontents

%%%%%%%%%%%%%%%%%%%%%%%%%%%%%%%%%%%%%%%%%%%%%%%%%%%%%%%%%%%%%%%%%%%%%%%%%%%%%
%\chapter{Introduction}
%%%%%%%%%%%%%%%%%%%%%%%%%%%%%%%%%%%%%%%%%%%%%%%%%%%%%%%%%%%%%%%%%%%%%%%%%%%%%
%%%%%%%%%%%%%%%%%%%%%%%%%%%%%%%%%%%%%%%%%%%%%%%%%%%%%%%%%%%%%%%%%%%%%%%%%%%%%%
% \chapter{Introduction}
%%%%%%%%%%%%%%%%%%%%%%%%%%%%%%%%%%%%%%%%%%%%%%%%%%%%%%%%%%%%%%%%%%%%%%%%%%%%%

This report defines the input data format used by the Pajé
visualization tool. Pajé is a versatile trace-based visualization tool
designed to help performance debugging of large-sized parallel
applications. From trace files, recorded during the execution of
parallel programs, Pajé builds a graphical representation of the
behavior of these programs, to help programmers identify their
``performance errors''. The main novelty of Pajé is an original
combination of three of the most desirable properties of visualisation
tools for parallel programs: extensibility, interactivity and
scalability. 

Scalability is the ability to represent the execution of
parallel programs executing during long periods on large-sized
systems; it is provided in Pajé by zooming and filtering
functionalities, both in space --- ability to synthesize the
information originating from several nodes of the system or to zoom in
one of these nodes --- and in time --- possibility to display  period
of time at various levels of detail. Interactivity is the ability to
interrogate visual objects --- events, thread states, communications,
etc. --- to obtain more details or check the source code whose
execution produced a given event; it is also the ability to move back
and forth in time or to zoom from a synthetic representation to a
detailed one or vice versa or to set or reset a filter. Extensibility
is the possibility ot extend the tool with new functionalities ---
visual representations, filters, etc. --- or to display new
programming models. Several characteristics of Pajé contribute to its
extensibility: careful modular design, independence of the
visualization modules from the programming model.

Key to the ability to build a visual representation of the behavior of
parallel programs, developed with various programming models, is the
\textit{genericity} of Pajé: ability to parameterize the tools with a
description of \textit{what} is to be represented and \textit{how}.
This description is provided in the trace file as a hierarchy of the
types of objects appearing in the visualization. The format of this
description as well as the format of the events of the trace are also
described in the trace file. The trace files\index{trace file} used as
input by Pajé thus contain four categories of data:
\begin{enumerate}
\item Description of the format of the generic instructions.
\item Generic instructions, describing the hierarchy of the types of
  objects appearing in the visualization.
\item Description of the format of the events recorded during the
  execution of the visualized program.
\item Events recorded during the execution of the program to be
  visualized.
\end{enumerate}

The aim of this technical report is to describe the Pajé trace data
format. The organization of the report is the following. After this
introduction, the extensibility and genericity of Pajé are described
in detail. The following section defines the Pajé data format and
gives an example of use before the conclusion.


%Pajé is an interactive visualization tool originally designed for
%displaying the execution of parallel applications where a
%(potentially) large number of communicating threads of various
%life-times execute on each node of a distributed memory parallel
%system.   To be easier
%to extend, Pajé was designed as a data-flow graph of modular
%components, most of them being independent of the semantics of the
%parallel programming model of the visualized parallel programs. In
%addition, application programmers can tailor the visualization to
%their needs, without having to know any insight nor to modify any
%component of Pajé. This can be done by defining the type hierarchy of
%objects to be visualized as well as how these objects should be
%visualized.


%%%%%%%%%%%%%%%%%%%%%%%%%%%%%%%%%%%%%%%%%%%%%%%%%%%%%%%%%%%%%%%%%%%%%%%%%%%%%
%\chapter{Extensibility of Paj�}
%\label{chap:paje}
%%%%%%%%%%%%%%%%%%%%%%%%%%%%%%%%%%%%%%%%%%%%%%%%%%%%%%%%%%%%%%%%%%%%%%%%%%%%%
%%%%%%%%%%%%%%%%%%%%%%%%%%%%%%%%%%%%%%%%%%%%%%%%%%%%%%%%%%%%%%%%%%%%%%%%%%%%%%
% \chapter{Extensibility of Pajé}
%%%%%%%%%%%%%%%%%%%%%%%%%%%%%%%%%%%%%%%%%%%%%%%%%%%%%%%%%%%%%%%%%%%%%%%%%%%%%

\section{Introduction}

The Pajé visualization tool described in this article\footnote{This
  chapter was published \emph{in: Euro-Par 2000 Parallel Processing,
    Proc. 6th International Euro-Par Conference}, A.~Bode, W.~Ludwig,
  T.~Karl, R.~Wism\"uller (r\'ed.), \emph{LNCS}, \emph{1900},
  Springer, p.~133--140, 2000.} was designed to allow programmers to
visualize the executions of parallel programs using a potentially
large number of communicating threads (lightweight processes) evolving
dynamically.  The visualization of the executions is an essential tool
to help tuning applications implemented using such a parallel
programming model.

Visualizing a large number of threads raises a number of problems such
as coping with the lack of space available on the screen to visualize
them and understanding such a complex display. The graphical displays
of most existing visualization tools for parallel programs
\cite{Heath:1991,upshot,Kranzlmueller:1996:PPV,PALLAS,pablo,ncstrl.gatech_cc//GIT-CC-95-21,ute}
show the activity of a fixed number of nodes and inter-nodes
communications; it is only possible to represent the activity of a
single thread of control on each of the nodes. It is of course
conceivable to use these systems to visualize the activity of
multi-threaded nodes, representing each thread as a node.  In this
case, the number of threads should be fairly limited and should not
vary during the execution of the program. These visualization tools
are therefore not adapted to visualize threads whose number varies
continuously and life-time is often short.  In addition, these tools
do not support the visualization of local thread synchronizations
using mutexes or semaphores.

Some tools were designed to display multithreaded
programs~\cite{HammondKev1995a,gthread}.  However, they support a
programming model involving a single level of parallelism within a
node, this node being in general a shared-memory multiprocessor. Our
programs execute on several nodes: within the same node, threads
communicate using synchronization primitives; however, threads
executing on different nodes communicate by message passing. Moreover,
compared to these systems, Pajé ought to represent a much larger
number of objects.

The most innovative feature of Pajé is to combine the characteristics
of interactivity and scalability with extensibility. In contrast with
passive visualization tools~\cite{Heath:1991,pablo} where parallel
program entities --- communications, changes in processor states, etc.
--- are displayed as soon as produced and cannot be interrogated, it
is possible to inspect all the objects displayed in the current screen
and to move back in time, displaying past objects again. Scalability
is the ability to cope with a large number of threads. Extensibility
is an important characteristic of visualization tools to cope with the
evolution of parallel programming interfaces and visualization
techniques. Extensibility gives the possibility to extend the
environment with new functionalities: processing of new types of
traces, adding new graphical displays, visualizing new programming
models, etc.

The interactivity and scalability characteristics of Pajé were
described in previous articles
\cite{ChassinS00,ChassinS:2000a,SteinC98}.  This article focuses on
the extensibility characteristics: modular design easing the addition
of new modules, semantics independent modules which allow them to be
used in a large variety of contexts and especially genericity of the
simulator component of Pajé which gives to application programmers the
ability to define what they want to visualize and how it must be done.

The organization of this article is the following. The next section
summarizes the main functionalities of Pajé.  The following section
describes the extensibility of Pajé before the conclusion.


\section{Outline of Pajé}
\label{sec:Pajé}

Pajé was designed to ease performance debugging of \ath programs by
visualizing their executions and because no existing visualization
tool could be used to visualize such multi-threaded programs.

\subsection{\ath: a thread-based parallel programming model}
\label{sec:ath}

Combining threads and communications is increasingly used to program
irregular applications, mask communication or I/O latencies, avoid
communication deadlocks, exploit shared-memory parallelism and
implement remote memory accesses
\cite{Fahringer:1995:UTD,FosterKT96,hicss95}.  The
\ath~\cite{ath0b-europar97} programming model was designed for
parallel hardware systems composed of shared-memory multi-processor
nodes connected by a communication network. It exploits two levels of
parallelism: inter-nodes parallelism and inner parallelism within each
of the nodes. The first type of parallelism is exploited by a fixed
number of system-level processes while the second type is implemented
by a network of communicating threads evolving dynamically. The main
functionalities of \ath are dynamic local or remote thread creation
and termination, sharing of memory space between the threads of the
same node which can synchronize using locks or semaphores, and
blocking or non-blocking message-passing communications between non
local threads, using ports. Combining the main functionalities of MPI
\cite{MPI} with those of \texttt{pthread} compliant libraries, \ath
can be seen as a ``thread aware'' implementation of MPI.

\subsection{Tracing of parallel programs}
\label{sec:tracing}

Execution traces are collected during an execution of the observed
application, using an instrumented version of the \ath\ library. A
non-intrusive, statistical method is used to estimate a precise global
time reference \cite{MailletT:1995}. The events are stored in local
event buffers, which are flushed when full to local event files.  The
collection of events into a single file is only done after the end of
the user's application to avoid interfering with it.  Recorded events
may contain source code information in order to implement source code
click-back --- from visualization to source code --- and click-forward
--- from source code to visualization --- in Pajé.

\subsection{Visualization of threads in Pajé}
\label{s-spacetime}

The visualization of the activity of multi-threaded nodes is mainly
performed in a diagram combining in a single representation the states
and communications of each thread(see figure~\ref{f-spacetime}) .
%
\makefigure{f-spacetime} {\includegraphics{FIG/spacetime-bact-e-2}}
{Visualization of an \ath program execution} {Blocked thread states
  are represented in clear color; runnable states in a dark color. The
  smaller window shows the inspection of an event.}

\makefigure{f-sema} {\includegraphics{FIG/sema-note-2}} {Visualization of
  semaphores} {Note the highlighting of a thread blocked state because
  the mouse pointer is over a semaphore blocked state, and the arrows
  that show the link between a ``V'' operation in a semaphore and the
  corresponding unblocking of a thread.}
%
The horizontal axis represents time while threads are displayed along
the vertical axis, grouped by node. The space allocated to each node
of the parallel system is dynamically adjusted to the number of
visualized threads of this node.  Communications are represented by
arrows while the states of threads are displayed by rectangles. Colors
are used to indicate either the type of a communication, or the
activity of a thread.  It is not the most compact or scalable
representation, but it is very convenient for analyzing detailed
threads relationship, load distribution and masking of communication
latency.  Pajé deals with the scalability problem of this
visualization by means of filters, discussed later in
section~\ref{s-filtering}.

The states of semaphores and locks are represented like the states of
threads: each possible state is associated with a color, and a
rectangle of this color is shown in a position corresponding to the
period of time when the semaphore was in this state. Each lock is
associated with a color, and a rectangle of this color is drawn close
to the thread that holds it (see figure~\ref{f-sema}).

\subsection{Interactivity}

Progresses of the simulation are entirely driven by user-controlled
time displacements: at any time during a simulation, it is possible to
move forward or backward in time, within the limits of the visualized
program execution.  In addition, Pajé offers many possible
interactions to programmers: displayed objects can be inspected to
obtain all the information available for them (see inspection window
in figure~\ref{f-spacetime}), identify related objects or check the
corresponding source code.  Moving the mouse pointer over the
representation of a blocked thread state highlights the corresponding
semaphore state, allowing an immediate recognition (see figure
\ref{f-sema}).  Similarly, all threads blocked in a semaphore are
highlighted when the pointer is moved over the corresponding state of
the semaphore.  From the visual representation of an event, it is
possible to display the corresponding source code line of the parallel
application being visualized.  Likewise, selecting a line in the
source code browser highlights the events that have been generated by
this line.

\subsection{Scalability: filtering of information and zooming capabilities}
\label{s-filtering}

It is not possible to represent simultaneously all the information
that can be deduced from the execution traces.  Screen space
limitation is not the only reason: part of the information may not be
needed all the time or cannot be represented in a graphical way or can
have several graphical representations.  Pajé offers several filtering
and zooming functionalities to help programmers cope with this large
amount of information to give users a simplified, abstract view of the
data. Accessing more detailed information can amount to exploding a
synthetic view into a more detailed view or getting to data that exist
but have not been used or are not directly related to the
visualization. Figure~\ref{f-filters-group} examplifies one of the
filtering facilities provided by Pajé where a single line represents
the number of active threads of a node and a pie graph the CPU
activity in the time slice selected in the space-time diagram (see
\cite{ChassinS00,ChassinS:2000a,Stein:1999}) for more details).

\makefigure{f-filters-group}
           {\includegraphics{FIG/busy+pie}} {CPU utilization} {Grouping
           the threads of each node to display the state of the whole
           system (lighter colors mean more active threads); the
           pie-chart shows the percentage of the selected time slice
           spent with each number of active threads in each node.}

\section{Extensibility}
\label{sec:extensibility}

Extensibility is a key property of a visualization tool. The
main reason is that a visualization tool being a very complex 
piece of software costly to implement, its lifetime ought to
be as long as possible. This will be possible only if the tool
can cope with the evolutions of parallel programming models ---
since this domain is still evolving rapidly --- and of the
visualization techniques. Several characteristics of Pajé were
designed to provide a high degree of extensibility: modular
architecture, flexibility of the visualization modules and genericity
of the simulation module.


\subsection{Modular architecture}
\label{sec:modular}

To favor extensibility, the architecture of Pajé is a data flow graph
of software modules or components (see figure~\ref{f-diagramme}). It
is therefore possible to add a new visualization component or adapt to
a change of trace format by changing the trace reader component
without changing the remaining of the environment.  This architectural
choice was inspired by Pablo \cite{pablo}, although the graph of Pajé
is not purely data-flow for interactivity reasons: it also includes
control-flow information, generated by the visualization modules to
process user interactions and triggering the flow of data in the graph
(see \cite{ChassinS00,ChassinS:2000a,Stein:1999} for more details on
the implementation of interactivity in Pajé).

\makefigure{f-diagramme}
           {\includegraphics{FIG/struct-obj-1b-e}} {Example data-flow
           graph} {The trace reader produces event objects from the
           data read from disk. These events are used by the simulator
           to produce more abstract objects, like thread states,
           communications, etc., traveling on the arcs of the
           data-flow graph to be used by the other components of the
           environment.}


\subsection{Flexibility of visualization modules}
\label{sec:flexibility}

The Pajé visualization components have no dependency whatsoever with
any parallel programming model. Prior to any visualization they
receive as input the description of the types of the objects to be
visualized as well as the relations between these objects and the way
these objects ought to be visualized (see
figure~\ref{fig:hierarchie1}). The only constraints are the
hierarchical nature of the type relations between the visualized
objects and the ability to place each of these objects on the
time-scale of the visualization. The hierarchical type description is
used by the visualization components to query objects from the
preceding components in the graph.

This type description can be changed to adapt to a new programming
model (see section~\ref{sec:genericity}) or during a visualization, to
change the visual representation of an object upon request from the
user. In addition to providing a high versatility for the
visualization components, this feature is used by the filtering
components. When a filter is dynamically inserted in a data-flow graph
--- for example between the simulation and visualization components of
figure~\ref{f-diagramme} to zoom from a detailed visualization to
obtain a more global view of the program execution such as
figure~\ref{f-filters-group} ---, it first sends a type description of
the hierarchy of objects to be visualized to the following components
of the data-flow graph.

\makefigure{fig:hierarchie1} {\includegraphics{FIG/hierarchya}} {Use of a
  simple type hierarchy} {The type hierarchy on the left-hand side of
  the figure defines the type hierarchical relations between the
  objects to be visualized and how how these objects should be
  represented: communications as arrows, thread events as triangles
  and thread states as rectangles.}
  
The type hierarchies used in Pajé are trees whose leaves are called
\textit{entities}\index{entities} and intermediate nodes
\textit{containers}\index{containers}. Entities are elementary objects
that can be displayed such as events, thread states or communications.
Containers are higher level objects used to structure the type
hierarchy\index{type hierarchy} (see figure~\ref{fig:hierarchie1}).
For example: all events occurring in thread~1 of node~0 belong to the
container ``thread-1-of-node-0''.


\subsection{Genericity of Pajé}
\label{sec:genericity}

The modular structure of Pajé as well as the fact that filter and
visualization components are independent of any programming model
makes it ``easy'' for tool developers to add a new component or extend
an existing one. These characteristics alone would not be sufficient
to use Pajé to visualize various programming models if the simulation
component were dependent on the programming model: visualizing a new
programming model would then require to develop a new simulation
component, which is still an important programming effort, reserved
for experienced tool developers.

On the contrary, the generic property of Pajé allows application
programmers to define \textit{what} they would like to visualize and
\textit{how} the visualized objects should be represented by Pajé.
Instead of being computed by a simulation component, designed for a
specific programming model such as \ath, the type hierarchy of the
visualized objects (see section~\ref{sec:flexibility}) can be defined
by inserting several definitions and commands in the trace file (see
format in chapter~\ref{chap:format}). If --- as it is the case with
the \ath-0 tracer and as it is assumed in this paper --- the tracer
(see section~\ref{sec:tracing}) can collect them, these definitions
and command can be inserted in the application program to be traced
and visualized.  The simulator uses these definitions to build a new
data type tree used to relate the objects to be displayed, this tree
being passed to the following modules of the data flow graph: filters
and visualization components.

\subsubsection{New data types definition.}

One function call is available to create new types of containers while
four can be used to create new types of entities which can be events,
states, links and variables. An ``event''\index{event} is an entity
representing an instantaneous action. ``States''\index{state} of
interest are those of containers.  A ``link''\index{link} represents
some form of connection between a source and a destination container.
A ``variable''\index{variable} stores the temporal evolution of the
successive values of a data associated with a container.
Table~\ref{t:defusertypes} contains the function calls that can be
used to define new types of containers and entities.
Figure~\ref{fig:hierarchie2} shows the effect of adding the
``threads'' container to the type hierarchy of
figure~\ref{fig:hierarchie1}.

\maketable{t:defusertypes}
{\small
\begin{tabular}{|>{\scshape}RL>{\scshape}L|}
\hline
\multicolumn{1}{|T}{Result}&
\multicolumn{1}{T}{Call}&
\multicolumn{1}{T|}{Parameters}  \\
\hline
ctype     & pajeDefineUserContainerType & ctype name                        \\
\hline
\hline
etype     & pajeDefineUserEventType     & ctype name                        \\
etype     & pajeDefineUserStateType     & ctype name                        \\
etype     & pajeDefineUserLinkType      & ctype name                        \\
etype     & pajeDefineUserVariableType  & ctype name                        \\
\hline
evalue    & pajeNewUserEntityValue      & etype name                        \\
\hline
\end{tabular}
} {Containers\index{container} and entities\index{entity} types
definitions} {The argument \textsc{\textsf{ctype}} is the type of the
father container of the newly defined type, in the type hierarchy (the
container ``Execution'' being always the root of the tree of types).}

%
\makefigure{fig:hierarchie2} {\includegraphics{FIG/hierarchyb}}{Adding
  a container to the type hierarchy\index{type hierarchy} of
  figure~\ref{fig:hierarchie1}}{}
  
\subsubsection{Data generation.}

Several functions can be used to create containers\index{container}
and entities\index{entity} whose types were defined using
table~\ref{t:defusertypes} primitives.  Specific functions are used to
create events, states (and embedded states using \textit{Push} and
\textit{Pop}), links --- each link being created by one source and one
destination calls, the coupling between them being performed by the
simulator when parameters \texttt{container, evalue} and \texttt{key}
of both source and destination calls match --- and change the values
of variables (see table~\ref{t:creationuser}).
%
\maketable{t:creationuser}
{\small
\begin{tabular}{|>{\scshape}RL>{\scshape}L|}
\hline
\multicolumn{1}{|T}{Result}&
\multicolumn{1}{T}{Call}&
\multicolumn{1}{T|}{Parameters}  \\
\hline
container & pajeCreateUserContainer     & ctype name in-container           \\
          & pajeDestroyUserContainer    & container                         \\
\hline
\hline
          & pajeUserEvent               & etype container evalue comment    \\
\hline
          & pajeSetUserState            & etype container evalue comment    \\
          & pajePushUserState           & etype container evalue comment    \\
          & pajePopUserState            & etype container comment           \\
\hline
          & pajeStartUserLink           & etype container srccontainer      \\
          &                           & \quad \quad evalue key comment    \\
          & pajeEndUserLink             & etype container destcontainer     \\
          &                           & \quad \quad evalue key comment    \\
\hline
          & pajeSetUserVariable         & etype container value comment     \\
          & pajeAddUserVariable         & etype container value comment     \\
\hline
\end{tabular}
} {Creation of containers and entities} {Calls to these functions are
inserted in the traced application to generate ``user events'' whose
processing by the Pajé simulator will use the type tree built from the
containers and entities types definitions done using the functions of
table~\ref{t:defusertypes}.}

In the example of figure~\ref{f:simpleprogramtraced}, a new event is
generated for each change of computation phase. This event is
interpreted by the Pajé simulator component to generate the
corresponding container state. For example the following call
indicates that the computation is entering in a ``Local computation''
phase: \\
{\small\tt\verb"  pajeSetUserState ( phase_state, node, local_phase, str_iter );"}\\
The second parameter indicates the container of the state (the
``node'' whose computation has just been changed).  The last parameter
is a comment that can be visualized by Pajé. In the example it is used
to display the current iteration value. The example program of
figure~\ref{f:simpleprogramtraced} includes the definitions and
creations of entities ``Computation phase'', allowing the visual
representation of an \ath program execution to be extended to
represent the phases of the computation.
Figure~\ref{f:simpleprogramvisu} includes two space-time diagrams
visualizing the execution of this example program, without and with
the definition of the new entities.
 
\codefigurestart{\scriptsize\alltt
 \textbf{unsigned phase_state, init_phase, local_phase, global_phase;
 phase_state  = pajeDefineUserStateType( A0_NODE, "Computation phase");
 init_phase   = pajeNewUserEntityValue( phase_state, "Initialization");
 local_phase  = pajeNewUserEntityValue( phase_state, "Local computation");
 global_phase = pajeNewUserEntityValue( phase_state,"Global computation");

 pajeSetUserState ( phase_state, node, init_phase, "" );}
 initialization();
 while (!converge) \{
     iter++;
     str_iter = itoa (iter);
     \textbf{pajeSetUserState ( phase_state, node, local_phase, str_iter );}
     local_computation();
     send (local_data);
     receive (remote_data);
     \textbf{pajeSetUserState ( phase_state, node, global_phase, str_iter );}
     global_computation();
 \}
}\codefigureend{f:simpleprogramtraced}
{Simplified algorithm of the example program}
{The five first lines written in bold face contain the generic
  instructions that have to be passed to Pajé through the trace file,
  to define a new type of state for the container \texttt{A0\_NODE}.
  Here it is assumed that the tracer is able to record these
  instructions in addition to the events of the program (
  \texttt{pajeSetUserState}).}
%
\makefigure{f:simpleprogramvisu} {\includegraphics{FIG/simpleprogram-2}}
{Visualization of the example program} {The second figure displays the
  entities ``Computation phases'' defined by the end-user. It is also
  possible to restrict the visualization to this information alone.}


\section{Conclusion}
\label{sec:conc}

Pajé provides solutions to interactively visualize the execution of
parallel applications using a varying number of threads communicating
by shared memory within each node and by message passing between
different nodes.  The most original feature of the tool is its unique
combination of extensibility, interactivity and scalability
properties. Extensibility means that the tool was defined to allow
tool developers to add new functionalities or extend existing ones
without having to change the rest of the tool. In addition, it is
possible to application programmers using the tool to define what they
wish to visualize and how this should be represented. To our knowledge
such a generic feature was not present in any previous visualization
tool for parallel programs executions.

The genericity property of Pajé was used to visualize \ath-1 programs
executions without having to perform any new development in Pajé.
\ath-1 is a high level parallel programming model where parallelism is
expressed by asynchronous task creations whose scheduling is performed
automatically by the run-time system \cite{a1-europar98,a1-pact98}.
The runtime system of \ath-1 is implemented using \ath. By extending
the type hierarchy defined for \ath and inserting few instructions to
the \ath-1 implementation, it was possible to visualize where the time
was spent during \ath-1 computations: computing the user program,
managing the task graph or scheduling the user-defined tasks.

Further developments include simplifying the generic description and
creation of visual objects, currently more complex when the generic
simulator is used instead of a specialized one. The generation of
traces for other thread-based programming models such as Java will
also be investigated to further validate the flexibility of Pajé.



%%%%%%%%%%%%%%%%%%%%%%%%%%%%%%%%%%%%%%%%%%%%%%%%%%%%%%%%%%%%%%%%%%%%%%%%%%%%%
\chapter{Definition of type hierarchies and trace event formats}
\label{chap:format}
%%%%%%%%%%%%%%%%%%%%%%%%%%%%%%%%%%%%%%%%%%%%%%%%%%%%%%%%%%%%%%%%%%%%%%%%%%%%%
%*
%   Copyright 1998, 1999, 2000, 2001, 2003, 2004 Benhur Stein
%   
%   This file is part of Paj�.
%
%   Paj� is free software; you can redistribute it and/or modify
%   it under the terms of the GNU General Public License as published by
%   the Free Software Foundation; either version 2 of the License, or
%   (at your option) any later version.
%
%   Foobar is distributed in the hope that it will be useful,
%   but WITHOUT ANY WARRANTY; without even the implied warranty of
%   MERCHANTABILITY or FITNESS FOR A PARTICULAR PURPOSE.  See the
%   GNU General Public License for more details.
%
%   You should have received a copy of the GNU General Public License
%   along with Foobar; if not, write to the Free Software
%   Foundation, Inc., 59 Temple Place, Suite 330, Boston, MA  02111-1307  USA
%/
%%%%%%%%%%%%%%%%%%%%%%%%%%%%%%%%%%%%%%%%%%%%%%%%%%%%%%%%%%%%%%%%%%%%%%%%%%%%%
% \chapter{Definition of type hierarchies and trace event formats}
%%%%%%%%%%%%%%%%%%%%%%%%%%%%%%%%%%%%%%%%%%%%%%%%%%%%%%%%%%%%%%%%%%%%%%%%%%%%%

% T, R, L and C already in definitions.tex
%\newcolumntype{T}{>{\sffamily\bfseries\color{white}\columncolor[gray]{.2}}l}
%\newcolumntype{R}{>{\sffamily}r}
%\newcolumntype{L}{>{\sffamily}l}
%\newcolumntype{C}{>{\sffamily}c}
\newcolumntype{P}{>{\sffamily}p{5cm}}

\section{Introduction}
\label{sec:traceintro}

A visualization constructed by Paj� is composed of objects organized
according to a tree type hierarchy whose nodes are called
\emph{containers}\index{container} and leaves
\emph{entities}\index{entity}% (see \S\ref{sec:genericity})
. The Paj�
data format is self-defined, although it does not comply with the
SDDF\index{SDDF} format used by Pablo \cite{sddf}. There exists a
``meta-format'' used to define:
\begin{itemize}
\item The format of the instructions defining containers and entities.
\item The format of the events recorded during the executions of
  parallel programs.
\end{itemize}

These definitions are usually inserted in trace files. They can even
be inserted in the observed programs source files, provided that the
tracers used to record the events of these programs are able to
capture a new definition as a ``user-defined'' event.% (see for example
%figure~\ref{f:simpleprogramtraced}).

Using these definitions, it is possible to define a hierarchy of
containers and entities adapted for a given programming model or
language. Definitions of type hierarchies as well as instructions and
events formats constitute a specialisation of the ``generic'' Paj�
visualization tool: it has been used so far to visualize distributed
applications written in Java \cite{OttogaliOSCV:2001} or help to
perform system monitoring on large sized clusters
\cite{GuilloudCAS:2001}. 

The organization of this chapter is the following. The next section
defines the meta format of Paj� used to define the format of type
definition instructions and events. The following sections describe
how containers and entities are defined and created. The next section
is dedicated to the trace events: self definition, recording. The last
section of this chapter contains a complete example of use of the Paj�
data format.

\section{Meta format of Paj�}
\label{sec:file}

A trace file is composed of events.
An event can be seen as a table composed of named fields, as shown in figure~\ref{f:event:table}. 
The first event in the figure can represent the sending of a message containing 320 bytes by process 5 to process 3, containing 320 bytes by process 5 to process 3, containing 320 bytes by process 5 to process 3, containing 320 bytes by process 5 to process 3, 3.233222 seconds after the process started executing.
The second event shows that process 5 unblocked at time 5.123002, and that this happened while executing line 98 of file sync.c.
Each event has some fields, each of them composed of a name, a type and a value. Generally, there are lots of similar events in a trace file (lots of ``SendMessage'' events, all with the same fields); a typical trace file contains thousands of events of tens of different types.
Usually, events of the same type have the same fields.
It is therefore wise, in order to reduce the trace file size, not to put the
information that is common to many events in each of those events.
The most common solution is to put only the type of each event and the values of its fields in the trace file. Information on what event types exist and the fields that constitute each of these event types being kept elsewhere.
In some trace file formats, this information is hardcoded in the trace generator and trace reader, making the trace structure hard to change in order to incorporate new types of events, new data in existing events or to remove unused or unknown data from those events.

\begin{figure}[htbp]
\begin{center}
\begin{tabular}{|>{\bf}rll|}
\hline
\textbf{Field Name} & \textbf{Field Type} & \textbf{Field Value} \\
\hline
EventName     & string    & SendMessage \\
Time          & timestamp & 3.233222    \\
ProcessId     & integer   & 5           \\
Receiver      & integer   & 3           \\
Size          & integer   & 320         \\
\hline
\end{tabular}
\quad\quad
\begin{tabular}{|>{\bf}rll|}
\hline
\textbf{Field Name} & \textbf{Field Type} & \textbf{Field Value} \\
\hline
EventName     & string    & UnblockProcess \\
Time          & timestamp & 5.123002    \\
ProcessId     & integer   & 5           \\
FileName      & string    & sync.c      \\
LineNumber    & integer   & 98          \\
\hline
\end{tabular}
\end{center}
\caption{Examples of events}
\label{f:event:table}
\end{figure}

A Paj� trace file is self defined, meaning that the event definition information is put inside the trace file itself, much like the SDDF file format used by the Pablo visualization tool \cite{sddf}.
The file is constituted of two parts: the definition of the events at the beginning of the file followed by the events themselves.
The definition of events contains the name of each event type and the names and types of each field.
The second part of the trace file contains the events, with the values associated to each field, in the same order as in the definition.
The correspondence of an event with its definition is made by means of a number, that must be unique for each event description; this number appears in an event definition and at the beginning of each event contained in the trace file.

The event definition part of a Paj� trace file follows the following format:
\begin{itemize}
\item all the lines start with a `\%' character;
\item each event definition starts with a \%EventDef\index{EventDef}
  line and terminates with a \%EndEventDef\index{EndEventDef} line;
\item the \%EventDef line contains the name and the unique number of an
  event type.  The number (an integer) will be used to identify the event
  in the second part of the trace file. The choice of this number is
  left to the user. The numbers given in the definitions below (see
%  \S\ref{sec:containers} and \S\ref{sec:entities}
  \S\ref{sec:example}) are thus
  \textbf{arbitrary}. The name of the event will be put in a field called
  ``PajeEventName''. There cannot be another field called so. The name is used
  to identify the type of an event;
\item the fields of an event are defined between the \%EventDef and
  the \%EndEventDef lines, one field per line, with the name
  of the field followed by its type (see below).
\end{itemize}

The structure of the two events of figure~\ref{f:event:table} are shown in figure~\ref{f:event:def}.

\begin{figure}
\begin{center}
\begin{minipage}{5.6cm}
\begin{verbatim}
%EventDef SendMessage 21
%   Time       date
%   ProcessId  int
%   Receiver   int
%   Size       int
%EndEventDef
\end{verbatim}
\end{minipage}
\quad\quad
\begin{minipage}{5.6cm}
\begin{verbatim}
%EventDef UnblockProcess 17
%   Time       date
%   ProcessId  int
%   LineNumber int
%   FileName   string
%EndEventDef
\end{verbatim}
\end{minipage}
\end{center}
\caption{Examples of event definitions}
\label{f:event:def}
\end{figure}

The type of a field can be one of the following:
\begin{description}
  \item [date:] for fields that represent dates\index{date}.
                It's a double precision floating-point number, usually meaning seconds since program start;
  \item [int:] for fields containing integer numeric values;
  \item [double:] for fields containing floating-point values;
  \item [hex:] for fields that represent addresses, in hexadecimal;
  \item [string:] for strings of characters.
  \item [color:] for fields that represent colors. A color is a sequence of
                 three floating-point numbers between 0 and 1, inside double 
                 quotes (").
                 The three numbers are the values of red, green and blue
                 components.
\end{description}

%Most events are dated. If that is the case, it must be defined with a field
named ``Time'' of type ``date'', for
%the date of generation of the event. 

The second part of the trace file contains one event per line, whose 
fields are separated by spaces or tabs, the first field being the number that
identifies the event type, followed by the other fields, in the same order that
they appear in the
definition of the event. 
Fields of type string must be inside double quotes (") if they contain space or
tab characters, or if they are empty.

For example, the two events of figure~\ref{f:event:table} are shown in figure~\ref{f:event}.
\begin{figure}
\begin{center}
\begin{minipage}{5cm}
\begin{verbatim}
21 3.233222 5 3 320
17 5.123002 5 98 sync.c
\end{verbatim}
\end{minipage}
\end{center}
\caption{Examples of events}
\label{f:event}
\end{figure}

In Paj�, event numbers are used only as a means to find the definition
of an event; they are discarded as soon as an event is read.
After being read, events are identified by their names.
Two different definitions can have the same name (and different numbers), making it possible to have, in the same trace file, two events of the same type containing different fields.
We use this feature to optionally include the source file identification in some
events. The ``UnblockProcess'' event in the examples above could also be defined without the fields FileName and LineNumber, for use in places where this information is not known or not necessary.

\section{Events treated by the Paj� simulator} %{Paj� "generic" events}
\label{sec:generic}

A Paj� visualization is best described as a typed hierarchy of objects
organized as a tree. Elementary objects are the leaves of the tree and
called ``entities'' while intermediate nodes of the tree are named
``containers''. 
Entities are the objects that can be visualized in Paj�'s space-time diagram,
while containers organize the space where those entities are displayed.

Paj� includes a simulator module which builds this hierarchical data
structure from the elementary event records of the trace files. 
Paj� has no
predefined containers or entities. 
Before an entity can be created and visualized, a hierarchy of container and
entity types must be defined, and containers must be instantiated.

For example, to visualize the states of threads in a program, one must
first define the container types ``Program'' and ``Thread'' and the entity
type ``Thread State''.  One must also define the possible values that
the entities of type ``Thread State'' can assume (for example,
``Executing'' and ``Blocked'').  Then, one must instantiate the program
creating a container of type ``Program'' (called ``Thread Testing
Program'', for example).  The threads of the program also have to be
instantiated; they are containers of type ``Thread'', called for example
``Thread 1''
and ``Thread 2'', and contained in container ``Thread Testing Program''.
Only then one is able to create visualizable entities of type ``Thread
State'', by means of events that represent changes in state, contained
either in ``Thread 1'' or ``Thread 2''.


The events that the Paj� simulator understands can be divided into four classes:
\begin{itemize}

\item events to define types of containers;
\item events to define types of entities and possible
values that entities can have; 
\item events to instantiate and destroy containers;
\item events to create visualizable entities.
\end{itemize}

Typically, the events of the first two classes are in the beginning of a trace
file, followed by events that instantiate containers, followed by a large number of events creating entities.
The simulator does
not impose this order, events of these four classes can be mixed in the trace
file. The limitation is that an entity or a container cannot be created before
its type has been defined and its container created.

The four classes of events are discussed in the following sessions.

\subsection{Definition of types of Containers}
\label{sec:contype}

Containers types are defined with events named ``PajeDefineContainerType''.

\subsubsection*{PajeDefineContainerType\index{PajeDefineContainerType}}

Events of this type (see figure~\ref{f:pajedefinecontainertype}) must contain the fields  ``Name'' and ``ContainerType''.
It defines a new container type called ``Name'', contained by a previously
defined container type ``ContainerType'' (or the special container type ``0'' or
``/'', if this container type is
the top of the container hierarchy).
Optionally this event can contain a field ``Alias'' with an alias name to be
used to identify this container. Aliases are usually short strings used when the
container name is too big and its use throughout the trace file would increase
the file's size.
When an alias is used in a definition, it must also be used in later references to the container type being defined.
When an alias is not used, a container type must be later referenced by its name.
The use of aliases allows for the definition of more than one container with the same name (and different aliases).

\begin{figure}[htbp]
\begin{center}
\begin{tabular}{|LLL|}
\hline
\multicolumn{3}{|T|}{\textsf{\textbf{PajeDefineContainerType}}}\\\hline
\textbf{Field Name} & \textbf{Field Type} & \textbf{Description}\\
\hline
Name          & string or integer & Name of new container type\\
ContainerType & string or integer & Parent container type\\
\hline
Alias         & string or integer & Alternative name of new container type\\
\hline
\end{tabular}%
\end{center}%
\caption{Fields of PajeDefineContainerType event}
\label{f:pajedefinecontainertype}
\end{figure}

For the example, one could need two events (see
%in section~\ref{s:generic}, one would need two events (see
figure~\ref{f:definecontainerexample} to indicate
that a ``Program'' contains ``Thread''s:

\begin{figure}[htbp]
\begin{center}
\begin{tabular}{|LL|}
\hline
\textbf{Field Name} & \textbf{Field Value} \\
\hline
PajeEventName & PajeDefineContainerType \\
Name          & Program\\
ContainerType & /\\
Alias         & P\\
\hline
\end{tabular}%
\quad%\quad
\begin{tabular}{|LL|}
\hline
\textbf{Field Name} & \textbf{Field Value} \\
\hline
PajeEventName & PajeDefineContainerType \\
Name          & Thread\\
ContainerType & P \emph{or} Program\\
Alias         & T\\
\hline
\end{tabular}%
\end{center}%
\caption{Examples of PajeDefineContainerType events}
\label{f:definecontainerexample}
\end{figure}




\subsection{Creation and destruction of containers}
\label{sec:instant}

Containers are created using the ``PajeCreateContainer'' event, and destroyed using
the ``PajeDestroyContainer'' event.

\subsubsection*{PajeCreateContainer\index{PajeCreateContainer}}

This event (see figure~\ref{f:pajecreatecontainer} must have the fields
``Time'', ``Name'', ``Type'' and ``Container''. Optionally
it can have a field named ``Alias''. The simulation of this event instantiates,
in the simulation time ``Time'', a
new container named ``Name'', of type
``Type'', contained in the preexisting
container ``Container''.
The field ``Type'' must have a value corresponding to the ``Name'' or ``Alias''
of a previous
PajeDefineContainerType event. The field ``Container'' must have a value
corresponding to the ``Name'' or ``Alias''
of a previous PajeCreateContainer event (or ``0'' or ``/'', if on top of the hierarchy).
This new container can be referenced in future events by the value of its
``Name'' or, if it has an ``Alias'' field, by it alias.

\begin{figure}[htbp]
\begin{center}
\begin{tabular}{|LLL|}
\hline
\multicolumn{3}{|T|}{\textsf{\textbf{PajeCreateContainer}}}\\\hline
\textbf{Field Name} & \textbf{Field Type} & \textbf{Description}\\
\hline
Time          & date              & Time of creation of container \\
Name          & string or integer & Name of new container \\
Type          & string or integer & Type of new container \\
Container     & string or integer & Parent of new container \\
\hline
Alias         & string or integer & Alternative name of new container \\
\hline
\end{tabular}%
\end{center}%
\caption{Fields of PajeCreateContainer event}
\label{f:pajecreatecontainer}
\end{figure}

Figure~\ref{f:createcontainerexample} shows the events necessary to create 
the containers ``Thread Testing
Program'' of type ``Program'' and ``Thread 1'' and ``Thread 2'' of
type ``Thread'', contained by ``Thread Testing Program''.%, from the example in section~\ref{s:generic}.

\begin{figure}[htbp]
\begin{center}
\begin{tabular}{|LL|}
\hline
\textbf{Field Name} & \textbf{Field Value} \\
\hline
PajeEventName & PajeCreateContainer \\
Time          & 0\\
Name          & "Thread Testing Program"\\
Container     & /\\
Type          & P\\
Alias         & TTP\\
\hline
\end{tabular}%

%\quad%\quad
\begin{tabular}{|LL|}
\hline
\textbf{Field Name} & \textbf{Field Value} \\
\hline
PajeEventName & PajeCreateContainer \\
Time          & 0.986789\\
Name          & "Thread 1"\\
Container     & TTP\\
Type          & T\\
Alias         & T1\\
\hline
\end{tabular}%
\quad%\quad
\begin{tabular}{|LL|}
\hline
\textbf{Field Name} & \textbf{Field Value} \\
\hline
PajeEventName & PajeCreateContainer \\
Time          & 1.012332\\
Name          & "Thread 2"\\
Container     & TTP\\
Type          & T\\
Alias         & T2\\
\hline
\end{tabular}%
\end{center}%
\caption{Examples of PajeCreateContainer events}
\label{f:createcontainerexample}
\end{figure}


\subsubsection*{PajeDestroyContainer\index{PajeDestroyContainer}}

Containers can be destroyed using the event named
``PajeDestroyContainer'' with fields ``Time'', ``Name'' and ``Type'' (see
figure~\ref{f:pajedestroycontainer}.
After simulating this event, the container named (or aliased) ``Name'' of type
``Type'' will be marked as being destroyed at time ``Time''.

\begin{figure}[htbp]
\begin{center}
\begin{tabular}{|LLL|}
\hline
\multicolumn{3}{|T|}{\textsf{\textbf{PajeDestroyContainer}}}\\\hline
\textbf{Field Name} & \textbf{Field Type} & \textbf{Description}\\
\hline
Time          & date              & Time of destruction of container \\
Name          & string or integer & Name of container \\
Type          & string or integer & Type of container \\
\hline
\end{tabular}%
\end{center}%
\caption{Fields of PajeDestroyContainer event}
\label{f:pajedestroycontainer}
\end{figure}

For example, if ``Thread 1'' finishes execution at time 4.34565. it can be
represented by the event in figure~\ref{f:destroycontainerexample}.


\begin{figure}[htbp]
\begin{center}
\begin{tabular}{|LL|}
\hline
\textbf{Field Name} & \textbf{Field Value} \\
\hline
PajeEventName & PajeDestroyContainer \\
Time          & 4.34565\\
Name          & "Thread 1"\\
Type          & T\\
\hline
\end{tabular}%
\end{center}%
\caption{Example of PajeDestroyContainer event}
\label{f:destroycontainerexample}
\end{figure}



\subsection{Definitions of types of entities}
\label{sec:entypedef}

Entities are the leaves of the type hierarchy tree of a Paj�
specialization. There exist four types of entities:
\begin{itemize}
\item \textbf{events}\index{event}, used to represent an event that happened in a
certain point in time, usually displayed as triangles in Paj�'s space-time
diagram;
\item \textbf{states}\index{state}, used to represent the fact that a certain
container was in a determined state during a certain amount of time, usually
displayed as rectangles in Paj�'s space-time diagram;
\item \textbf{links}\index{link}, used to represent a relation between two
containers that started in a certain time and finished in a possibly different
time (for example, a communication between two nodes), usually displayed as arrows; and
\item \textbf{variables}\index{variable}, used to represent the evolution in
time of a
certain value associated to a container, displayed as
graphs in the space-time diagram.
\end{itemize}

An event of type event, state or link can have a value associated with it, and
all possible values must be defined before an event with this value can be
created.
There are four different events to create an entity type in Paj�,
``PajeDefineEventType'', ``PajeDefineStateType'', ``PajeDefineLinkType'' and
``PajeDefineVariableType'' and one event to define a possible value of an
entity, ``PajeDefineEntityValue''.

\subsubsection*{PajeDefineEventType\index{PajeDefineEventType}}

Entities of this new type represent a remarkable type of event
recorded during the visualized executions and are displayed as
triangles in the space-time diagram.  Event types are defined with the
"PajeDefineEventType" event.
This event (see figure~\ref{f:pajedefineevent}) contains the
fields ``Name'' and ``ContainerType''.  It defines a
new event entity type called ``Name'', subtype of the previously defined
container type ``ContainerType''. Optionally it can have a field named ``Alias''
to have an alternative way to identify this type of entity.

\begin{figure}[htbp]
\begin{center}
\begin{tabular}{|LLL|}
\hline
\multicolumn{3}{|T|}{\textsf{\textbf{PajeDefineEventType}}}\\\hline
\textbf{Field Name} & \textbf{Field Type} & \textbf{Description}\\
\hline
Name          & string or integer & Name of new entity type \\
ContainerType & string or integer & Type of container of entity\\
\hline
Alias         & string or integer & Alternative name of new entity type \\
Shape         & string            & Name of shape used to represent entities\\
Height        & integer           & Height of shape, in points\\
Width         & integer           & Width of shape, in points\\
\hline
\end{tabular}%
\end{center}%
\caption{Fields of PajeDefineEventType event}
\label{f:pajedefineevent}
\end{figure}

\subsubsection*{PajeDefineStateType\index{PajeDefineStateType}}

Entities of this new type will represent ``states'', and are displayed
as rectangles in the space-time diagram.  The definition contains (see
figure~\ref{f:pajedefinestate}) the
fields ``Name'' and ``ContainerType''. Optionally, it can have a field
``Alias''.

\begin{figure}[htbp]
\begin{center}
\begin{tabular}{|LLL|}
\hline
\multicolumn{3}{|T|}{\textsf{\textbf{PajeDefineStateType}}}\\\hline
\textbf{Field Name} & \textbf{Field Type} & \textbf{Description}\\
\hline
Name          & string or integer & Name of new entity type \\
ContainerType & string or integer & Type of container of entity\\
\hline
Alias         & string or integer & Alternative name of new entity type \\
Shape         & string            & Name of shape used to represent entities\\
Height        & integer           & Height of shape, in points\\
\hline
\end{tabular}%
\end{center}%
\caption{Fields of PajeDefineStateType event}
\label{f:pajedefinestate}
\end{figure}

In the example of \S\ref{sec:instant}, the event in
figure~\ref{f:definestateexample} could be used to define the entity type that will represent the states of
the threads of the program.

\begin{figure}[htbp]
\begin{center}
\begin{tabular}{|LL|}
\hline
\textbf{Field Name} & \textbf{Field Value} \\
\hline
PajeEventName & PajeDefineStateType \\
Name          & "Thread State"\\
Alias         & S\\
ContainerType & Thread\\
\hline
\end{tabular}%
\end{center}%
\caption{Example of PajeDefineStateType event}
\label{f:definestateexample}
\end{figure}


\subsubsection*{PajeDefineVariableType\index{PajeDefineVariableType}}

Entities of this new type represent variables, whose evolutions are to
be visualized as graphs during the execution of parallel programs.
Variables are created with the \texttt{PajeDefineVariableType} event (see
figure~\ref{f:pajedefinevariable}), containing fields ``Name'',
``ContainerType'' and optionally ``Alias''.
Their value represent an attribute of a container, whose value (a
double) is set by the \texttt{PajeSetVariable} event.

\begin{figure}[htbp]
\begin{center}
\begin{tabular}{|LLL|}
\hline
\multicolumn{3}{|T|}{\textsf{\textbf{PajeDefineVariableType}}}\\\hline
\textbf{Field Name} & \textbf{Field Type} & \textbf{Description}\\
\hline
Name          & string or integer & Name of new entity type \\
ContainerType & string or integer & Type of container of entity\\
\hline
Alias         & string or integer & Alternative name of new entity type \\
Height        & integer           & Height of shape, in points\\
\hline
\end{tabular}%
\end{center}%
\caption{Fields of PajeDefineVariableType event}
\label{f:pajedefinevariable}
\end{figure}

\subsubsection*{PajeDefineLinkType\index{PajeDefineLinkType}}

Links are used to display a directed link between two containers such
as a communication or the identification of a reaction in a container
corresponding to an action on another one.  The source and destination
containers must have a common ancestral in the container hierarchy
(identified by ``Container'' in the events below). Links are usually displayed
as arrows.

\begin{figure}[htbp]
\begin{center}
\begin{tabular}{|LLL|}
\hline
\multicolumn{3}{|T|}{\textsf{\textbf{PajeDefineLinkType}}}\\\hline
\textbf{Field Name} & \textbf{Field Type} & \textbf{Description}\\
\hline
Name          & string or integer & Name of new link type \\
ContainerType & string or integer & Type of common ancestral container \\
SourceContainerType & string or integer & Type of source container of link\\
DestContainerType & string or integer & Type of destination container of link\\
\hline
Alias         & string or integer & Alternative name of new link type \\
Shape         & string            & Name of shape used to represent entities\\
\hline
\end{tabular}%
\end{center}%
\caption{Fields of PajeDefineLinkType event}
\label{f:pajedefinelink}
\end{figure}



\subsubsection*{PajeDefineEntityValue\index{PajeDefineEntityValue}}
\label{sec:entvaldef}

Contains fields ``Name'', ``EntityType'' and optionally ``Alias''.  Used to give
names to the possible values of an entity type.  ``Alias''
will represent the value ``Name'' that entities of type ``EntityType''
can have.  

\begin{figure}[htbp]
\begin{center}
\begin{tabular}{|LLL|}
\hline
\multicolumn{3}{|T|}{\textsf{\textbf{PajeDefineEntityValue}}}\\\hline
\textbf{Field Name} & \textbf{Field Type} & \textbf{Description}\\
\hline
Name          & string or integer & Value of entity \\
EntityType    & string or integer & Type of entity that can have this value \\
\hline
Alias         & string or integer & Alternative name of new value \\
Color         & color             & Color of entities of this value\\
\hline
\end{tabular}%
\end{center}%
\caption{Fields of PajeDefineEntityValue event}
\label{f:pajedefinevalue}
\end{figure}

In the example started in \S\ref{sec:instant}, ``Thread State''s can be
``Executing'' or ``Blocked'', as shown in figure~\ref{f:definevalueexample}.

\begin{figure}[htbp]
\begin{center}
\begin{tabular}{|LL|}
\hline
\textbf{Field Name} & \textbf{Field Value} \\
\hline
PajeEventName & PajeDefineEntityValue \\
Name          & Executing\\
Alias         & E\\
EntityType    & S\\
Color         & "0 1 0"\\
\hline
\end{tabular}%
\quad\begin{tabular}{|LL|}
\hline
\textbf{Field Name} & \textbf{Field Value} \\
\hline
PajeEventName & PajeDefineEntityValue \\
Name          & Blocked\\
Alias         & B\\
EntityType    & S\\
Color         & "0.9 0 0.1"\\
\hline
\end{tabular}%
\end{center}%
\caption{Example of PajeDefineEntityValue event}
\label{f:definevalueexample}
\end{figure}

\subsection{Creation of visualizable entities}
\label{sec:creation}

There are different events to create entities of each possible type
(states, events, variables or links).
In events that create entities, the optional fields named "FileName" and "LineNumber" can be used to relate the created event to a position in a file, that can be obtained in Paj� during the inspection of the entity.
These events can also have the optional field named "RelationKey", to group entities that are somehow related to each other. All entities with the same key are highlighted in the space-time diagram when the mouse cursor is over one of them.

\subsubsection{States}

There are events to change a state
("PajeSetState")\index{PajeSetState}, to push a state, saving the old
state ("PajePushState")\index{PajePushState}, and to pop the
previously saved state ("PajePopState")\index{PajePopState}.

\begin{figure}[htbp]
\begin{center}
\begin{tabular}{|LLL|}
\hline
\multicolumn{3}{|T|}{\textsf{\textbf{PajeSetState}}}\\\hline
\textbf{Field Name} & \textbf{Field Type} & \textbf{Description}\\
\hline
Time          & date              & Time the state changed \\
Type          & string or integer & Type of state \\
Container     & string or integer & Container whose state changed \\
Value         & string or integer & Value of new state of container \\
\hline
\end{tabular}

\begin{tabular}{|LLL|}
\hline
\multicolumn{3}{|T|}{\textsf{\textbf{PajePushState}}}\\\hline
\textbf{Field Name} & \textbf{Field Type} & \textbf{Description}\\
\hline
Time          & date              & Time the state changed \\
Type          & string or integer & Type of state \\
Container     & string or integer & Container whose state changed \\
Value         & string or integer & Value of new state of container \\
\hline
\end{tabular}

\begin{tabular}{|LLL|}
\hline
\multicolumn{3}{|T|}{\textsf{\textbf{PajePopState}}}\\\hline
\textbf{Field Name} & \textbf{Field Type} & \textbf{Description}\\
\hline
Time          & date              & Time the state changed \\
Type          & string or integer & Type of state \\
Container     & string or integer & Container whose state changed \\
\hline
\end{tabular}%
\end{center}%
\caption{Fields of state changing events}
\label{f:pajesetstate}
\end{figure}

For example, if "Thread 1" blocks at time 2.34567 and unblocks at time
2.456789, the trace file could contain the events shown in
figure~\ref{f:setstateexample}.

\begin{figure}[htbp]
\begin{center}
\begin{tabular}{|LL|}
\hline
\textbf{Field Name} & \textbf{Field Value} \\
\hline
PajeEventName & PajeSetState \\
Time          & 2.34567\\
Type          & "Thread State" \\
Container     & "Thread 1"\\
Value         & Blocked \\
\hline
\end{tabular}%
\quad\begin{tabular}{|LL|}
\hline
\textbf{Field Name} & \textbf{Field Value} \\
\hline
PajeEventName & PajeSetState \\
Time          & 2.456789\\
Type          & S \\
Container     & T1\\
Value         & E \\
\hline
\end{tabular}%
\end{center}%
\caption{Example of PajeSetState event}
\label{f:setstateexample}
\end{figure}

\subsubsection{Events}

Events are created with the event named ``PajeNewEvent''\index{PajeNewEvent}.
Just like states, the values of events must be previously
defined by ``PajeDefineEntityValue''.

\begin{figure}[htbp]
\begin{center}
\begin{tabular}{|LLL|}
\hline
\multicolumn{3}{|T|}{\textsf{\textbf{PajeNewEvent}}}\\\hline
\textbf{Field Name} & \textbf{Field Type} & \textbf{Description}\\
\hline
Time          & date              & Time the event happened \\
Type          & string or integer & Type of event \\
Container     & string or integer & Container that produced event \\
Value         & string or integer & Value of new event \\
\hline
\end{tabular}%
\end{center}%
\caption{Fields of PajeNewEvent event}
\label{f:pajenewevent}
\end{figure}


\subsubsection{Variables}

There exist several events to set, add or subtract a value to/from a
variable\index{PajeSetVariable}\index{PajeAddVariable}\index{PajeSubVariable}.

\begin{figure}[htbp]
\begin{center}
\begin{tabular}{|LLL|}
\hline
\multicolumn{3}{|T|}{\textsf{\textbf{PajeSetVariable}}}\\\hline
\textbf{Field Name} & \textbf{Field Type} & \textbf{Description}\\
\hline
Time          & date              & Time the variable changed value\\
Type          & string or integer & Type of variable \\
Container     & string or integer & Container whose value changed \\
Value         & double            & New value of variable \\
\hline
\end{tabular}

\begin{tabular}{|LLL|}
\hline
\multicolumn{3}{|T|}{\textsf{\textbf{PajeAddVariable}}}\\\hline
\textbf{Field Name} & \textbf{Field Type} & \textbf{Description}\\
\hline
Time          & date              & Time the variable changed value\\
Type          & string or integer & Type of variable \\
Container     & string or integer & Container whose value changed \\
Value         & double            & Value to be added to variable \\
\hline
\end{tabular}

\begin{tabular}{|LLL|}
\hline
\multicolumn{3}{|T|}{\textsf{\textbf{PajeSubVariable}}}\\\hline
\textbf{Field Name} & \textbf{Field Type} & \textbf{Description}\\
\hline
Time          & date              & Time the variable changed value\\
Type          & string or integer & Type of variable \\
Container     & string or integer & Container whose value changed \\
Value         & double            & Value to be subtracted from variable \\
\hline
\end{tabular}%
\end{center}%
\caption{Fields of events that change value of variables}
\label{f:pajesetvalue}
\end{figure}


\subsubsection{Links}

A link is defined by two events, a
``PajeStartLink''\index{PajeStartLink} and a
``PajeEndLink''\index{PajeEndLink}.  These two events are matched and
considered to form a link when their respective ``Container'',
``Value'' and ``Key'' fields are the same.

\begin{figure}[htbp]
\begin{center}
\begin{tabular}{|LLL|}
\hline
\multicolumn{3}{|T|}{\textsf{\textbf{PajeStartLink}}}\\\hline
\textbf{Field Name} & \textbf{Field Type} & \textbf{Description}\\
\hline
Time          & date              & Time the link started\\
Type          & string or integer & Type of link \\
Container     & string or integer & Container that has the link \\
SourceContainer&string or integer & Container where link started \\
Value         & string or integer & Value of link \\
Key           & string or integer & Used to match to link end \\
\hline
\end{tabular}

\begin{tabular}{|LLL|}
\hline
\multicolumn{3}{|T|}{\textsf{\textbf{PajeEndLink}}}\\\hline
\textbf{Field Name} & \textbf{Field Type} & \textbf{Description}\\
\hline
Time          & date              & Time the link started\\
Type          & string or integer & Type of link \\
Container     & string or integer & Container that has the link \\
DestContainer & string or integer & Container where link ended \\
Value         & string or integer & Value of link \\
Key           & string or integer & Used to match to link start \\
\hline
\end{tabular}%
\end{center}%
\caption{Fields of events that create links}
\label{f:pajelink}
\end{figure}


\section{Example}
\label{sec:example}

The whole trace file of the example would be:

\begin{verbatim}

%EventDef       PajeDefineContainerType 1
%       Alias           string
%       ContainerType   string
%       Name            string
%EndEventDef
%EventDef       PajeDefineStateType     3
%       Alias           string
%       ContainerType   string
%       Name            string
%EndEventDef
%EventDef       PajeDefineEntityValue   6
%       Alias           string
%       EntityType      string
%       Name            string
%EndEventDef
%EventDef       PajeCreateContainer     7
%       Time            date
%       Alias           string
%       Type            string
%       Container       string
%       Name            string
%EndEventDef
%EventDef       PajeDestroyContainer    8
%       Time            date
%       Name            string
%       Type            string
%EndEventDef
%EventDef       PajeSetState           10
%       Time            date
%       Type            string
%       Container       string
%       Value           string
%EndEventDef
1 P 0 Program
1 T P Thread
3 S T "Thread State"
6 E S Executing
6 B S Blocked
7 0 TTP P 0 "Thread Testing Program"
7 0.986789 T1 T TTP "Thread 1"
10 0.986789 S T1 E
7 1.012332 T2 T TTP "Thread 2"
10 1.012332 S T2 E
10 2.34567 S T1 B
10 2.405678 S T2 B
10 2.456789 S T1 E
10 4.001543 S T2 E
8 4.295677 T2 T
8 4.34565 T1 T
8 4.3498 TTP P

\end{verbatim}


\section{Visualisation of the activity of the processors of a cluster}
\label{sec:admin}

The genericity of Paj� made it possible to visualize the system
activity of the processors of a large-sized (200 PEs) cluster of
personal computers \cite{GuilloudCAS:2001}. Several visualizations
were built from the system information available in the \texttt{/proc}
pseudo-directory of each PE. It was thus possible to represent the
processor and memory load of each PE, the most time consuming process
of each PE., etc. Paj� was also used to visualize the reservations of
PEs by the users of the cluster, the reservations being done using the
PBS system \cite{PBS} (see figure~\ref{fig:pbs}). The main bits of the
Paj� trace file analyzed to produce this figure are given below.

\begin{figure}[ht]
\epsfxsize=\linewidth
%\epsfysize=5cm
%\centerline{\epsfbox{FIG/pbs5.eps}}
\caption{Scheduling of jobs on a large-sized cluster}
\label{fig:pbs}
\end{figure}

\begin{verbatim}
%EventDef       PajeDefineContainerType 1
%       Alias           string
%       ContainerType   string
%       Name            string
%EndEventDef
%EventDef       PajeDefineEventType     2
%       Alias           string
%       ContainerType   string
%       Name            string
%EndEventDef
%EventDef       PajeDefineStateType     3
%       Name            string
%       ContainerType   string
%EndEventDef
%EventDef       PajeDefineEntityValue   6
%       Name            string
%       EntityType      string
%EndEventDef
%EventDef       PajeCreateContainer     7
%       Time            date
%       Alias           string
%       Type            string
%       Container       string
%       Name            string
%EndEventDef
%EventDef       PajeDestroyContainer    8
%       Time            date
%       Name            string
%       Type            string
%EndEventDef
%EventDef       PajeSetState           10
%       Time            date
%       Type            string
%       Container       string
%       Value           string
%EndEventDef
1 MG        0      M-Grappe
1 G         MG     Grappe
1 M         G      Machine
1 CPU       M      Processeur
3 pbs-task  CPU
7 7 MG1     MG     0   M-grappe_1
7 7 G1      G      MG1 Grappe_1
6 nobody pbs-task
6 chapron pbs-task
6 charao pbs-task
6 fchaussum pbs-task
6 feliot pbs-task
6 guilloud pbs-task
6 gustavo pbs-task
6 leblanc pbs-task
6 maillard pbs-task
6 mpillon pbs-task
6 paugerat pbs-task
6 plumejea pbs-task
6 romagnol pbs-task
6 sderr pbs-task         
7 8 M_icluster11  M  G1 M_icluster11
7 8 P_icluster11  CPU  M_icluster11 P_icluster11
7 8 M_icluster21  M  G1 M_icluster21
7 8 P_icluster21  CPU  M_icluster21 P_icluster21
7 8 M_icluster31  M  G1 M_icluster31
7 8 P_icluster31  CPU  M_icluster31 P_icluster31
7 8 M_icluster41  M  G1 M_icluster41

[...]

10 1 pbs-task P_icluster5 nobody
10 4273 pbs-task P_icluster5 nobody
10 5323 pbs-task P_icluster5 plumejea
10 7893 pbs-task P_icluster5 nobody
10 8277 pbs-task P_icluster5 feliot
10 8611 pbs-task P_icluster5 nobody
10 8633 pbs-task P_icluster5 feliot
10 8804 pbs-task P_icluster5 nobody
10 8836 pbs-task P_icluster5 feliot
10 9655 pbs-task P_icluster5 nobody
10 10038 pbs-task P_icluster5 feliot
10 10899 pbs-task P_icluster5 nobody
10 10930 pbs-task P_icluster5 feliot
10 10944 pbs-task P_icluster5 nobody

[...]

10 438224 pbs-task P_icluster100 feliot
10 438278 pbs-task P_icluster100 nobody
10 438339 pbs-task P_icluster100 feliot
10 438713 pbs-task P_icluster100 nobody
10 665686 pbs-task P_icluster100 sderr
10 665727 pbs-task P_icluster100 nobody
8 1465976  P_icluster100 P
8 1465976  M_icluster100 M
8 1465976 G1 G
8 1465976 MG1 MG


\end{verbatim}


%%%%%%%%%%%%%%%%%%%%%%%%%%%%%%%%%%%%%%%%%%%%%%%%%%%%%%%%%%%%%%%%%%%%%%%%%%%%%
%\chapter{Conclusion}
%%%%%%%%%%%%%%%%%%%%%%%%%%%%%%%%%%%%%%%%%%%%%%%%%%%%%%%%%%%%%%%%%%%%%%%%%%%%%
%%%%%%%%%%%%%%%%%%%%%%%%%%%%%%%%%%%%%%%%%%%%%%%%%%%%%%%%%%%%%%%%%%%%%%%%%%%%%%
%\chapter{Conclusion}
%%%%%%%%%%%%%%%%%%%%%%%%%%%%%%%%%%%%%%%%%%%%%%%%%%%%%%%%%%%%%%%%%%%%%%%%%%%%%

Paj� is a versatile visualization tool which can be used in a large
variety of contexts. This report describes the data format used by
Paj�. Paj� being trace-based, the data actually used for the
visualisation is to be presented as a set of execution events. In
addition, a description of the type hierarchy of the visual objects
needs to be included in the data (trace) file. Both the formats of the
type hierarchy description and of the events being self defined, there
also need to be a definition of these formats in the input data
(trace) file used by Paj�.

The versatility property has been used so far to visualize a
distributed Java application and the activity of the nodes of a
large-sized cluster of PCs. 

To enlarge the applicability of Paj�, a translator from traces
produced by Tau \cite{ShendeMCLBK:1998} into the Paj� format is
currently being implemented.




\bibliography{lang-paje}
\bibliographystyle{abbrv}

\addcontentsline{toc}{chapter}{Index}
\IfFileExists{lang-paje.ind}{%
  \documentclass[11pt,twoside]{report}
\usepackage{fullpage}
\usepackage{epsfig}
\usepackage{xspace}
\usepackage{alltt}

% \usepackage[francais]{babel}        % Pour Linux
\usepackage[latin1]{inputenc}
\usepackage[T1]{fontenc}

\setcounter{secnumdepth}{3}  %% pour num�roter les subsubsections
\setcounter{tocdepth}{3}     %% profondeur dans la table des mati�res

\usepackage{times}


\usepackage{amsmath}
\usepackage{pifont} % use of dingbats
\usepackage{array} % extension to the tabular env.
\usepackage{color} % for adding colors to things
\usepackage{colortbl} % for adding colors to tables

\newenvironment{captiontext}{\begin{quote} \footnotesize}{\end{quote}}
\newcommand{\ath}{\textsc{Atha\-pas\-can}\xspace}
\newcommand{\comment}[1]{}
\newlength{\figurewidth}\setlength{\figurewidth}{\textwidth}
%\addtolength{\figurewidth}{-\fboxsep}
%\addtolength{\figurewidth}{-\fboxsep}
\newsavebox{\figurebox}
\newcommand{\makefigure}[4]{%
   \begin{figure}[bt]\centering%
   \sbox{\figurebox}{#2}%
   \ifdim \wd\figurebox >\figurewidth%
      \resizebox{\figurewidth}{!}{\usebox{\figurebox}}%
   \else%
      \makebox[\figurewidth]{\usebox{\figurebox}}%
   \fi%
\vspace*{-3mm}
   \caption[#3]{\parbox[t]{11cm}{#3\newline\raggedright\protect\scriptsize #4}}%
\vspace*{2mm}
   \label{#1}%
   \end{figure}%
}
%% Tables
%
% Usage:
% \maketable{label}{figure command}{main caption}{secondary caption}
%
% there is a new column type for titles, T, to be used in \multicolumns,
% to put titles in reverse color, bold, sans serif, left aligned.
% there are also types R, L and C; they are like r, l, c but sans serif
%
\newlength{\tablewidth}\setlength{\tablewidth}{\textwidth}
\newcommand{\maketable}[4]{%
   \begin{table}[bt]\centering%
   \caption[#3]{#3\newline\footnotesize #4}\label{#1}%
   \sbox{\figurebox}{#2}%
   \ifdim \wd\figurebox >\tablewidth
      \resizebox{\tablewidth}{!}{\usebox{\figurebox}}%
   \else%
      \usebox{\figurebox}%
   \fi%
   \end{table}%
}
\newcolumntype{T}{>{\sffamily\bfseries\color{white}\columncolor[gray]{.2}}l}
\newcolumntype{R}{>{\sffamily}r}
\newcolumntype{L}{>{\sffamily}l}
\newcolumntype{C}{>{\sffamily}c}

\newcommand{\codefigurestart}{
  \begin{figure}[hbt]
    \begin{tabular}{|l|}
      \hline
      \begin{minipage}{\codewidth}
        \medskip\small
}

\newcommand{\codefigureend}[3]{
        \medskip
      \end{minipage}
      \\\hline
    \end{tabular}
%    \caption[#2]{#2\newline\footnotesize #3}%
    \caption[#2]{\parbox[t]{11cm}{#2\newline\raggedright\protect\scriptsize #3}}%
    \label{#1}%
  \end{figure}
}
\newlength{\codewidth}\setlength{\codewidth}{\figurewidth}
\addtolength{\codewidth}{-\fboxsep}
\addtolength{\codewidth}{-\fboxsep}
 % D�finitions de la th�se de Benhur

\makeindex

\title{Paj� trace file format}

\author{B. de Oliveira Stein\\ 
Departamento de Eletr\^onica e Computa\c{c}\~ao\\
Universidade Federal de Santa Maria - RS, Brazil.\\
Email: benhur@inf.UFSM.br\\
\and
J. Chassin de Kergommeaux\\
Laboratoire Informatique et Distribution (ID-IMAG)\\
ENSIMAG - antenne de Montbonnot,\\ ZIRST, 51, avenue Jean Kuntzmann\\
F-38330 Montbonnot Saint Martin, France \\ 
Email:Jacques.Chassin-de-Kergommeaux@imag.fr\\
http://www-apache.imag.fr/\~\/chassin
}

\begin{document}

\maketitle

\begin{abstract}
  
  Paj� is an interactive and scalable trace-based visualization tool
  which can be used for a large variety of visualizations including
  performance monitoring of parallel applications, monitoring the
  execution of processors in a large scale PC cluster or representing
  the behavior of distributed applications. Users of Paj� can tailor
  the visualization to their needs, without having to know any insight
  nor to modify any component of Paj�. This can be done by defining
  the type hierarchy of objects to be visualized as well as how these
  objects should be visualized. This feature allows the use of Paj�
  for a wide variety of visualizations such as the use of resources by
  applications in a large-size cluster or the behavior of distributed
  Java applications.  This report describes the trace format used by
  Paj�. Traces include three different kind of informations:
  definition of the formats of the event, definition of the type
  hierarchy of the objects to be visualized, definition of the formats
  of the events of the trace and a set of recorded events, complying
  with the format definition, to be used to build visualizations
  according to the type hierarchy.

 \textbf{Keywords:} performance debugging, visualization, MPI, pthread, 
parallel programming, self defined data format.

  
\end{abstract}

\tableofcontents

%%%%%%%%%%%%%%%%%%%%%%%%%%%%%%%%%%%%%%%%%%%%%%%%%%%%%%%%%%%%%%%%%%%%%%%%%%%%%
%\chapter{Introduction}
%%%%%%%%%%%%%%%%%%%%%%%%%%%%%%%%%%%%%%%%%%%%%%%%%%%%%%%%%%%%%%%%%%%%%%%%%%%%%
%%%%%%%%%%%%%%%%%%%%%%%%%%%%%%%%%%%%%%%%%%%%%%%%%%%%%%%%%%%%%%%%%%%%%%%%%%%%%%
% \chapter{Introduction}
%%%%%%%%%%%%%%%%%%%%%%%%%%%%%%%%%%%%%%%%%%%%%%%%%%%%%%%%%%%%%%%%%%%%%%%%%%%%%

This report defines the input data format used by the Pajé
visualization tool. Pajé is a versatile trace-based visualization tool
designed to help performance debugging of large-sized parallel
applications. From trace files, recorded during the execution of
parallel programs, Pajé builds a graphical representation of the
behavior of these programs, to help programmers identify their
``performance errors''. The main novelty of Pajé is an original
combination of three of the most desirable properties of visualisation
tools for parallel programs: extensibility, interactivity and
scalability. 

Scalability is the ability to represent the execution of
parallel programs executing during long periods on large-sized
systems; it is provided in Pajé by zooming and filtering
functionalities, both in space --- ability to synthesize the
information originating from several nodes of the system or to zoom in
one of these nodes --- and in time --- possibility to display  period
of time at various levels of detail. Interactivity is the ability to
interrogate visual objects --- events, thread states, communications,
etc. --- to obtain more details or check the source code whose
execution produced a given event; it is also the ability to move back
and forth in time or to zoom from a synthetic representation to a
detailed one or vice versa or to set or reset a filter. Extensibility
is the possibility ot extend the tool with new functionalities ---
visual representations, filters, etc. --- or to display new
programming models. Several characteristics of Pajé contribute to its
extensibility: careful modular design, independence of the
visualization modules from the programming model.

Key to the ability to build a visual representation of the behavior of
parallel programs, developed with various programming models, is the
\textit{genericity} of Pajé: ability to parameterize the tools with a
description of \textit{what} is to be represented and \textit{how}.
This description is provided in the trace file as a hierarchy of the
types of objects appearing in the visualization. The format of this
description as well as the format of the events of the trace are also
described in the trace file. The trace files\index{trace file} used as
input by Pajé thus contain four categories of data:
\begin{enumerate}
\item Description of the format of the generic instructions.
\item Generic instructions, describing the hierarchy of the types of
  objects appearing in the visualization.
\item Description of the format of the events recorded during the
  execution of the visualized program.
\item Events recorded during the execution of the program to be
  visualized.
\end{enumerate}

The aim of this technical report is to describe the Pajé trace data
format. The organization of the report is the following. After this
introduction, the extensibility and genericity of Pajé are described
in detail. The following section defines the Pajé data format and
gives an example of use before the conclusion.


%Pajé is an interactive visualization tool originally designed for
%displaying the execution of parallel applications where a
%(potentially) large number of communicating threads of various
%life-times execute on each node of a distributed memory parallel
%system.   To be easier
%to extend, Pajé was designed as a data-flow graph of modular
%components, most of them being independent of the semantics of the
%parallel programming model of the visualized parallel programs. In
%addition, application programmers can tailor the visualization to
%their needs, without having to know any insight nor to modify any
%component of Pajé. This can be done by defining the type hierarchy of
%objects to be visualized as well as how these objects should be
%visualized.


%%%%%%%%%%%%%%%%%%%%%%%%%%%%%%%%%%%%%%%%%%%%%%%%%%%%%%%%%%%%%%%%%%%%%%%%%%%%%
%\chapter{Extensibility of Paj�}
%\label{chap:paje}
%%%%%%%%%%%%%%%%%%%%%%%%%%%%%%%%%%%%%%%%%%%%%%%%%%%%%%%%%%%%%%%%%%%%%%%%%%%%%
%%%%%%%%%%%%%%%%%%%%%%%%%%%%%%%%%%%%%%%%%%%%%%%%%%%%%%%%%%%%%%%%%%%%%%%%%%%%%%
% \chapter{Extensibility of Pajé}
%%%%%%%%%%%%%%%%%%%%%%%%%%%%%%%%%%%%%%%%%%%%%%%%%%%%%%%%%%%%%%%%%%%%%%%%%%%%%

\section{Introduction}

The Pajé visualization tool described in this article\footnote{This
  chapter was published \emph{in: Euro-Par 2000 Parallel Processing,
    Proc. 6th International Euro-Par Conference}, A.~Bode, W.~Ludwig,
  T.~Karl, R.~Wism\"uller (r\'ed.), \emph{LNCS}, \emph{1900},
  Springer, p.~133--140, 2000.} was designed to allow programmers to
visualize the executions of parallel programs using a potentially
large number of communicating threads (lightweight processes) evolving
dynamically.  The visualization of the executions is an essential tool
to help tuning applications implemented using such a parallel
programming model.

Visualizing a large number of threads raises a number of problems such
as coping with the lack of space available on the screen to visualize
them and understanding such a complex display. The graphical displays
of most existing visualization tools for parallel programs
\cite{Heath:1991,upshot,Kranzlmueller:1996:PPV,PALLAS,pablo,ncstrl.gatech_cc//GIT-CC-95-21,ute}
show the activity of a fixed number of nodes and inter-nodes
communications; it is only possible to represent the activity of a
single thread of control on each of the nodes. It is of course
conceivable to use these systems to visualize the activity of
multi-threaded nodes, representing each thread as a node.  In this
case, the number of threads should be fairly limited and should not
vary during the execution of the program. These visualization tools
are therefore not adapted to visualize threads whose number varies
continuously and life-time is often short.  In addition, these tools
do not support the visualization of local thread synchronizations
using mutexes or semaphores.

Some tools were designed to display multithreaded
programs~\cite{HammondKev1995a,gthread}.  However, they support a
programming model involving a single level of parallelism within a
node, this node being in general a shared-memory multiprocessor. Our
programs execute on several nodes: within the same node, threads
communicate using synchronization primitives; however, threads
executing on different nodes communicate by message passing. Moreover,
compared to these systems, Pajé ought to represent a much larger
number of objects.

The most innovative feature of Pajé is to combine the characteristics
of interactivity and scalability with extensibility. In contrast with
passive visualization tools~\cite{Heath:1991,pablo} where parallel
program entities --- communications, changes in processor states, etc.
--- are displayed as soon as produced and cannot be interrogated, it
is possible to inspect all the objects displayed in the current screen
and to move back in time, displaying past objects again. Scalability
is the ability to cope with a large number of threads. Extensibility
is an important characteristic of visualization tools to cope with the
evolution of parallel programming interfaces and visualization
techniques. Extensibility gives the possibility to extend the
environment with new functionalities: processing of new types of
traces, adding new graphical displays, visualizing new programming
models, etc.

The interactivity and scalability characteristics of Pajé were
described in previous articles
\cite{ChassinS00,ChassinS:2000a,SteinC98}.  This article focuses on
the extensibility characteristics: modular design easing the addition
of new modules, semantics independent modules which allow them to be
used in a large variety of contexts and especially genericity of the
simulator component of Pajé which gives to application programmers the
ability to define what they want to visualize and how it must be done.

The organization of this article is the following. The next section
summarizes the main functionalities of Pajé.  The following section
describes the extensibility of Pajé before the conclusion.


\section{Outline of Pajé}
\label{sec:Pajé}

Pajé was designed to ease performance debugging of \ath programs by
visualizing their executions and because no existing visualization
tool could be used to visualize such multi-threaded programs.

\subsection{\ath: a thread-based parallel programming model}
\label{sec:ath}

Combining threads and communications is increasingly used to program
irregular applications, mask communication or I/O latencies, avoid
communication deadlocks, exploit shared-memory parallelism and
implement remote memory accesses
\cite{Fahringer:1995:UTD,FosterKT96,hicss95}.  The
\ath~\cite{ath0b-europar97} programming model was designed for
parallel hardware systems composed of shared-memory multi-processor
nodes connected by a communication network. It exploits two levels of
parallelism: inter-nodes parallelism and inner parallelism within each
of the nodes. The first type of parallelism is exploited by a fixed
number of system-level processes while the second type is implemented
by a network of communicating threads evolving dynamically. The main
functionalities of \ath are dynamic local or remote thread creation
and termination, sharing of memory space between the threads of the
same node which can synchronize using locks or semaphores, and
blocking or non-blocking message-passing communications between non
local threads, using ports. Combining the main functionalities of MPI
\cite{MPI} with those of \texttt{pthread} compliant libraries, \ath
can be seen as a ``thread aware'' implementation of MPI.

\subsection{Tracing of parallel programs}
\label{sec:tracing}

Execution traces are collected during an execution of the observed
application, using an instrumented version of the \ath\ library. A
non-intrusive, statistical method is used to estimate a precise global
time reference \cite{MailletT:1995}. The events are stored in local
event buffers, which are flushed when full to local event files.  The
collection of events into a single file is only done after the end of
the user's application to avoid interfering with it.  Recorded events
may contain source code information in order to implement source code
click-back --- from visualization to source code --- and click-forward
--- from source code to visualization --- in Pajé.

\subsection{Visualization of threads in Pajé}
\label{s-spacetime}

The visualization of the activity of multi-threaded nodes is mainly
performed in a diagram combining in a single representation the states
and communications of each thread(see figure~\ref{f-spacetime}) .
%
\makefigure{f-spacetime} {\includegraphics{FIG/spacetime-bact-e-2}}
{Visualization of an \ath program execution} {Blocked thread states
  are represented in clear color; runnable states in a dark color. The
  smaller window shows the inspection of an event.}

\makefigure{f-sema} {\includegraphics{FIG/sema-note-2}} {Visualization of
  semaphores} {Note the highlighting of a thread blocked state because
  the mouse pointer is over a semaphore blocked state, and the arrows
  that show the link between a ``V'' operation in a semaphore and the
  corresponding unblocking of a thread.}
%
The horizontal axis represents time while threads are displayed along
the vertical axis, grouped by node. The space allocated to each node
of the parallel system is dynamically adjusted to the number of
visualized threads of this node.  Communications are represented by
arrows while the states of threads are displayed by rectangles. Colors
are used to indicate either the type of a communication, or the
activity of a thread.  It is not the most compact or scalable
representation, but it is very convenient for analyzing detailed
threads relationship, load distribution and masking of communication
latency.  Pajé deals with the scalability problem of this
visualization by means of filters, discussed later in
section~\ref{s-filtering}.

The states of semaphores and locks are represented like the states of
threads: each possible state is associated with a color, and a
rectangle of this color is shown in a position corresponding to the
period of time when the semaphore was in this state. Each lock is
associated with a color, and a rectangle of this color is drawn close
to the thread that holds it (see figure~\ref{f-sema}).

\subsection{Interactivity}

Progresses of the simulation are entirely driven by user-controlled
time displacements: at any time during a simulation, it is possible to
move forward or backward in time, within the limits of the visualized
program execution.  In addition, Pajé offers many possible
interactions to programmers: displayed objects can be inspected to
obtain all the information available for them (see inspection window
in figure~\ref{f-spacetime}), identify related objects or check the
corresponding source code.  Moving the mouse pointer over the
representation of a blocked thread state highlights the corresponding
semaphore state, allowing an immediate recognition (see figure
\ref{f-sema}).  Similarly, all threads blocked in a semaphore are
highlighted when the pointer is moved over the corresponding state of
the semaphore.  From the visual representation of an event, it is
possible to display the corresponding source code line of the parallel
application being visualized.  Likewise, selecting a line in the
source code browser highlights the events that have been generated by
this line.

\subsection{Scalability: filtering of information and zooming capabilities}
\label{s-filtering}

It is not possible to represent simultaneously all the information
that can be deduced from the execution traces.  Screen space
limitation is not the only reason: part of the information may not be
needed all the time or cannot be represented in a graphical way or can
have several graphical representations.  Pajé offers several filtering
and zooming functionalities to help programmers cope with this large
amount of information to give users a simplified, abstract view of the
data. Accessing more detailed information can amount to exploding a
synthetic view into a more detailed view or getting to data that exist
but have not been used or are not directly related to the
visualization. Figure~\ref{f-filters-group} examplifies one of the
filtering facilities provided by Pajé where a single line represents
the number of active threads of a node and a pie graph the CPU
activity in the time slice selected in the space-time diagram (see
\cite{ChassinS00,ChassinS:2000a,Stein:1999}) for more details).

\makefigure{f-filters-group}
           {\includegraphics{FIG/busy+pie}} {CPU utilization} {Grouping
           the threads of each node to display the state of the whole
           system (lighter colors mean more active threads); the
           pie-chart shows the percentage of the selected time slice
           spent with each number of active threads in each node.}

\section{Extensibility}
\label{sec:extensibility}

Extensibility is a key property of a visualization tool. The
main reason is that a visualization tool being a very complex 
piece of software costly to implement, its lifetime ought to
be as long as possible. This will be possible only if the tool
can cope with the evolutions of parallel programming models ---
since this domain is still evolving rapidly --- and of the
visualization techniques. Several characteristics of Pajé were
designed to provide a high degree of extensibility: modular
architecture, flexibility of the visualization modules and genericity
of the simulation module.


\subsection{Modular architecture}
\label{sec:modular}

To favor extensibility, the architecture of Pajé is a data flow graph
of software modules or components (see figure~\ref{f-diagramme}). It
is therefore possible to add a new visualization component or adapt to
a change of trace format by changing the trace reader component
without changing the remaining of the environment.  This architectural
choice was inspired by Pablo \cite{pablo}, although the graph of Pajé
is not purely data-flow for interactivity reasons: it also includes
control-flow information, generated by the visualization modules to
process user interactions and triggering the flow of data in the graph
(see \cite{ChassinS00,ChassinS:2000a,Stein:1999} for more details on
the implementation of interactivity in Pajé).

\makefigure{f-diagramme}
           {\includegraphics{FIG/struct-obj-1b-e}} {Example data-flow
           graph} {The trace reader produces event objects from the
           data read from disk. These events are used by the simulator
           to produce more abstract objects, like thread states,
           communications, etc., traveling on the arcs of the
           data-flow graph to be used by the other components of the
           environment.}


\subsection{Flexibility of visualization modules}
\label{sec:flexibility}

The Pajé visualization components have no dependency whatsoever with
any parallel programming model. Prior to any visualization they
receive as input the description of the types of the objects to be
visualized as well as the relations between these objects and the way
these objects ought to be visualized (see
figure~\ref{fig:hierarchie1}). The only constraints are the
hierarchical nature of the type relations between the visualized
objects and the ability to place each of these objects on the
time-scale of the visualization. The hierarchical type description is
used by the visualization components to query objects from the
preceding components in the graph.

This type description can be changed to adapt to a new programming
model (see section~\ref{sec:genericity}) or during a visualization, to
change the visual representation of an object upon request from the
user. In addition to providing a high versatility for the
visualization components, this feature is used by the filtering
components. When a filter is dynamically inserted in a data-flow graph
--- for example between the simulation and visualization components of
figure~\ref{f-diagramme} to zoom from a detailed visualization to
obtain a more global view of the program execution such as
figure~\ref{f-filters-group} ---, it first sends a type description of
the hierarchy of objects to be visualized to the following components
of the data-flow graph.

\makefigure{fig:hierarchie1} {\includegraphics{FIG/hierarchya}} {Use of a
  simple type hierarchy} {The type hierarchy on the left-hand side of
  the figure defines the type hierarchical relations between the
  objects to be visualized and how how these objects should be
  represented: communications as arrows, thread events as triangles
  and thread states as rectangles.}
  
The type hierarchies used in Pajé are trees whose leaves are called
\textit{entities}\index{entities} and intermediate nodes
\textit{containers}\index{containers}. Entities are elementary objects
that can be displayed such as events, thread states or communications.
Containers are higher level objects used to structure the type
hierarchy\index{type hierarchy} (see figure~\ref{fig:hierarchie1}).
For example: all events occurring in thread~1 of node~0 belong to the
container ``thread-1-of-node-0''.


\subsection{Genericity of Pajé}
\label{sec:genericity}

The modular structure of Pajé as well as the fact that filter and
visualization components are independent of any programming model
makes it ``easy'' for tool developers to add a new component or extend
an existing one. These characteristics alone would not be sufficient
to use Pajé to visualize various programming models if the simulation
component were dependent on the programming model: visualizing a new
programming model would then require to develop a new simulation
component, which is still an important programming effort, reserved
for experienced tool developers.

On the contrary, the generic property of Pajé allows application
programmers to define \textit{what} they would like to visualize and
\textit{how} the visualized objects should be represented by Pajé.
Instead of being computed by a simulation component, designed for a
specific programming model such as \ath, the type hierarchy of the
visualized objects (see section~\ref{sec:flexibility}) can be defined
by inserting several definitions and commands in the trace file (see
format in chapter~\ref{chap:format}). If --- as it is the case with
the \ath-0 tracer and as it is assumed in this paper --- the tracer
(see section~\ref{sec:tracing}) can collect them, these definitions
and command can be inserted in the application program to be traced
and visualized.  The simulator uses these definitions to build a new
data type tree used to relate the objects to be displayed, this tree
being passed to the following modules of the data flow graph: filters
and visualization components.

\subsubsection{New data types definition.}

One function call is available to create new types of containers while
four can be used to create new types of entities which can be events,
states, links and variables. An ``event''\index{event} is an entity
representing an instantaneous action. ``States''\index{state} of
interest are those of containers.  A ``link''\index{link} represents
some form of connection between a source and a destination container.
A ``variable''\index{variable} stores the temporal evolution of the
successive values of a data associated with a container.
Table~\ref{t:defusertypes} contains the function calls that can be
used to define new types of containers and entities.
Figure~\ref{fig:hierarchie2} shows the effect of adding the
``threads'' container to the type hierarchy of
figure~\ref{fig:hierarchie1}.

\maketable{t:defusertypes}
{\small
\begin{tabular}{|>{\scshape}RL>{\scshape}L|}
\hline
\multicolumn{1}{|T}{Result}&
\multicolumn{1}{T}{Call}&
\multicolumn{1}{T|}{Parameters}  \\
\hline
ctype     & pajeDefineUserContainerType & ctype name                        \\
\hline
\hline
etype     & pajeDefineUserEventType     & ctype name                        \\
etype     & pajeDefineUserStateType     & ctype name                        \\
etype     & pajeDefineUserLinkType      & ctype name                        \\
etype     & pajeDefineUserVariableType  & ctype name                        \\
\hline
evalue    & pajeNewUserEntityValue      & etype name                        \\
\hline
\end{tabular}
} {Containers\index{container} and entities\index{entity} types
definitions} {The argument \textsc{\textsf{ctype}} is the type of the
father container of the newly defined type, in the type hierarchy (the
container ``Execution'' being always the root of the tree of types).}

%
\makefigure{fig:hierarchie2} {\includegraphics{FIG/hierarchyb}}{Adding
  a container to the type hierarchy\index{type hierarchy} of
  figure~\ref{fig:hierarchie1}}{}
  
\subsubsection{Data generation.}

Several functions can be used to create containers\index{container}
and entities\index{entity} whose types were defined using
table~\ref{t:defusertypes} primitives.  Specific functions are used to
create events, states (and embedded states using \textit{Push} and
\textit{Pop}), links --- each link being created by one source and one
destination calls, the coupling between them being performed by the
simulator when parameters \texttt{container, evalue} and \texttt{key}
of both source and destination calls match --- and change the values
of variables (see table~\ref{t:creationuser}).
%
\maketable{t:creationuser}
{\small
\begin{tabular}{|>{\scshape}RL>{\scshape}L|}
\hline
\multicolumn{1}{|T}{Result}&
\multicolumn{1}{T}{Call}&
\multicolumn{1}{T|}{Parameters}  \\
\hline
container & pajeCreateUserContainer     & ctype name in-container           \\
          & pajeDestroyUserContainer    & container                         \\
\hline
\hline
          & pajeUserEvent               & etype container evalue comment    \\
\hline
          & pajeSetUserState            & etype container evalue comment    \\
          & pajePushUserState           & etype container evalue comment    \\
          & pajePopUserState            & etype container comment           \\
\hline
          & pajeStartUserLink           & etype container srccontainer      \\
          &                           & \quad \quad evalue key comment    \\
          & pajeEndUserLink             & etype container destcontainer     \\
          &                           & \quad \quad evalue key comment    \\
\hline
          & pajeSetUserVariable         & etype container value comment     \\
          & pajeAddUserVariable         & etype container value comment     \\
\hline
\end{tabular}
} {Creation of containers and entities} {Calls to these functions are
inserted in the traced application to generate ``user events'' whose
processing by the Pajé simulator will use the type tree built from the
containers and entities types definitions done using the functions of
table~\ref{t:defusertypes}.}

In the example of figure~\ref{f:simpleprogramtraced}, a new event is
generated for each change of computation phase. This event is
interpreted by the Pajé simulator component to generate the
corresponding container state. For example the following call
indicates that the computation is entering in a ``Local computation''
phase: \\
{\small\tt\verb"  pajeSetUserState ( phase_state, node, local_phase, str_iter );"}\\
The second parameter indicates the container of the state (the
``node'' whose computation has just been changed).  The last parameter
is a comment that can be visualized by Pajé. In the example it is used
to display the current iteration value. The example program of
figure~\ref{f:simpleprogramtraced} includes the definitions and
creations of entities ``Computation phase'', allowing the visual
representation of an \ath program execution to be extended to
represent the phases of the computation.
Figure~\ref{f:simpleprogramvisu} includes two space-time diagrams
visualizing the execution of this example program, without and with
the definition of the new entities.
 
\codefigurestart{\scriptsize\alltt
 \textbf{unsigned phase_state, init_phase, local_phase, global_phase;
 phase_state  = pajeDefineUserStateType( A0_NODE, "Computation phase");
 init_phase   = pajeNewUserEntityValue( phase_state, "Initialization");
 local_phase  = pajeNewUserEntityValue( phase_state, "Local computation");
 global_phase = pajeNewUserEntityValue( phase_state,"Global computation");

 pajeSetUserState ( phase_state, node, init_phase, "" );}
 initialization();
 while (!converge) \{
     iter++;
     str_iter = itoa (iter);
     \textbf{pajeSetUserState ( phase_state, node, local_phase, str_iter );}
     local_computation();
     send (local_data);
     receive (remote_data);
     \textbf{pajeSetUserState ( phase_state, node, global_phase, str_iter );}
     global_computation();
 \}
}\codefigureend{f:simpleprogramtraced}
{Simplified algorithm of the example program}
{The five first lines written in bold face contain the generic
  instructions that have to be passed to Pajé through the trace file,
  to define a new type of state for the container \texttt{A0\_NODE}.
  Here it is assumed that the tracer is able to record these
  instructions in addition to the events of the program (
  \texttt{pajeSetUserState}).}
%
\makefigure{f:simpleprogramvisu} {\includegraphics{FIG/simpleprogram-2}}
{Visualization of the example program} {The second figure displays the
  entities ``Computation phases'' defined by the end-user. It is also
  possible to restrict the visualization to this information alone.}


\section{Conclusion}
\label{sec:conc}

Pajé provides solutions to interactively visualize the execution of
parallel applications using a varying number of threads communicating
by shared memory within each node and by message passing between
different nodes.  The most original feature of the tool is its unique
combination of extensibility, interactivity and scalability
properties. Extensibility means that the tool was defined to allow
tool developers to add new functionalities or extend existing ones
without having to change the rest of the tool. In addition, it is
possible to application programmers using the tool to define what they
wish to visualize and how this should be represented. To our knowledge
such a generic feature was not present in any previous visualization
tool for parallel programs executions.

The genericity property of Pajé was used to visualize \ath-1 programs
executions without having to perform any new development in Pajé.
\ath-1 is a high level parallel programming model where parallelism is
expressed by asynchronous task creations whose scheduling is performed
automatically by the run-time system \cite{a1-europar98,a1-pact98}.
The runtime system of \ath-1 is implemented using \ath. By extending
the type hierarchy defined for \ath and inserting few instructions to
the \ath-1 implementation, it was possible to visualize where the time
was spent during \ath-1 computations: computing the user program,
managing the task graph or scheduling the user-defined tasks.

Further developments include simplifying the generic description and
creation of visual objects, currently more complex when the generic
simulator is used instead of a specialized one. The generation of
traces for other thread-based programming models such as Java will
also be investigated to further validate the flexibility of Pajé.



%%%%%%%%%%%%%%%%%%%%%%%%%%%%%%%%%%%%%%%%%%%%%%%%%%%%%%%%%%%%%%%%%%%%%%%%%%%%%
\chapter{Definition of type hierarchies and trace event formats}
\label{chap:format}
%%%%%%%%%%%%%%%%%%%%%%%%%%%%%%%%%%%%%%%%%%%%%%%%%%%%%%%%%%%%%%%%%%%%%%%%%%%%%
%*
%   Copyright 1998, 1999, 2000, 2001, 2003, 2004 Benhur Stein
%   
%   This file is part of Paj�.
%
%   Paj� is free software; you can redistribute it and/or modify
%   it under the terms of the GNU General Public License as published by
%   the Free Software Foundation; either version 2 of the License, or
%   (at your option) any later version.
%
%   Foobar is distributed in the hope that it will be useful,
%   but WITHOUT ANY WARRANTY; without even the implied warranty of
%   MERCHANTABILITY or FITNESS FOR A PARTICULAR PURPOSE.  See the
%   GNU General Public License for more details.
%
%   You should have received a copy of the GNU General Public License
%   along with Foobar; if not, write to the Free Software
%   Foundation, Inc., 59 Temple Place, Suite 330, Boston, MA  02111-1307  USA
%/
%%%%%%%%%%%%%%%%%%%%%%%%%%%%%%%%%%%%%%%%%%%%%%%%%%%%%%%%%%%%%%%%%%%%%%%%%%%%%
% \chapter{Definition of type hierarchies and trace event formats}
%%%%%%%%%%%%%%%%%%%%%%%%%%%%%%%%%%%%%%%%%%%%%%%%%%%%%%%%%%%%%%%%%%%%%%%%%%%%%

% T, R, L and C already in definitions.tex
%\newcolumntype{T}{>{\sffamily\bfseries\color{white}\columncolor[gray]{.2}}l}
%\newcolumntype{R}{>{\sffamily}r}
%\newcolumntype{L}{>{\sffamily}l}
%\newcolumntype{C}{>{\sffamily}c}
\newcolumntype{P}{>{\sffamily}p{5cm}}

\section{Introduction}
\label{sec:traceintro}

A visualization constructed by Paj� is composed of objects organized
according to a tree type hierarchy whose nodes are called
\emph{containers}\index{container} and leaves
\emph{entities}\index{entity}% (see \S\ref{sec:genericity})
. The Paj�
data format is self-defined, although it does not comply with the
SDDF\index{SDDF} format used by Pablo \cite{sddf}. There exists a
``meta-format'' used to define:
\begin{itemize}
\item The format of the instructions defining containers and entities.
\item The format of the events recorded during the executions of
  parallel programs.
\end{itemize}

These definitions are usually inserted in trace files. They can even
be inserted in the observed programs source files, provided that the
tracers used to record the events of these programs are able to
capture a new definition as a ``user-defined'' event.% (see for example
%figure~\ref{f:simpleprogramtraced}).

Using these definitions, it is possible to define a hierarchy of
containers and entities adapted for a given programming model or
language. Definitions of type hierarchies as well as instructions and
events formats constitute a specialisation of the ``generic'' Paj�
visualization tool: it has been used so far to visualize distributed
applications written in Java \cite{OttogaliOSCV:2001} or help to
perform system monitoring on large sized clusters
\cite{GuilloudCAS:2001}. 

The organization of this chapter is the following. The next section
defines the meta format of Paj� used to define the format of type
definition instructions and events. The following sections describe
how containers and entities are defined and created. The next section
is dedicated to the trace events: self definition, recording. The last
section of this chapter contains a complete example of use of the Paj�
data format.

\section{Meta format of Paj�}
\label{sec:file}

A trace file is composed of events.
An event can be seen as a table composed of named fields, as shown in figure~\ref{f:event:table}. 
The first event in the figure can represent the sending of a message containing 320 bytes by process 5 to process 3, containing 320 bytes by process 5 to process 3, containing 320 bytes by process 5 to process 3, containing 320 bytes by process 5 to process 3, 3.233222 seconds after the process started executing.
The second event shows that process 5 unblocked at time 5.123002, and that this happened while executing line 98 of file sync.c.
Each event has some fields, each of them composed of a name, a type and a value. Generally, there are lots of similar events in a trace file (lots of ``SendMessage'' events, all with the same fields); a typical trace file contains thousands of events of tens of different types.
Usually, events of the same type have the same fields.
It is therefore wise, in order to reduce the trace file size, not to put the
information that is common to many events in each of those events.
The most common solution is to put only the type of each event and the values of its fields in the trace file. Information on what event types exist and the fields that constitute each of these event types being kept elsewhere.
In some trace file formats, this information is hardcoded in the trace generator and trace reader, making the trace structure hard to change in order to incorporate new types of events, new data in existing events or to remove unused or unknown data from those events.

\begin{figure}[htbp]
\begin{center}
\begin{tabular}{|>{\bf}rll|}
\hline
\textbf{Field Name} & \textbf{Field Type} & \textbf{Field Value} \\
\hline
EventName     & string    & SendMessage \\
Time          & timestamp & 3.233222    \\
ProcessId     & integer   & 5           \\
Receiver      & integer   & 3           \\
Size          & integer   & 320         \\
\hline
\end{tabular}
\quad\quad
\begin{tabular}{|>{\bf}rll|}
\hline
\textbf{Field Name} & \textbf{Field Type} & \textbf{Field Value} \\
\hline
EventName     & string    & UnblockProcess \\
Time          & timestamp & 5.123002    \\
ProcessId     & integer   & 5           \\
FileName      & string    & sync.c      \\
LineNumber    & integer   & 98          \\
\hline
\end{tabular}
\end{center}
\caption{Examples of events}
\label{f:event:table}
\end{figure}

A Paj� trace file is self defined, meaning that the event definition information is put inside the trace file itself, much like the SDDF file format used by the Pablo visualization tool \cite{sddf}.
The file is constituted of two parts: the definition of the events at the beginning of the file followed by the events themselves.
The definition of events contains the name of each event type and the names and types of each field.
The second part of the trace file contains the events, with the values associated to each field, in the same order as in the definition.
The correspondence of an event with its definition is made by means of a number, that must be unique for each event description; this number appears in an event definition and at the beginning of each event contained in the trace file.

The event definition part of a Paj� trace file follows the following format:
\begin{itemize}
\item all the lines start with a `\%' character;
\item each event definition starts with a \%EventDef\index{EventDef}
  line and terminates with a \%EndEventDef\index{EndEventDef} line;
\item the \%EventDef line contains the name and the unique number of an
  event type.  The number (an integer) will be used to identify the event
  in the second part of the trace file. The choice of this number is
  left to the user. The numbers given in the definitions below (see
%  \S\ref{sec:containers} and \S\ref{sec:entities}
  \S\ref{sec:example}) are thus
  \textbf{arbitrary}. The name of the event will be put in a field called
  ``PajeEventName''. There cannot be another field called so. The name is used
  to identify the type of an event;
\item the fields of an event are defined between the \%EventDef and
  the \%EndEventDef lines, one field per line, with the name
  of the field followed by its type (see below).
\end{itemize}

The structure of the two events of figure~\ref{f:event:table} are shown in figure~\ref{f:event:def}.

\begin{figure}
\begin{center}
\begin{minipage}{5.6cm}
\begin{verbatim}
%EventDef SendMessage 21
%   Time       date
%   ProcessId  int
%   Receiver   int
%   Size       int
%EndEventDef
\end{verbatim}
\end{minipage}
\quad\quad
\begin{minipage}{5.6cm}
\begin{verbatim}
%EventDef UnblockProcess 17
%   Time       date
%   ProcessId  int
%   LineNumber int
%   FileName   string
%EndEventDef
\end{verbatim}
\end{minipage}
\end{center}
\caption{Examples of event definitions}
\label{f:event:def}
\end{figure}

The type of a field can be one of the following:
\begin{description}
  \item [date:] for fields that represent dates\index{date}.
                It's a double precision floating-point number, usually meaning seconds since program start;
  \item [int:] for fields containing integer numeric values;
  \item [double:] for fields containing floating-point values;
  \item [hex:] for fields that represent addresses, in hexadecimal;
  \item [string:] for strings of characters.
  \item [color:] for fields that represent colors. A color is a sequence of
                 three floating-point numbers between 0 and 1, inside double 
                 quotes (").
                 The three numbers are the values of red, green and blue
                 components.
\end{description}

%Most events are dated. If that is the case, it must be defined with a field
named ``Time'' of type ``date'', for
%the date of generation of the event. 

The second part of the trace file contains one event per line, whose 
fields are separated by spaces or tabs, the first field being the number that
identifies the event type, followed by the other fields, in the same order that
they appear in the
definition of the event. 
Fields of type string must be inside double quotes (") if they contain space or
tab characters, or if they are empty.

For example, the two events of figure~\ref{f:event:table} are shown in figure~\ref{f:event}.
\begin{figure}
\begin{center}
\begin{minipage}{5cm}
\begin{verbatim}
21 3.233222 5 3 320
17 5.123002 5 98 sync.c
\end{verbatim}
\end{minipage}
\end{center}
\caption{Examples of events}
\label{f:event}
\end{figure}

In Paj�, event numbers are used only as a means to find the definition
of an event; they are discarded as soon as an event is read.
After being read, events are identified by their names.
Two different definitions can have the same name (and different numbers), making it possible to have, in the same trace file, two events of the same type containing different fields.
We use this feature to optionally include the source file identification in some
events. The ``UnblockProcess'' event in the examples above could also be defined without the fields FileName and LineNumber, for use in places where this information is not known or not necessary.

\section{Events treated by the Paj� simulator} %{Paj� "generic" events}
\label{sec:generic}

A Paj� visualization is best described as a typed hierarchy of objects
organized as a tree. Elementary objects are the leaves of the tree and
called ``entities'' while intermediate nodes of the tree are named
``containers''. 
Entities are the objects that can be visualized in Paj�'s space-time diagram,
while containers organize the space where those entities are displayed.

Paj� includes a simulator module which builds this hierarchical data
structure from the elementary event records of the trace files. 
Paj� has no
predefined containers or entities. 
Before an entity can be created and visualized, a hierarchy of container and
entity types must be defined, and containers must be instantiated.

For example, to visualize the states of threads in a program, one must
first define the container types ``Program'' and ``Thread'' and the entity
type ``Thread State''.  One must also define the possible values that
the entities of type ``Thread State'' can assume (for example,
``Executing'' and ``Blocked'').  Then, one must instantiate the program
creating a container of type ``Program'' (called ``Thread Testing
Program'', for example).  The threads of the program also have to be
instantiated; they are containers of type ``Thread'', called for example
``Thread 1''
and ``Thread 2'', and contained in container ``Thread Testing Program''.
Only then one is able to create visualizable entities of type ``Thread
State'', by means of events that represent changes in state, contained
either in ``Thread 1'' or ``Thread 2''.


The events that the Paj� simulator understands can be divided into four classes:
\begin{itemize}

\item events to define types of containers;
\item events to define types of entities and possible
values that entities can have; 
\item events to instantiate and destroy containers;
\item events to create visualizable entities.
\end{itemize}

Typically, the events of the first two classes are in the beginning of a trace
file, followed by events that instantiate containers, followed by a large number of events creating entities.
The simulator does
not impose this order, events of these four classes can be mixed in the trace
file. The limitation is that an entity or a container cannot be created before
its type has been defined and its container created.

The four classes of events are discussed in the following sessions.

\subsection{Definition of types of Containers}
\label{sec:contype}

Containers types are defined with events named ``PajeDefineContainerType''.

\subsubsection*{PajeDefineContainerType\index{PajeDefineContainerType}}

Events of this type (see figure~\ref{f:pajedefinecontainertype}) must contain the fields  ``Name'' and ``ContainerType''.
It defines a new container type called ``Name'', contained by a previously
defined container type ``ContainerType'' (or the special container type ``0'' or
``/'', if this container type is
the top of the container hierarchy).
Optionally this event can contain a field ``Alias'' with an alias name to be
used to identify this container. Aliases are usually short strings used when the
container name is too big and its use throughout the trace file would increase
the file's size.
When an alias is used in a definition, it must also be used in later references to the container type being defined.
When an alias is not used, a container type must be later referenced by its name.
The use of aliases allows for the definition of more than one container with the same name (and different aliases).

\begin{figure}[htbp]
\begin{center}
\begin{tabular}{|LLL|}
\hline
\multicolumn{3}{|T|}{\textsf{\textbf{PajeDefineContainerType}}}\\\hline
\textbf{Field Name} & \textbf{Field Type} & \textbf{Description}\\
\hline
Name          & string or integer & Name of new container type\\
ContainerType & string or integer & Parent container type\\
\hline
Alias         & string or integer & Alternative name of new container type\\
\hline
\end{tabular}%
\end{center}%
\caption{Fields of PajeDefineContainerType event}
\label{f:pajedefinecontainertype}
\end{figure}

For the example, one could need two events (see
%in section~\ref{s:generic}, one would need two events (see
figure~\ref{f:definecontainerexample} to indicate
that a ``Program'' contains ``Thread''s:

\begin{figure}[htbp]
\begin{center}
\begin{tabular}{|LL|}
\hline
\textbf{Field Name} & \textbf{Field Value} \\
\hline
PajeEventName & PajeDefineContainerType \\
Name          & Program\\
ContainerType & /\\
Alias         & P\\
\hline
\end{tabular}%
\quad%\quad
\begin{tabular}{|LL|}
\hline
\textbf{Field Name} & \textbf{Field Value} \\
\hline
PajeEventName & PajeDefineContainerType \\
Name          & Thread\\
ContainerType & P \emph{or} Program\\
Alias         & T\\
\hline
\end{tabular}%
\end{center}%
\caption{Examples of PajeDefineContainerType events}
\label{f:definecontainerexample}
\end{figure}




\subsection{Creation and destruction of containers}
\label{sec:instant}

Containers are created using the ``PajeCreateContainer'' event, and destroyed using
the ``PajeDestroyContainer'' event.

\subsubsection*{PajeCreateContainer\index{PajeCreateContainer}}

This event (see figure~\ref{f:pajecreatecontainer} must have the fields
``Time'', ``Name'', ``Type'' and ``Container''. Optionally
it can have a field named ``Alias''. The simulation of this event instantiates,
in the simulation time ``Time'', a
new container named ``Name'', of type
``Type'', contained in the preexisting
container ``Container''.
The field ``Type'' must have a value corresponding to the ``Name'' or ``Alias''
of a previous
PajeDefineContainerType event. The field ``Container'' must have a value
corresponding to the ``Name'' or ``Alias''
of a previous PajeCreateContainer event (or ``0'' or ``/'', if on top of the hierarchy).
This new container can be referenced in future events by the value of its
``Name'' or, if it has an ``Alias'' field, by it alias.

\begin{figure}[htbp]
\begin{center}
\begin{tabular}{|LLL|}
\hline
\multicolumn{3}{|T|}{\textsf{\textbf{PajeCreateContainer}}}\\\hline
\textbf{Field Name} & \textbf{Field Type} & \textbf{Description}\\
\hline
Time          & date              & Time of creation of container \\
Name          & string or integer & Name of new container \\
Type          & string or integer & Type of new container \\
Container     & string or integer & Parent of new container \\
\hline
Alias         & string or integer & Alternative name of new container \\
\hline
\end{tabular}%
\end{center}%
\caption{Fields of PajeCreateContainer event}
\label{f:pajecreatecontainer}
\end{figure}

Figure~\ref{f:createcontainerexample} shows the events necessary to create 
the containers ``Thread Testing
Program'' of type ``Program'' and ``Thread 1'' and ``Thread 2'' of
type ``Thread'', contained by ``Thread Testing Program''.%, from the example in section~\ref{s:generic}.

\begin{figure}[htbp]
\begin{center}
\begin{tabular}{|LL|}
\hline
\textbf{Field Name} & \textbf{Field Value} \\
\hline
PajeEventName & PajeCreateContainer \\
Time          & 0\\
Name          & "Thread Testing Program"\\
Container     & /\\
Type          & P\\
Alias         & TTP\\
\hline
\end{tabular}%

%\quad%\quad
\begin{tabular}{|LL|}
\hline
\textbf{Field Name} & \textbf{Field Value} \\
\hline
PajeEventName & PajeCreateContainer \\
Time          & 0.986789\\
Name          & "Thread 1"\\
Container     & TTP\\
Type          & T\\
Alias         & T1\\
\hline
\end{tabular}%
\quad%\quad
\begin{tabular}{|LL|}
\hline
\textbf{Field Name} & \textbf{Field Value} \\
\hline
PajeEventName & PajeCreateContainer \\
Time          & 1.012332\\
Name          & "Thread 2"\\
Container     & TTP\\
Type          & T\\
Alias         & T2\\
\hline
\end{tabular}%
\end{center}%
\caption{Examples of PajeCreateContainer events}
\label{f:createcontainerexample}
\end{figure}


\subsubsection*{PajeDestroyContainer\index{PajeDestroyContainer}}

Containers can be destroyed using the event named
``PajeDestroyContainer'' with fields ``Time'', ``Name'' and ``Type'' (see
figure~\ref{f:pajedestroycontainer}.
After simulating this event, the container named (or aliased) ``Name'' of type
``Type'' will be marked as being destroyed at time ``Time''.

\begin{figure}[htbp]
\begin{center}
\begin{tabular}{|LLL|}
\hline
\multicolumn{3}{|T|}{\textsf{\textbf{PajeDestroyContainer}}}\\\hline
\textbf{Field Name} & \textbf{Field Type} & \textbf{Description}\\
\hline
Time          & date              & Time of destruction of container \\
Name          & string or integer & Name of container \\
Type          & string or integer & Type of container \\
\hline
\end{tabular}%
\end{center}%
\caption{Fields of PajeDestroyContainer event}
\label{f:pajedestroycontainer}
\end{figure}

For example, if ``Thread 1'' finishes execution at time 4.34565. it can be
represented by the event in figure~\ref{f:destroycontainerexample}.


\begin{figure}[htbp]
\begin{center}
\begin{tabular}{|LL|}
\hline
\textbf{Field Name} & \textbf{Field Value} \\
\hline
PajeEventName & PajeDestroyContainer \\
Time          & 4.34565\\
Name          & "Thread 1"\\
Type          & T\\
\hline
\end{tabular}%
\end{center}%
\caption{Example of PajeDestroyContainer event}
\label{f:destroycontainerexample}
\end{figure}



\subsection{Definitions of types of entities}
\label{sec:entypedef}

Entities are the leaves of the type hierarchy tree of a Paj�
specialization. There exist four types of entities:
\begin{itemize}
\item \textbf{events}\index{event}, used to represent an event that happened in a
certain point in time, usually displayed as triangles in Paj�'s space-time
diagram;
\item \textbf{states}\index{state}, used to represent the fact that a certain
container was in a determined state during a certain amount of time, usually
displayed as rectangles in Paj�'s space-time diagram;
\item \textbf{links}\index{link}, used to represent a relation between two
containers that started in a certain time and finished in a possibly different
time (for example, a communication between two nodes), usually displayed as arrows; and
\item \textbf{variables}\index{variable}, used to represent the evolution in
time of a
certain value associated to a container, displayed as
graphs in the space-time diagram.
\end{itemize}

An event of type event, state or link can have a value associated with it, and
all possible values must be defined before an event with this value can be
created.
There are four different events to create an entity type in Paj�,
``PajeDefineEventType'', ``PajeDefineStateType'', ``PajeDefineLinkType'' and
``PajeDefineVariableType'' and one event to define a possible value of an
entity, ``PajeDefineEntityValue''.

\subsubsection*{PajeDefineEventType\index{PajeDefineEventType}}

Entities of this new type represent a remarkable type of event
recorded during the visualized executions and are displayed as
triangles in the space-time diagram.  Event types are defined with the
"PajeDefineEventType" event.
This event (see figure~\ref{f:pajedefineevent}) contains the
fields ``Name'' and ``ContainerType''.  It defines a
new event entity type called ``Name'', subtype of the previously defined
container type ``ContainerType''. Optionally it can have a field named ``Alias''
to have an alternative way to identify this type of entity.

\begin{figure}[htbp]
\begin{center}
\begin{tabular}{|LLL|}
\hline
\multicolumn{3}{|T|}{\textsf{\textbf{PajeDefineEventType}}}\\\hline
\textbf{Field Name} & \textbf{Field Type} & \textbf{Description}\\
\hline
Name          & string or integer & Name of new entity type \\
ContainerType & string or integer & Type of container of entity\\
\hline
Alias         & string or integer & Alternative name of new entity type \\
Shape         & string            & Name of shape used to represent entities\\
Height        & integer           & Height of shape, in points\\
Width         & integer           & Width of shape, in points\\
\hline
\end{tabular}%
\end{center}%
\caption{Fields of PajeDefineEventType event}
\label{f:pajedefineevent}
\end{figure}

\subsubsection*{PajeDefineStateType\index{PajeDefineStateType}}

Entities of this new type will represent ``states'', and are displayed
as rectangles in the space-time diagram.  The definition contains (see
figure~\ref{f:pajedefinestate}) the
fields ``Name'' and ``ContainerType''. Optionally, it can have a field
``Alias''.

\begin{figure}[htbp]
\begin{center}
\begin{tabular}{|LLL|}
\hline
\multicolumn{3}{|T|}{\textsf{\textbf{PajeDefineStateType}}}\\\hline
\textbf{Field Name} & \textbf{Field Type} & \textbf{Description}\\
\hline
Name          & string or integer & Name of new entity type \\
ContainerType & string or integer & Type of container of entity\\
\hline
Alias         & string or integer & Alternative name of new entity type \\
Shape         & string            & Name of shape used to represent entities\\
Height        & integer           & Height of shape, in points\\
\hline
\end{tabular}%
\end{center}%
\caption{Fields of PajeDefineStateType event}
\label{f:pajedefinestate}
\end{figure}

In the example of \S\ref{sec:instant}, the event in
figure~\ref{f:definestateexample} could be used to define the entity type that will represent the states of
the threads of the program.

\begin{figure}[htbp]
\begin{center}
\begin{tabular}{|LL|}
\hline
\textbf{Field Name} & \textbf{Field Value} \\
\hline
PajeEventName & PajeDefineStateType \\
Name          & "Thread State"\\
Alias         & S\\
ContainerType & Thread\\
\hline
\end{tabular}%
\end{center}%
\caption{Example of PajeDefineStateType event}
\label{f:definestateexample}
\end{figure}


\subsubsection*{PajeDefineVariableType\index{PajeDefineVariableType}}

Entities of this new type represent variables, whose evolutions are to
be visualized as graphs during the execution of parallel programs.
Variables are created with the \texttt{PajeDefineVariableType} event (see
figure~\ref{f:pajedefinevariable}), containing fields ``Name'',
``ContainerType'' and optionally ``Alias''.
Their value represent an attribute of a container, whose value (a
double) is set by the \texttt{PajeSetVariable} event.

\begin{figure}[htbp]
\begin{center}
\begin{tabular}{|LLL|}
\hline
\multicolumn{3}{|T|}{\textsf{\textbf{PajeDefineVariableType}}}\\\hline
\textbf{Field Name} & \textbf{Field Type} & \textbf{Description}\\
\hline
Name          & string or integer & Name of new entity type \\
ContainerType & string or integer & Type of container of entity\\
\hline
Alias         & string or integer & Alternative name of new entity type \\
Height        & integer           & Height of shape, in points\\
\hline
\end{tabular}%
\end{center}%
\caption{Fields of PajeDefineVariableType event}
\label{f:pajedefinevariable}
\end{figure}

\subsubsection*{PajeDefineLinkType\index{PajeDefineLinkType}}

Links are used to display a directed link between two containers such
as a communication or the identification of a reaction in a container
corresponding to an action on another one.  The source and destination
containers must have a common ancestral in the container hierarchy
(identified by ``Container'' in the events below). Links are usually displayed
as arrows.

\begin{figure}[htbp]
\begin{center}
\begin{tabular}{|LLL|}
\hline
\multicolumn{3}{|T|}{\textsf{\textbf{PajeDefineLinkType}}}\\\hline
\textbf{Field Name} & \textbf{Field Type} & \textbf{Description}\\
\hline
Name          & string or integer & Name of new link type \\
ContainerType & string or integer & Type of common ancestral container \\
SourceContainerType & string or integer & Type of source container of link\\
DestContainerType & string or integer & Type of destination container of link\\
\hline
Alias         & string or integer & Alternative name of new link type \\
Shape         & string            & Name of shape used to represent entities\\
\hline
\end{tabular}%
\end{center}%
\caption{Fields of PajeDefineLinkType event}
\label{f:pajedefinelink}
\end{figure}



\subsubsection*{PajeDefineEntityValue\index{PajeDefineEntityValue}}
\label{sec:entvaldef}

Contains fields ``Name'', ``EntityType'' and optionally ``Alias''.  Used to give
names to the possible values of an entity type.  ``Alias''
will represent the value ``Name'' that entities of type ``EntityType''
can have.  

\begin{figure}[htbp]
\begin{center}
\begin{tabular}{|LLL|}
\hline
\multicolumn{3}{|T|}{\textsf{\textbf{PajeDefineEntityValue}}}\\\hline
\textbf{Field Name} & \textbf{Field Type} & \textbf{Description}\\
\hline
Name          & string or integer & Value of entity \\
EntityType    & string or integer & Type of entity that can have this value \\
\hline
Alias         & string or integer & Alternative name of new value \\
Color         & color             & Color of entities of this value\\
\hline
\end{tabular}%
\end{center}%
\caption{Fields of PajeDefineEntityValue event}
\label{f:pajedefinevalue}
\end{figure}

In the example started in \S\ref{sec:instant}, ``Thread State''s can be
``Executing'' or ``Blocked'', as shown in figure~\ref{f:definevalueexample}.

\begin{figure}[htbp]
\begin{center}
\begin{tabular}{|LL|}
\hline
\textbf{Field Name} & \textbf{Field Value} \\
\hline
PajeEventName & PajeDefineEntityValue \\
Name          & Executing\\
Alias         & E\\
EntityType    & S\\
Color         & "0 1 0"\\
\hline
\end{tabular}%
\quad\begin{tabular}{|LL|}
\hline
\textbf{Field Name} & \textbf{Field Value} \\
\hline
PajeEventName & PajeDefineEntityValue \\
Name          & Blocked\\
Alias         & B\\
EntityType    & S\\
Color         & "0.9 0 0.1"\\
\hline
\end{tabular}%
\end{center}%
\caption{Example of PajeDefineEntityValue event}
\label{f:definevalueexample}
\end{figure}

\subsection{Creation of visualizable entities}
\label{sec:creation}

There are different events to create entities of each possible type
(states, events, variables or links).
In events that create entities, the optional fields named "FileName" and "LineNumber" can be used to relate the created event to a position in a file, that can be obtained in Paj� during the inspection of the entity.
These events can also have the optional field named "RelationKey", to group entities that are somehow related to each other. All entities with the same key are highlighted in the space-time diagram when the mouse cursor is over one of them.

\subsubsection{States}

There are events to change a state
("PajeSetState")\index{PajeSetState}, to push a state, saving the old
state ("PajePushState")\index{PajePushState}, and to pop the
previously saved state ("PajePopState")\index{PajePopState}.

\begin{figure}[htbp]
\begin{center}
\begin{tabular}{|LLL|}
\hline
\multicolumn{3}{|T|}{\textsf{\textbf{PajeSetState}}}\\\hline
\textbf{Field Name} & \textbf{Field Type} & \textbf{Description}\\
\hline
Time          & date              & Time the state changed \\
Type          & string or integer & Type of state \\
Container     & string or integer & Container whose state changed \\
Value         & string or integer & Value of new state of container \\
\hline
\end{tabular}

\begin{tabular}{|LLL|}
\hline
\multicolumn{3}{|T|}{\textsf{\textbf{PajePushState}}}\\\hline
\textbf{Field Name} & \textbf{Field Type} & \textbf{Description}\\
\hline
Time          & date              & Time the state changed \\
Type          & string or integer & Type of state \\
Container     & string or integer & Container whose state changed \\
Value         & string or integer & Value of new state of container \\
\hline
\end{tabular}

\begin{tabular}{|LLL|}
\hline
\multicolumn{3}{|T|}{\textsf{\textbf{PajePopState}}}\\\hline
\textbf{Field Name} & \textbf{Field Type} & \textbf{Description}\\
\hline
Time          & date              & Time the state changed \\
Type          & string or integer & Type of state \\
Container     & string or integer & Container whose state changed \\
\hline
\end{tabular}%
\end{center}%
\caption{Fields of state changing events}
\label{f:pajesetstate}
\end{figure}

For example, if "Thread 1" blocks at time 2.34567 and unblocks at time
2.456789, the trace file could contain the events shown in
figure~\ref{f:setstateexample}.

\begin{figure}[htbp]
\begin{center}
\begin{tabular}{|LL|}
\hline
\textbf{Field Name} & \textbf{Field Value} \\
\hline
PajeEventName & PajeSetState \\
Time          & 2.34567\\
Type          & "Thread State" \\
Container     & "Thread 1"\\
Value         & Blocked \\
\hline
\end{tabular}%
\quad\begin{tabular}{|LL|}
\hline
\textbf{Field Name} & \textbf{Field Value} \\
\hline
PajeEventName & PajeSetState \\
Time          & 2.456789\\
Type          & S \\
Container     & T1\\
Value         & E \\
\hline
\end{tabular}%
\end{center}%
\caption{Example of PajeSetState event}
\label{f:setstateexample}
\end{figure}

\subsubsection{Events}

Events are created with the event named ``PajeNewEvent''\index{PajeNewEvent}.
Just like states, the values of events must be previously
defined by ``PajeDefineEntityValue''.

\begin{figure}[htbp]
\begin{center}
\begin{tabular}{|LLL|}
\hline
\multicolumn{3}{|T|}{\textsf{\textbf{PajeNewEvent}}}\\\hline
\textbf{Field Name} & \textbf{Field Type} & \textbf{Description}\\
\hline
Time          & date              & Time the event happened \\
Type          & string or integer & Type of event \\
Container     & string or integer & Container that produced event \\
Value         & string or integer & Value of new event \\
\hline
\end{tabular}%
\end{center}%
\caption{Fields of PajeNewEvent event}
\label{f:pajenewevent}
\end{figure}


\subsubsection{Variables}

There exist several events to set, add or subtract a value to/from a
variable\index{PajeSetVariable}\index{PajeAddVariable}\index{PajeSubVariable}.

\begin{figure}[htbp]
\begin{center}
\begin{tabular}{|LLL|}
\hline
\multicolumn{3}{|T|}{\textsf{\textbf{PajeSetVariable}}}\\\hline
\textbf{Field Name} & \textbf{Field Type} & \textbf{Description}\\
\hline
Time          & date              & Time the variable changed value\\
Type          & string or integer & Type of variable \\
Container     & string or integer & Container whose value changed \\
Value         & double            & New value of variable \\
\hline
\end{tabular}

\begin{tabular}{|LLL|}
\hline
\multicolumn{3}{|T|}{\textsf{\textbf{PajeAddVariable}}}\\\hline
\textbf{Field Name} & \textbf{Field Type} & \textbf{Description}\\
\hline
Time          & date              & Time the variable changed value\\
Type          & string or integer & Type of variable \\
Container     & string or integer & Container whose value changed \\
Value         & double            & Value to be added to variable \\
\hline
\end{tabular}

\begin{tabular}{|LLL|}
\hline
\multicolumn{3}{|T|}{\textsf{\textbf{PajeSubVariable}}}\\\hline
\textbf{Field Name} & \textbf{Field Type} & \textbf{Description}\\
\hline
Time          & date              & Time the variable changed value\\
Type          & string or integer & Type of variable \\
Container     & string or integer & Container whose value changed \\
Value         & double            & Value to be subtracted from variable \\
\hline
\end{tabular}%
\end{center}%
\caption{Fields of events that change value of variables}
\label{f:pajesetvalue}
\end{figure}


\subsubsection{Links}

A link is defined by two events, a
``PajeStartLink''\index{PajeStartLink} and a
``PajeEndLink''\index{PajeEndLink}.  These two events are matched and
considered to form a link when their respective ``Container'',
``Value'' and ``Key'' fields are the same.

\begin{figure}[htbp]
\begin{center}
\begin{tabular}{|LLL|}
\hline
\multicolumn{3}{|T|}{\textsf{\textbf{PajeStartLink}}}\\\hline
\textbf{Field Name} & \textbf{Field Type} & \textbf{Description}\\
\hline
Time          & date              & Time the link started\\
Type          & string or integer & Type of link \\
Container     & string or integer & Container that has the link \\
SourceContainer&string or integer & Container where link started \\
Value         & string or integer & Value of link \\
Key           & string or integer & Used to match to link end \\
\hline
\end{tabular}

\begin{tabular}{|LLL|}
\hline
\multicolumn{3}{|T|}{\textsf{\textbf{PajeEndLink}}}\\\hline
\textbf{Field Name} & \textbf{Field Type} & \textbf{Description}\\
\hline
Time          & date              & Time the link started\\
Type          & string or integer & Type of link \\
Container     & string or integer & Container that has the link \\
DestContainer & string or integer & Container where link ended \\
Value         & string or integer & Value of link \\
Key           & string or integer & Used to match to link start \\
\hline
\end{tabular}%
\end{center}%
\caption{Fields of events that create links}
\label{f:pajelink}
\end{figure}


\section{Example}
\label{sec:example}

The whole trace file of the example would be:

\begin{verbatim}

%EventDef       PajeDefineContainerType 1
%       Alias           string
%       ContainerType   string
%       Name            string
%EndEventDef
%EventDef       PajeDefineStateType     3
%       Alias           string
%       ContainerType   string
%       Name            string
%EndEventDef
%EventDef       PajeDefineEntityValue   6
%       Alias           string
%       EntityType      string
%       Name            string
%EndEventDef
%EventDef       PajeCreateContainer     7
%       Time            date
%       Alias           string
%       Type            string
%       Container       string
%       Name            string
%EndEventDef
%EventDef       PajeDestroyContainer    8
%       Time            date
%       Name            string
%       Type            string
%EndEventDef
%EventDef       PajeSetState           10
%       Time            date
%       Type            string
%       Container       string
%       Value           string
%EndEventDef
1 P 0 Program
1 T P Thread
3 S T "Thread State"
6 E S Executing
6 B S Blocked
7 0 TTP P 0 "Thread Testing Program"
7 0.986789 T1 T TTP "Thread 1"
10 0.986789 S T1 E
7 1.012332 T2 T TTP "Thread 2"
10 1.012332 S T2 E
10 2.34567 S T1 B
10 2.405678 S T2 B
10 2.456789 S T1 E
10 4.001543 S T2 E
8 4.295677 T2 T
8 4.34565 T1 T
8 4.3498 TTP P

\end{verbatim}


\section{Visualisation of the activity of the processors of a cluster}
\label{sec:admin}

The genericity of Paj� made it possible to visualize the system
activity of the processors of a large-sized (200 PEs) cluster of
personal computers \cite{GuilloudCAS:2001}. Several visualizations
were built from the system information available in the \texttt{/proc}
pseudo-directory of each PE. It was thus possible to represent the
processor and memory load of each PE, the most time consuming process
of each PE., etc. Paj� was also used to visualize the reservations of
PEs by the users of the cluster, the reservations being done using the
PBS system \cite{PBS} (see figure~\ref{fig:pbs}). The main bits of the
Paj� trace file analyzed to produce this figure are given below.

\begin{figure}[ht]
\epsfxsize=\linewidth
%\epsfysize=5cm
%\centerline{\epsfbox{FIG/pbs5.eps}}
\caption{Scheduling of jobs on a large-sized cluster}
\label{fig:pbs}
\end{figure}

\begin{verbatim}
%EventDef       PajeDefineContainerType 1
%       Alias           string
%       ContainerType   string
%       Name            string
%EndEventDef
%EventDef       PajeDefineEventType     2
%       Alias           string
%       ContainerType   string
%       Name            string
%EndEventDef
%EventDef       PajeDefineStateType     3
%       Name            string
%       ContainerType   string
%EndEventDef
%EventDef       PajeDefineEntityValue   6
%       Name            string
%       EntityType      string
%EndEventDef
%EventDef       PajeCreateContainer     7
%       Time            date
%       Alias           string
%       Type            string
%       Container       string
%       Name            string
%EndEventDef
%EventDef       PajeDestroyContainer    8
%       Time            date
%       Name            string
%       Type            string
%EndEventDef
%EventDef       PajeSetState           10
%       Time            date
%       Type            string
%       Container       string
%       Value           string
%EndEventDef
1 MG        0      M-Grappe
1 G         MG     Grappe
1 M         G      Machine
1 CPU       M      Processeur
3 pbs-task  CPU
7 7 MG1     MG     0   M-grappe_1
7 7 G1      G      MG1 Grappe_1
6 nobody pbs-task
6 chapron pbs-task
6 charao pbs-task
6 fchaussum pbs-task
6 feliot pbs-task
6 guilloud pbs-task
6 gustavo pbs-task
6 leblanc pbs-task
6 maillard pbs-task
6 mpillon pbs-task
6 paugerat pbs-task
6 plumejea pbs-task
6 romagnol pbs-task
6 sderr pbs-task         
7 8 M_icluster11  M  G1 M_icluster11
7 8 P_icluster11  CPU  M_icluster11 P_icluster11
7 8 M_icluster21  M  G1 M_icluster21
7 8 P_icluster21  CPU  M_icluster21 P_icluster21
7 8 M_icluster31  M  G1 M_icluster31
7 8 P_icluster31  CPU  M_icluster31 P_icluster31
7 8 M_icluster41  M  G1 M_icluster41

[...]

10 1 pbs-task P_icluster5 nobody
10 4273 pbs-task P_icluster5 nobody
10 5323 pbs-task P_icluster5 plumejea
10 7893 pbs-task P_icluster5 nobody
10 8277 pbs-task P_icluster5 feliot
10 8611 pbs-task P_icluster5 nobody
10 8633 pbs-task P_icluster5 feliot
10 8804 pbs-task P_icluster5 nobody
10 8836 pbs-task P_icluster5 feliot
10 9655 pbs-task P_icluster5 nobody
10 10038 pbs-task P_icluster5 feliot
10 10899 pbs-task P_icluster5 nobody
10 10930 pbs-task P_icluster5 feliot
10 10944 pbs-task P_icluster5 nobody

[...]

10 438224 pbs-task P_icluster100 feliot
10 438278 pbs-task P_icluster100 nobody
10 438339 pbs-task P_icluster100 feliot
10 438713 pbs-task P_icluster100 nobody
10 665686 pbs-task P_icluster100 sderr
10 665727 pbs-task P_icluster100 nobody
8 1465976  P_icluster100 P
8 1465976  M_icluster100 M
8 1465976 G1 G
8 1465976 MG1 MG


\end{verbatim}


%%%%%%%%%%%%%%%%%%%%%%%%%%%%%%%%%%%%%%%%%%%%%%%%%%%%%%%%%%%%%%%%%%%%%%%%%%%%%
%\chapter{Conclusion}
%%%%%%%%%%%%%%%%%%%%%%%%%%%%%%%%%%%%%%%%%%%%%%%%%%%%%%%%%%%%%%%%%%%%%%%%%%%%%
%%%%%%%%%%%%%%%%%%%%%%%%%%%%%%%%%%%%%%%%%%%%%%%%%%%%%%%%%%%%%%%%%%%%%%%%%%%%%%
%\chapter{Conclusion}
%%%%%%%%%%%%%%%%%%%%%%%%%%%%%%%%%%%%%%%%%%%%%%%%%%%%%%%%%%%%%%%%%%%%%%%%%%%%%

Paj� is a versatile visualization tool which can be used in a large
variety of contexts. This report describes the data format used by
Paj�. Paj� being trace-based, the data actually used for the
visualisation is to be presented as a set of execution events. In
addition, a description of the type hierarchy of the visual objects
needs to be included in the data (trace) file. Both the formats of the
type hierarchy description and of the events being self defined, there
also need to be a definition of these formats in the input data
(trace) file used by Paj�.

The versatility property has been used so far to visualize a
distributed Java application and the activity of the nodes of a
large-sized cluster of PCs. 

To enlarge the applicability of Paj�, a translator from traces
produced by Tau \cite{ShendeMCLBK:1998} into the Paj� format is
currently being implemented.




\bibliography{lang-paje}
\bibliographystyle{abbrv}

\addcontentsline{toc}{chapter}{Index}
\IfFileExists{lang-paje.ind}{%
  \documentclass[11pt,twoside]{report}
\usepackage{fullpage}
\usepackage{epsfig}
\usepackage{xspace}
\usepackage{alltt}

% \usepackage[francais]{babel}        % Pour Linux
\usepackage[latin1]{inputenc}
\usepackage[T1]{fontenc}

\setcounter{secnumdepth}{3}  %% pour num�roter les subsubsections
\setcounter{tocdepth}{3}     %% profondeur dans la table des mati�res

\usepackage{times}


\usepackage{amsmath}
\usepackage{pifont} % use of dingbats
\usepackage{array} % extension to the tabular env.
\usepackage{color} % for adding colors to things
\usepackage{colortbl} % for adding colors to tables

\newenvironment{captiontext}{\begin{quote} \footnotesize}{\end{quote}}
\newcommand{\ath}{\textsc{Atha\-pas\-can}\xspace}
\newcommand{\comment}[1]{}
\newlength{\figurewidth}\setlength{\figurewidth}{\textwidth}
%\addtolength{\figurewidth}{-\fboxsep}
%\addtolength{\figurewidth}{-\fboxsep}
\newsavebox{\figurebox}
\newcommand{\makefigure}[4]{%
   \begin{figure}[bt]\centering%
   \sbox{\figurebox}{#2}%
   \ifdim \wd\figurebox >\figurewidth%
      \resizebox{\figurewidth}{!}{\usebox{\figurebox}}%
   \else%
      \makebox[\figurewidth]{\usebox{\figurebox}}%
   \fi%
\vspace*{-3mm}
   \caption[#3]{\parbox[t]{11cm}{#3\newline\raggedright\protect\scriptsize #4}}%
\vspace*{2mm}
   \label{#1}%
   \end{figure}%
}
%% Tables
%
% Usage:
% \maketable{label}{figure command}{main caption}{secondary caption}
%
% there is a new column type for titles, T, to be used in \multicolumns,
% to put titles in reverse color, bold, sans serif, left aligned.
% there are also types R, L and C; they are like r, l, c but sans serif
%
\newlength{\tablewidth}\setlength{\tablewidth}{\textwidth}
\newcommand{\maketable}[4]{%
   \begin{table}[bt]\centering%
   \caption[#3]{#3\newline\footnotesize #4}\label{#1}%
   \sbox{\figurebox}{#2}%
   \ifdim \wd\figurebox >\tablewidth
      \resizebox{\tablewidth}{!}{\usebox{\figurebox}}%
   \else%
      \usebox{\figurebox}%
   \fi%
   \end{table}%
}
\newcolumntype{T}{>{\sffamily\bfseries\color{white}\columncolor[gray]{.2}}l}
\newcolumntype{R}{>{\sffamily}r}
\newcolumntype{L}{>{\sffamily}l}
\newcolumntype{C}{>{\sffamily}c}

\newcommand{\codefigurestart}{
  \begin{figure}[hbt]
    \begin{tabular}{|l|}
      \hline
      \begin{minipage}{\codewidth}
        \medskip\small
}

\newcommand{\codefigureend}[3]{
        \medskip
      \end{minipage}
      \\\hline
    \end{tabular}
%    \caption[#2]{#2\newline\footnotesize #3}%
    \caption[#2]{\parbox[t]{11cm}{#2\newline\raggedright\protect\scriptsize #3}}%
    \label{#1}%
  \end{figure}
}
\newlength{\codewidth}\setlength{\codewidth}{\figurewidth}
\addtolength{\codewidth}{-\fboxsep}
\addtolength{\codewidth}{-\fboxsep}
 % D�finitions de la th�se de Benhur

\makeindex

\title{Paj� trace file format}

\author{B. de Oliveira Stein\\ 
Departamento de Eletr\^onica e Computa\c{c}\~ao\\
Universidade Federal de Santa Maria - RS, Brazil.\\
Email: benhur@inf.UFSM.br\\
\and
J. Chassin de Kergommeaux\\
Laboratoire Informatique et Distribution (ID-IMAG)\\
ENSIMAG - antenne de Montbonnot,\\ ZIRST, 51, avenue Jean Kuntzmann\\
F-38330 Montbonnot Saint Martin, France \\ 
Email:Jacques.Chassin-de-Kergommeaux@imag.fr\\
http://www-apache.imag.fr/\~\/chassin
}

\begin{document}

\maketitle

\begin{abstract}
  
  Paj� is an interactive and scalable trace-based visualization tool
  which can be used for a large variety of visualizations including
  performance monitoring of parallel applications, monitoring the
  execution of processors in a large scale PC cluster or representing
  the behavior of distributed applications. Users of Paj� can tailor
  the visualization to their needs, without having to know any insight
  nor to modify any component of Paj�. This can be done by defining
  the type hierarchy of objects to be visualized as well as how these
  objects should be visualized. This feature allows the use of Paj�
  for a wide variety of visualizations such as the use of resources by
  applications in a large-size cluster or the behavior of distributed
  Java applications.  This report describes the trace format used by
  Paj�. Traces include three different kind of informations:
  definition of the formats of the event, definition of the type
  hierarchy of the objects to be visualized, definition of the formats
  of the events of the trace and a set of recorded events, complying
  with the format definition, to be used to build visualizations
  according to the type hierarchy.

 \textbf{Keywords:} performance debugging, visualization, MPI, pthread, 
parallel programming, self defined data format.

  
\end{abstract}

\tableofcontents

%%%%%%%%%%%%%%%%%%%%%%%%%%%%%%%%%%%%%%%%%%%%%%%%%%%%%%%%%%%%%%%%%%%%%%%%%%%%%
%\chapter{Introduction}
%%%%%%%%%%%%%%%%%%%%%%%%%%%%%%%%%%%%%%%%%%%%%%%%%%%%%%%%%%%%%%%%%%%%%%%%%%%%%
%%%%%%%%%%%%%%%%%%%%%%%%%%%%%%%%%%%%%%%%%%%%%%%%%%%%%%%%%%%%%%%%%%%%%%%%%%%%%%
% \chapter{Introduction}
%%%%%%%%%%%%%%%%%%%%%%%%%%%%%%%%%%%%%%%%%%%%%%%%%%%%%%%%%%%%%%%%%%%%%%%%%%%%%

This report defines the input data format used by the Pajé
visualization tool. Pajé is a versatile trace-based visualization tool
designed to help performance debugging of large-sized parallel
applications. From trace files, recorded during the execution of
parallel programs, Pajé builds a graphical representation of the
behavior of these programs, to help programmers identify their
``performance errors''. The main novelty of Pajé is an original
combination of three of the most desirable properties of visualisation
tools for parallel programs: extensibility, interactivity and
scalability. 

Scalability is the ability to represent the execution of
parallel programs executing during long periods on large-sized
systems; it is provided in Pajé by zooming and filtering
functionalities, both in space --- ability to synthesize the
information originating from several nodes of the system or to zoom in
one of these nodes --- and in time --- possibility to display  period
of time at various levels of detail. Interactivity is the ability to
interrogate visual objects --- events, thread states, communications,
etc. --- to obtain more details or check the source code whose
execution produced a given event; it is also the ability to move back
and forth in time or to zoom from a synthetic representation to a
detailed one or vice versa or to set or reset a filter. Extensibility
is the possibility ot extend the tool with new functionalities ---
visual representations, filters, etc. --- or to display new
programming models. Several characteristics of Pajé contribute to its
extensibility: careful modular design, independence of the
visualization modules from the programming model.

Key to the ability to build a visual representation of the behavior of
parallel programs, developed with various programming models, is the
\textit{genericity} of Pajé: ability to parameterize the tools with a
description of \textit{what} is to be represented and \textit{how}.
This description is provided in the trace file as a hierarchy of the
types of objects appearing in the visualization. The format of this
description as well as the format of the events of the trace are also
described in the trace file. The trace files\index{trace file} used as
input by Pajé thus contain four categories of data:
\begin{enumerate}
\item Description of the format of the generic instructions.
\item Generic instructions, describing the hierarchy of the types of
  objects appearing in the visualization.
\item Description of the format of the events recorded during the
  execution of the visualized program.
\item Events recorded during the execution of the program to be
  visualized.
\end{enumerate}

The aim of this technical report is to describe the Pajé trace data
format. The organization of the report is the following. After this
introduction, the extensibility and genericity of Pajé are described
in detail. The following section defines the Pajé data format and
gives an example of use before the conclusion.


%Pajé is an interactive visualization tool originally designed for
%displaying the execution of parallel applications where a
%(potentially) large number of communicating threads of various
%life-times execute on each node of a distributed memory parallel
%system.   To be easier
%to extend, Pajé was designed as a data-flow graph of modular
%components, most of them being independent of the semantics of the
%parallel programming model of the visualized parallel programs. In
%addition, application programmers can tailor the visualization to
%their needs, without having to know any insight nor to modify any
%component of Pajé. This can be done by defining the type hierarchy of
%objects to be visualized as well as how these objects should be
%visualized.


%%%%%%%%%%%%%%%%%%%%%%%%%%%%%%%%%%%%%%%%%%%%%%%%%%%%%%%%%%%%%%%%%%%%%%%%%%%%%
%\chapter{Extensibility of Paj�}
%\label{chap:paje}
%%%%%%%%%%%%%%%%%%%%%%%%%%%%%%%%%%%%%%%%%%%%%%%%%%%%%%%%%%%%%%%%%%%%%%%%%%%%%
%%%%%%%%%%%%%%%%%%%%%%%%%%%%%%%%%%%%%%%%%%%%%%%%%%%%%%%%%%%%%%%%%%%%%%%%%%%%%%
% \chapter{Extensibility of Pajé}
%%%%%%%%%%%%%%%%%%%%%%%%%%%%%%%%%%%%%%%%%%%%%%%%%%%%%%%%%%%%%%%%%%%%%%%%%%%%%

\section{Introduction}

The Pajé visualization tool described in this article\footnote{This
  chapter was published \emph{in: Euro-Par 2000 Parallel Processing,
    Proc. 6th International Euro-Par Conference}, A.~Bode, W.~Ludwig,
  T.~Karl, R.~Wism\"uller (r\'ed.), \emph{LNCS}, \emph{1900},
  Springer, p.~133--140, 2000.} was designed to allow programmers to
visualize the executions of parallel programs using a potentially
large number of communicating threads (lightweight processes) evolving
dynamically.  The visualization of the executions is an essential tool
to help tuning applications implemented using such a parallel
programming model.

Visualizing a large number of threads raises a number of problems such
as coping with the lack of space available on the screen to visualize
them and understanding such a complex display. The graphical displays
of most existing visualization tools for parallel programs
\cite{Heath:1991,upshot,Kranzlmueller:1996:PPV,PALLAS,pablo,ncstrl.gatech_cc//GIT-CC-95-21,ute}
show the activity of a fixed number of nodes and inter-nodes
communications; it is only possible to represent the activity of a
single thread of control on each of the nodes. It is of course
conceivable to use these systems to visualize the activity of
multi-threaded nodes, representing each thread as a node.  In this
case, the number of threads should be fairly limited and should not
vary during the execution of the program. These visualization tools
are therefore not adapted to visualize threads whose number varies
continuously and life-time is often short.  In addition, these tools
do not support the visualization of local thread synchronizations
using mutexes or semaphores.

Some tools were designed to display multithreaded
programs~\cite{HammondKev1995a,gthread}.  However, they support a
programming model involving a single level of parallelism within a
node, this node being in general a shared-memory multiprocessor. Our
programs execute on several nodes: within the same node, threads
communicate using synchronization primitives; however, threads
executing on different nodes communicate by message passing. Moreover,
compared to these systems, Pajé ought to represent a much larger
number of objects.

The most innovative feature of Pajé is to combine the characteristics
of interactivity and scalability with extensibility. In contrast with
passive visualization tools~\cite{Heath:1991,pablo} where parallel
program entities --- communications, changes in processor states, etc.
--- are displayed as soon as produced and cannot be interrogated, it
is possible to inspect all the objects displayed in the current screen
and to move back in time, displaying past objects again. Scalability
is the ability to cope with a large number of threads. Extensibility
is an important characteristic of visualization tools to cope with the
evolution of parallel programming interfaces and visualization
techniques. Extensibility gives the possibility to extend the
environment with new functionalities: processing of new types of
traces, adding new graphical displays, visualizing new programming
models, etc.

The interactivity and scalability characteristics of Pajé were
described in previous articles
\cite{ChassinS00,ChassinS:2000a,SteinC98}.  This article focuses on
the extensibility characteristics: modular design easing the addition
of new modules, semantics independent modules which allow them to be
used in a large variety of contexts and especially genericity of the
simulator component of Pajé which gives to application programmers the
ability to define what they want to visualize and how it must be done.

The organization of this article is the following. The next section
summarizes the main functionalities of Pajé.  The following section
describes the extensibility of Pajé before the conclusion.


\section{Outline of Pajé}
\label{sec:Pajé}

Pajé was designed to ease performance debugging of \ath programs by
visualizing their executions and because no existing visualization
tool could be used to visualize such multi-threaded programs.

\subsection{\ath: a thread-based parallel programming model}
\label{sec:ath}

Combining threads and communications is increasingly used to program
irregular applications, mask communication or I/O latencies, avoid
communication deadlocks, exploit shared-memory parallelism and
implement remote memory accesses
\cite{Fahringer:1995:UTD,FosterKT96,hicss95}.  The
\ath~\cite{ath0b-europar97} programming model was designed for
parallel hardware systems composed of shared-memory multi-processor
nodes connected by a communication network. It exploits two levels of
parallelism: inter-nodes parallelism and inner parallelism within each
of the nodes. The first type of parallelism is exploited by a fixed
number of system-level processes while the second type is implemented
by a network of communicating threads evolving dynamically. The main
functionalities of \ath are dynamic local or remote thread creation
and termination, sharing of memory space between the threads of the
same node which can synchronize using locks or semaphores, and
blocking or non-blocking message-passing communications between non
local threads, using ports. Combining the main functionalities of MPI
\cite{MPI} with those of \texttt{pthread} compliant libraries, \ath
can be seen as a ``thread aware'' implementation of MPI.

\subsection{Tracing of parallel programs}
\label{sec:tracing}

Execution traces are collected during an execution of the observed
application, using an instrumented version of the \ath\ library. A
non-intrusive, statistical method is used to estimate a precise global
time reference \cite{MailletT:1995}. The events are stored in local
event buffers, which are flushed when full to local event files.  The
collection of events into a single file is only done after the end of
the user's application to avoid interfering with it.  Recorded events
may contain source code information in order to implement source code
click-back --- from visualization to source code --- and click-forward
--- from source code to visualization --- in Pajé.

\subsection{Visualization of threads in Pajé}
\label{s-spacetime}

The visualization of the activity of multi-threaded nodes is mainly
performed in a diagram combining in a single representation the states
and communications of each thread(see figure~\ref{f-spacetime}) .
%
\makefigure{f-spacetime} {\includegraphics{FIG/spacetime-bact-e-2}}
{Visualization of an \ath program execution} {Blocked thread states
  are represented in clear color; runnable states in a dark color. The
  smaller window shows the inspection of an event.}

\makefigure{f-sema} {\includegraphics{FIG/sema-note-2}} {Visualization of
  semaphores} {Note the highlighting of a thread blocked state because
  the mouse pointer is over a semaphore blocked state, and the arrows
  that show the link between a ``V'' operation in a semaphore and the
  corresponding unblocking of a thread.}
%
The horizontal axis represents time while threads are displayed along
the vertical axis, grouped by node. The space allocated to each node
of the parallel system is dynamically adjusted to the number of
visualized threads of this node.  Communications are represented by
arrows while the states of threads are displayed by rectangles. Colors
are used to indicate either the type of a communication, or the
activity of a thread.  It is not the most compact or scalable
representation, but it is very convenient for analyzing detailed
threads relationship, load distribution and masking of communication
latency.  Pajé deals with the scalability problem of this
visualization by means of filters, discussed later in
section~\ref{s-filtering}.

The states of semaphores and locks are represented like the states of
threads: each possible state is associated with a color, and a
rectangle of this color is shown in a position corresponding to the
period of time when the semaphore was in this state. Each lock is
associated with a color, and a rectangle of this color is drawn close
to the thread that holds it (see figure~\ref{f-sema}).

\subsection{Interactivity}

Progresses of the simulation are entirely driven by user-controlled
time displacements: at any time during a simulation, it is possible to
move forward or backward in time, within the limits of the visualized
program execution.  In addition, Pajé offers many possible
interactions to programmers: displayed objects can be inspected to
obtain all the information available for them (see inspection window
in figure~\ref{f-spacetime}), identify related objects or check the
corresponding source code.  Moving the mouse pointer over the
representation of a blocked thread state highlights the corresponding
semaphore state, allowing an immediate recognition (see figure
\ref{f-sema}).  Similarly, all threads blocked in a semaphore are
highlighted when the pointer is moved over the corresponding state of
the semaphore.  From the visual representation of an event, it is
possible to display the corresponding source code line of the parallel
application being visualized.  Likewise, selecting a line in the
source code browser highlights the events that have been generated by
this line.

\subsection{Scalability: filtering of information and zooming capabilities}
\label{s-filtering}

It is not possible to represent simultaneously all the information
that can be deduced from the execution traces.  Screen space
limitation is not the only reason: part of the information may not be
needed all the time or cannot be represented in a graphical way or can
have several graphical representations.  Pajé offers several filtering
and zooming functionalities to help programmers cope with this large
amount of information to give users a simplified, abstract view of the
data. Accessing more detailed information can amount to exploding a
synthetic view into a more detailed view or getting to data that exist
but have not been used or are not directly related to the
visualization. Figure~\ref{f-filters-group} examplifies one of the
filtering facilities provided by Pajé where a single line represents
the number of active threads of a node and a pie graph the CPU
activity in the time slice selected in the space-time diagram (see
\cite{ChassinS00,ChassinS:2000a,Stein:1999}) for more details).

\makefigure{f-filters-group}
           {\includegraphics{FIG/busy+pie}} {CPU utilization} {Grouping
           the threads of each node to display the state of the whole
           system (lighter colors mean more active threads); the
           pie-chart shows the percentage of the selected time slice
           spent with each number of active threads in each node.}

\section{Extensibility}
\label{sec:extensibility}

Extensibility is a key property of a visualization tool. The
main reason is that a visualization tool being a very complex 
piece of software costly to implement, its lifetime ought to
be as long as possible. This will be possible only if the tool
can cope with the evolutions of parallel programming models ---
since this domain is still evolving rapidly --- and of the
visualization techniques. Several characteristics of Pajé were
designed to provide a high degree of extensibility: modular
architecture, flexibility of the visualization modules and genericity
of the simulation module.


\subsection{Modular architecture}
\label{sec:modular}

To favor extensibility, the architecture of Pajé is a data flow graph
of software modules or components (see figure~\ref{f-diagramme}). It
is therefore possible to add a new visualization component or adapt to
a change of trace format by changing the trace reader component
without changing the remaining of the environment.  This architectural
choice was inspired by Pablo \cite{pablo}, although the graph of Pajé
is not purely data-flow for interactivity reasons: it also includes
control-flow information, generated by the visualization modules to
process user interactions and triggering the flow of data in the graph
(see \cite{ChassinS00,ChassinS:2000a,Stein:1999} for more details on
the implementation of interactivity in Pajé).

\makefigure{f-diagramme}
           {\includegraphics{FIG/struct-obj-1b-e}} {Example data-flow
           graph} {The trace reader produces event objects from the
           data read from disk. These events are used by the simulator
           to produce more abstract objects, like thread states,
           communications, etc., traveling on the arcs of the
           data-flow graph to be used by the other components of the
           environment.}


\subsection{Flexibility of visualization modules}
\label{sec:flexibility}

The Pajé visualization components have no dependency whatsoever with
any parallel programming model. Prior to any visualization they
receive as input the description of the types of the objects to be
visualized as well as the relations between these objects and the way
these objects ought to be visualized (see
figure~\ref{fig:hierarchie1}). The only constraints are the
hierarchical nature of the type relations between the visualized
objects and the ability to place each of these objects on the
time-scale of the visualization. The hierarchical type description is
used by the visualization components to query objects from the
preceding components in the graph.

This type description can be changed to adapt to a new programming
model (see section~\ref{sec:genericity}) or during a visualization, to
change the visual representation of an object upon request from the
user. In addition to providing a high versatility for the
visualization components, this feature is used by the filtering
components. When a filter is dynamically inserted in a data-flow graph
--- for example between the simulation and visualization components of
figure~\ref{f-diagramme} to zoom from a detailed visualization to
obtain a more global view of the program execution such as
figure~\ref{f-filters-group} ---, it first sends a type description of
the hierarchy of objects to be visualized to the following components
of the data-flow graph.

\makefigure{fig:hierarchie1} {\includegraphics{FIG/hierarchya}} {Use of a
  simple type hierarchy} {The type hierarchy on the left-hand side of
  the figure defines the type hierarchical relations between the
  objects to be visualized and how how these objects should be
  represented: communications as arrows, thread events as triangles
  and thread states as rectangles.}
  
The type hierarchies used in Pajé are trees whose leaves are called
\textit{entities}\index{entities} and intermediate nodes
\textit{containers}\index{containers}. Entities are elementary objects
that can be displayed such as events, thread states or communications.
Containers are higher level objects used to structure the type
hierarchy\index{type hierarchy} (see figure~\ref{fig:hierarchie1}).
For example: all events occurring in thread~1 of node~0 belong to the
container ``thread-1-of-node-0''.


\subsection{Genericity of Pajé}
\label{sec:genericity}

The modular structure of Pajé as well as the fact that filter and
visualization components are independent of any programming model
makes it ``easy'' for tool developers to add a new component or extend
an existing one. These characteristics alone would not be sufficient
to use Pajé to visualize various programming models if the simulation
component were dependent on the programming model: visualizing a new
programming model would then require to develop a new simulation
component, which is still an important programming effort, reserved
for experienced tool developers.

On the contrary, the generic property of Pajé allows application
programmers to define \textit{what} they would like to visualize and
\textit{how} the visualized objects should be represented by Pajé.
Instead of being computed by a simulation component, designed for a
specific programming model such as \ath, the type hierarchy of the
visualized objects (see section~\ref{sec:flexibility}) can be defined
by inserting several definitions and commands in the trace file (see
format in chapter~\ref{chap:format}). If --- as it is the case with
the \ath-0 tracer and as it is assumed in this paper --- the tracer
(see section~\ref{sec:tracing}) can collect them, these definitions
and command can be inserted in the application program to be traced
and visualized.  The simulator uses these definitions to build a new
data type tree used to relate the objects to be displayed, this tree
being passed to the following modules of the data flow graph: filters
and visualization components.

\subsubsection{New data types definition.}

One function call is available to create new types of containers while
four can be used to create new types of entities which can be events,
states, links and variables. An ``event''\index{event} is an entity
representing an instantaneous action. ``States''\index{state} of
interest are those of containers.  A ``link''\index{link} represents
some form of connection between a source and a destination container.
A ``variable''\index{variable} stores the temporal evolution of the
successive values of a data associated with a container.
Table~\ref{t:defusertypes} contains the function calls that can be
used to define new types of containers and entities.
Figure~\ref{fig:hierarchie2} shows the effect of adding the
``threads'' container to the type hierarchy of
figure~\ref{fig:hierarchie1}.

\maketable{t:defusertypes}
{\small
\begin{tabular}{|>{\scshape}RL>{\scshape}L|}
\hline
\multicolumn{1}{|T}{Result}&
\multicolumn{1}{T}{Call}&
\multicolumn{1}{T|}{Parameters}  \\
\hline
ctype     & pajeDefineUserContainerType & ctype name                        \\
\hline
\hline
etype     & pajeDefineUserEventType     & ctype name                        \\
etype     & pajeDefineUserStateType     & ctype name                        \\
etype     & pajeDefineUserLinkType      & ctype name                        \\
etype     & pajeDefineUserVariableType  & ctype name                        \\
\hline
evalue    & pajeNewUserEntityValue      & etype name                        \\
\hline
\end{tabular}
} {Containers\index{container} and entities\index{entity} types
definitions} {The argument \textsc{\textsf{ctype}} is the type of the
father container of the newly defined type, in the type hierarchy (the
container ``Execution'' being always the root of the tree of types).}

%
\makefigure{fig:hierarchie2} {\includegraphics{FIG/hierarchyb}}{Adding
  a container to the type hierarchy\index{type hierarchy} of
  figure~\ref{fig:hierarchie1}}{}
  
\subsubsection{Data generation.}

Several functions can be used to create containers\index{container}
and entities\index{entity} whose types were defined using
table~\ref{t:defusertypes} primitives.  Specific functions are used to
create events, states (and embedded states using \textit{Push} and
\textit{Pop}), links --- each link being created by one source and one
destination calls, the coupling between them being performed by the
simulator when parameters \texttt{container, evalue} and \texttt{key}
of both source and destination calls match --- and change the values
of variables (see table~\ref{t:creationuser}).
%
\maketable{t:creationuser}
{\small
\begin{tabular}{|>{\scshape}RL>{\scshape}L|}
\hline
\multicolumn{1}{|T}{Result}&
\multicolumn{1}{T}{Call}&
\multicolumn{1}{T|}{Parameters}  \\
\hline
container & pajeCreateUserContainer     & ctype name in-container           \\
          & pajeDestroyUserContainer    & container                         \\
\hline
\hline
          & pajeUserEvent               & etype container evalue comment    \\
\hline
          & pajeSetUserState            & etype container evalue comment    \\
          & pajePushUserState           & etype container evalue comment    \\
          & pajePopUserState            & etype container comment           \\
\hline
          & pajeStartUserLink           & etype container srccontainer      \\
          &                           & \quad \quad evalue key comment    \\
          & pajeEndUserLink             & etype container destcontainer     \\
          &                           & \quad \quad evalue key comment    \\
\hline
          & pajeSetUserVariable         & etype container value comment     \\
          & pajeAddUserVariable         & etype container value comment     \\
\hline
\end{tabular}
} {Creation of containers and entities} {Calls to these functions are
inserted in the traced application to generate ``user events'' whose
processing by the Pajé simulator will use the type tree built from the
containers and entities types definitions done using the functions of
table~\ref{t:defusertypes}.}

In the example of figure~\ref{f:simpleprogramtraced}, a new event is
generated for each change of computation phase. This event is
interpreted by the Pajé simulator component to generate the
corresponding container state. For example the following call
indicates that the computation is entering in a ``Local computation''
phase: \\
{\small\tt\verb"  pajeSetUserState ( phase_state, node, local_phase, str_iter );"}\\
The second parameter indicates the container of the state (the
``node'' whose computation has just been changed).  The last parameter
is a comment that can be visualized by Pajé. In the example it is used
to display the current iteration value. The example program of
figure~\ref{f:simpleprogramtraced} includes the definitions and
creations of entities ``Computation phase'', allowing the visual
representation of an \ath program execution to be extended to
represent the phases of the computation.
Figure~\ref{f:simpleprogramvisu} includes two space-time diagrams
visualizing the execution of this example program, without and with
the definition of the new entities.
 
\codefigurestart{\scriptsize\alltt
 \textbf{unsigned phase_state, init_phase, local_phase, global_phase;
 phase_state  = pajeDefineUserStateType( A0_NODE, "Computation phase");
 init_phase   = pajeNewUserEntityValue( phase_state, "Initialization");
 local_phase  = pajeNewUserEntityValue( phase_state, "Local computation");
 global_phase = pajeNewUserEntityValue( phase_state,"Global computation");

 pajeSetUserState ( phase_state, node, init_phase, "" );}
 initialization();
 while (!converge) \{
     iter++;
     str_iter = itoa (iter);
     \textbf{pajeSetUserState ( phase_state, node, local_phase, str_iter );}
     local_computation();
     send (local_data);
     receive (remote_data);
     \textbf{pajeSetUserState ( phase_state, node, global_phase, str_iter );}
     global_computation();
 \}
}\codefigureend{f:simpleprogramtraced}
{Simplified algorithm of the example program}
{The five first lines written in bold face contain the generic
  instructions that have to be passed to Pajé through the trace file,
  to define a new type of state for the container \texttt{A0\_NODE}.
  Here it is assumed that the tracer is able to record these
  instructions in addition to the events of the program (
  \texttt{pajeSetUserState}).}
%
\makefigure{f:simpleprogramvisu} {\includegraphics{FIG/simpleprogram-2}}
{Visualization of the example program} {The second figure displays the
  entities ``Computation phases'' defined by the end-user. It is also
  possible to restrict the visualization to this information alone.}


\section{Conclusion}
\label{sec:conc}

Pajé provides solutions to interactively visualize the execution of
parallel applications using a varying number of threads communicating
by shared memory within each node and by message passing between
different nodes.  The most original feature of the tool is its unique
combination of extensibility, interactivity and scalability
properties. Extensibility means that the tool was defined to allow
tool developers to add new functionalities or extend existing ones
without having to change the rest of the tool. In addition, it is
possible to application programmers using the tool to define what they
wish to visualize and how this should be represented. To our knowledge
such a generic feature was not present in any previous visualization
tool for parallel programs executions.

The genericity property of Pajé was used to visualize \ath-1 programs
executions without having to perform any new development in Pajé.
\ath-1 is a high level parallel programming model where parallelism is
expressed by asynchronous task creations whose scheduling is performed
automatically by the run-time system \cite{a1-europar98,a1-pact98}.
The runtime system of \ath-1 is implemented using \ath. By extending
the type hierarchy defined for \ath and inserting few instructions to
the \ath-1 implementation, it was possible to visualize where the time
was spent during \ath-1 computations: computing the user program,
managing the task graph or scheduling the user-defined tasks.

Further developments include simplifying the generic description and
creation of visual objects, currently more complex when the generic
simulator is used instead of a specialized one. The generation of
traces for other thread-based programming models such as Java will
also be investigated to further validate the flexibility of Pajé.



%%%%%%%%%%%%%%%%%%%%%%%%%%%%%%%%%%%%%%%%%%%%%%%%%%%%%%%%%%%%%%%%%%%%%%%%%%%%%
\chapter{Definition of type hierarchies and trace event formats}
\label{chap:format}
%%%%%%%%%%%%%%%%%%%%%%%%%%%%%%%%%%%%%%%%%%%%%%%%%%%%%%%%%%%%%%%%%%%%%%%%%%%%%
%*
%   Copyright 1998, 1999, 2000, 2001, 2003, 2004 Benhur Stein
%   
%   This file is part of Paj�.
%
%   Paj� is free software; you can redistribute it and/or modify
%   it under the terms of the GNU General Public License as published by
%   the Free Software Foundation; either version 2 of the License, or
%   (at your option) any later version.
%
%   Foobar is distributed in the hope that it will be useful,
%   but WITHOUT ANY WARRANTY; without even the implied warranty of
%   MERCHANTABILITY or FITNESS FOR A PARTICULAR PURPOSE.  See the
%   GNU General Public License for more details.
%
%   You should have received a copy of the GNU General Public License
%   along with Foobar; if not, write to the Free Software
%   Foundation, Inc., 59 Temple Place, Suite 330, Boston, MA  02111-1307  USA
%/
%%%%%%%%%%%%%%%%%%%%%%%%%%%%%%%%%%%%%%%%%%%%%%%%%%%%%%%%%%%%%%%%%%%%%%%%%%%%%
% \chapter{Definition of type hierarchies and trace event formats}
%%%%%%%%%%%%%%%%%%%%%%%%%%%%%%%%%%%%%%%%%%%%%%%%%%%%%%%%%%%%%%%%%%%%%%%%%%%%%

% T, R, L and C already in definitions.tex
%\newcolumntype{T}{>{\sffamily\bfseries\color{white}\columncolor[gray]{.2}}l}
%\newcolumntype{R}{>{\sffamily}r}
%\newcolumntype{L}{>{\sffamily}l}
%\newcolumntype{C}{>{\sffamily}c}
\newcolumntype{P}{>{\sffamily}p{5cm}}

\section{Introduction}
\label{sec:traceintro}

A visualization constructed by Paj� is composed of objects organized
according to a tree type hierarchy whose nodes are called
\emph{containers}\index{container} and leaves
\emph{entities}\index{entity}% (see \S\ref{sec:genericity})
. The Paj�
data format is self-defined, although it does not comply with the
SDDF\index{SDDF} format used by Pablo \cite{sddf}. There exists a
``meta-format'' used to define:
\begin{itemize}
\item The format of the instructions defining containers and entities.
\item The format of the events recorded during the executions of
  parallel programs.
\end{itemize}

These definitions are usually inserted in trace files. They can even
be inserted in the observed programs source files, provided that the
tracers used to record the events of these programs are able to
capture a new definition as a ``user-defined'' event.% (see for example
%figure~\ref{f:simpleprogramtraced}).

Using these definitions, it is possible to define a hierarchy of
containers and entities adapted for a given programming model or
language. Definitions of type hierarchies as well as instructions and
events formats constitute a specialisation of the ``generic'' Paj�
visualization tool: it has been used so far to visualize distributed
applications written in Java \cite{OttogaliOSCV:2001} or help to
perform system monitoring on large sized clusters
\cite{GuilloudCAS:2001}. 

The organization of this chapter is the following. The next section
defines the meta format of Paj� used to define the format of type
definition instructions and events. The following sections describe
how containers and entities are defined and created. The next section
is dedicated to the trace events: self definition, recording. The last
section of this chapter contains a complete example of use of the Paj�
data format.

\section{Meta format of Paj�}
\label{sec:file}

A trace file is composed of events.
An event can be seen as a table composed of named fields, as shown in figure~\ref{f:event:table}. 
The first event in the figure can represent the sending of a message containing 320 bytes by process 5 to process 3, containing 320 bytes by process 5 to process 3, containing 320 bytes by process 5 to process 3, containing 320 bytes by process 5 to process 3, 3.233222 seconds after the process started executing.
The second event shows that process 5 unblocked at time 5.123002, and that this happened while executing line 98 of file sync.c.
Each event has some fields, each of them composed of a name, a type and a value. Generally, there are lots of similar events in a trace file (lots of ``SendMessage'' events, all with the same fields); a typical trace file contains thousands of events of tens of different types.
Usually, events of the same type have the same fields.
It is therefore wise, in order to reduce the trace file size, not to put the
information that is common to many events in each of those events.
The most common solution is to put only the type of each event and the values of its fields in the trace file. Information on what event types exist and the fields that constitute each of these event types being kept elsewhere.
In some trace file formats, this information is hardcoded in the trace generator and trace reader, making the trace structure hard to change in order to incorporate new types of events, new data in existing events or to remove unused or unknown data from those events.

\begin{figure}[htbp]
\begin{center}
\begin{tabular}{|>{\bf}rll|}
\hline
\textbf{Field Name} & \textbf{Field Type} & \textbf{Field Value} \\
\hline
EventName     & string    & SendMessage \\
Time          & timestamp & 3.233222    \\
ProcessId     & integer   & 5           \\
Receiver      & integer   & 3           \\
Size          & integer   & 320         \\
\hline
\end{tabular}
\quad\quad
\begin{tabular}{|>{\bf}rll|}
\hline
\textbf{Field Name} & \textbf{Field Type} & \textbf{Field Value} \\
\hline
EventName     & string    & UnblockProcess \\
Time          & timestamp & 5.123002    \\
ProcessId     & integer   & 5           \\
FileName      & string    & sync.c      \\
LineNumber    & integer   & 98          \\
\hline
\end{tabular}
\end{center}
\caption{Examples of events}
\label{f:event:table}
\end{figure}

A Paj� trace file is self defined, meaning that the event definition information is put inside the trace file itself, much like the SDDF file format used by the Pablo visualization tool \cite{sddf}.
The file is constituted of two parts: the definition of the events at the beginning of the file followed by the events themselves.
The definition of events contains the name of each event type and the names and types of each field.
The second part of the trace file contains the events, with the values associated to each field, in the same order as in the definition.
The correspondence of an event with its definition is made by means of a number, that must be unique for each event description; this number appears in an event definition and at the beginning of each event contained in the trace file.

The event definition part of a Paj� trace file follows the following format:
\begin{itemize}
\item all the lines start with a `\%' character;
\item each event definition starts with a \%EventDef\index{EventDef}
  line and terminates with a \%EndEventDef\index{EndEventDef} line;
\item the \%EventDef line contains the name and the unique number of an
  event type.  The number (an integer) will be used to identify the event
  in the second part of the trace file. The choice of this number is
  left to the user. The numbers given in the definitions below (see
%  \S\ref{sec:containers} and \S\ref{sec:entities}
  \S\ref{sec:example}) are thus
  \textbf{arbitrary}. The name of the event will be put in a field called
  ``PajeEventName''. There cannot be another field called so. The name is used
  to identify the type of an event;
\item the fields of an event are defined between the \%EventDef and
  the \%EndEventDef lines, one field per line, with the name
  of the field followed by its type (see below).
\end{itemize}

The structure of the two events of figure~\ref{f:event:table} are shown in figure~\ref{f:event:def}.

\begin{figure}
\begin{center}
\begin{minipage}{5.6cm}
\begin{verbatim}
%EventDef SendMessage 21
%   Time       date
%   ProcessId  int
%   Receiver   int
%   Size       int
%EndEventDef
\end{verbatim}
\end{minipage}
\quad\quad
\begin{minipage}{5.6cm}
\begin{verbatim}
%EventDef UnblockProcess 17
%   Time       date
%   ProcessId  int
%   LineNumber int
%   FileName   string
%EndEventDef
\end{verbatim}
\end{minipage}
\end{center}
\caption{Examples of event definitions}
\label{f:event:def}
\end{figure}

The type of a field can be one of the following:
\begin{description}
  \item [date:] for fields that represent dates\index{date}.
                It's a double precision floating-point number, usually meaning seconds since program start;
  \item [int:] for fields containing integer numeric values;
  \item [double:] for fields containing floating-point values;
  \item [hex:] for fields that represent addresses, in hexadecimal;
  \item [string:] for strings of characters.
  \item [color:] for fields that represent colors. A color is a sequence of
                 three floating-point numbers between 0 and 1, inside double 
                 quotes (").
                 The three numbers are the values of red, green and blue
                 components.
\end{description}

%Most events are dated. If that is the case, it must be defined with a field
named ``Time'' of type ``date'', for
%the date of generation of the event. 

The second part of the trace file contains one event per line, whose 
fields are separated by spaces or tabs, the first field being the number that
identifies the event type, followed by the other fields, in the same order that
they appear in the
definition of the event. 
Fields of type string must be inside double quotes (") if they contain space or
tab characters, or if they are empty.

For example, the two events of figure~\ref{f:event:table} are shown in figure~\ref{f:event}.
\begin{figure}
\begin{center}
\begin{minipage}{5cm}
\begin{verbatim}
21 3.233222 5 3 320
17 5.123002 5 98 sync.c
\end{verbatim}
\end{minipage}
\end{center}
\caption{Examples of events}
\label{f:event}
\end{figure}

In Paj�, event numbers are used only as a means to find the definition
of an event; they are discarded as soon as an event is read.
After being read, events are identified by their names.
Two different definitions can have the same name (and different numbers), making it possible to have, in the same trace file, two events of the same type containing different fields.
We use this feature to optionally include the source file identification in some
events. The ``UnblockProcess'' event in the examples above could also be defined without the fields FileName and LineNumber, for use in places where this information is not known or not necessary.

\section{Events treated by the Paj� simulator} %{Paj� "generic" events}
\label{sec:generic}

A Paj� visualization is best described as a typed hierarchy of objects
organized as a tree. Elementary objects are the leaves of the tree and
called ``entities'' while intermediate nodes of the tree are named
``containers''. 
Entities are the objects that can be visualized in Paj�'s space-time diagram,
while containers organize the space where those entities are displayed.

Paj� includes a simulator module which builds this hierarchical data
structure from the elementary event records of the trace files. 
Paj� has no
predefined containers or entities. 
Before an entity can be created and visualized, a hierarchy of container and
entity types must be defined, and containers must be instantiated.

For example, to visualize the states of threads in a program, one must
first define the container types ``Program'' and ``Thread'' and the entity
type ``Thread State''.  One must also define the possible values that
the entities of type ``Thread State'' can assume (for example,
``Executing'' and ``Blocked'').  Then, one must instantiate the program
creating a container of type ``Program'' (called ``Thread Testing
Program'', for example).  The threads of the program also have to be
instantiated; they are containers of type ``Thread'', called for example
``Thread 1''
and ``Thread 2'', and contained in container ``Thread Testing Program''.
Only then one is able to create visualizable entities of type ``Thread
State'', by means of events that represent changes in state, contained
either in ``Thread 1'' or ``Thread 2''.


The events that the Paj� simulator understands can be divided into four classes:
\begin{itemize}

\item events to define types of containers;
\item events to define types of entities and possible
values that entities can have; 
\item events to instantiate and destroy containers;
\item events to create visualizable entities.
\end{itemize}

Typically, the events of the first two classes are in the beginning of a trace
file, followed by events that instantiate containers, followed by a large number of events creating entities.
The simulator does
not impose this order, events of these four classes can be mixed in the trace
file. The limitation is that an entity or a container cannot be created before
its type has been defined and its container created.

The four classes of events are discussed in the following sessions.

\subsection{Definition of types of Containers}
\label{sec:contype}

Containers types are defined with events named ``PajeDefineContainerType''.

\subsubsection*{PajeDefineContainerType\index{PajeDefineContainerType}}

Events of this type (see figure~\ref{f:pajedefinecontainertype}) must contain the fields  ``Name'' and ``ContainerType''.
It defines a new container type called ``Name'', contained by a previously
defined container type ``ContainerType'' (or the special container type ``0'' or
``/'', if this container type is
the top of the container hierarchy).
Optionally this event can contain a field ``Alias'' with an alias name to be
used to identify this container. Aliases are usually short strings used when the
container name is too big and its use throughout the trace file would increase
the file's size.
When an alias is used in a definition, it must also be used in later references to the container type being defined.
When an alias is not used, a container type must be later referenced by its name.
The use of aliases allows for the definition of more than one container with the same name (and different aliases).

\begin{figure}[htbp]
\begin{center}
\begin{tabular}{|LLL|}
\hline
\multicolumn{3}{|T|}{\textsf{\textbf{PajeDefineContainerType}}}\\\hline
\textbf{Field Name} & \textbf{Field Type} & \textbf{Description}\\
\hline
Name          & string or integer & Name of new container type\\
ContainerType & string or integer & Parent container type\\
\hline
Alias         & string or integer & Alternative name of new container type\\
\hline
\end{tabular}%
\end{center}%
\caption{Fields of PajeDefineContainerType event}
\label{f:pajedefinecontainertype}
\end{figure}

For the example, one could need two events (see
%in section~\ref{s:generic}, one would need two events (see
figure~\ref{f:definecontainerexample} to indicate
that a ``Program'' contains ``Thread''s:

\begin{figure}[htbp]
\begin{center}
\begin{tabular}{|LL|}
\hline
\textbf{Field Name} & \textbf{Field Value} \\
\hline
PajeEventName & PajeDefineContainerType \\
Name          & Program\\
ContainerType & /\\
Alias         & P\\
\hline
\end{tabular}%
\quad%\quad
\begin{tabular}{|LL|}
\hline
\textbf{Field Name} & \textbf{Field Value} \\
\hline
PajeEventName & PajeDefineContainerType \\
Name          & Thread\\
ContainerType & P \emph{or} Program\\
Alias         & T\\
\hline
\end{tabular}%
\end{center}%
\caption{Examples of PajeDefineContainerType events}
\label{f:definecontainerexample}
\end{figure}




\subsection{Creation and destruction of containers}
\label{sec:instant}

Containers are created using the ``PajeCreateContainer'' event, and destroyed using
the ``PajeDestroyContainer'' event.

\subsubsection*{PajeCreateContainer\index{PajeCreateContainer}}

This event (see figure~\ref{f:pajecreatecontainer} must have the fields
``Time'', ``Name'', ``Type'' and ``Container''. Optionally
it can have a field named ``Alias''. The simulation of this event instantiates,
in the simulation time ``Time'', a
new container named ``Name'', of type
``Type'', contained in the preexisting
container ``Container''.
The field ``Type'' must have a value corresponding to the ``Name'' or ``Alias''
of a previous
PajeDefineContainerType event. The field ``Container'' must have a value
corresponding to the ``Name'' or ``Alias''
of a previous PajeCreateContainer event (or ``0'' or ``/'', if on top of the hierarchy).
This new container can be referenced in future events by the value of its
``Name'' or, if it has an ``Alias'' field, by it alias.

\begin{figure}[htbp]
\begin{center}
\begin{tabular}{|LLL|}
\hline
\multicolumn{3}{|T|}{\textsf{\textbf{PajeCreateContainer}}}\\\hline
\textbf{Field Name} & \textbf{Field Type} & \textbf{Description}\\
\hline
Time          & date              & Time of creation of container \\
Name          & string or integer & Name of new container \\
Type          & string or integer & Type of new container \\
Container     & string or integer & Parent of new container \\
\hline
Alias         & string or integer & Alternative name of new container \\
\hline
\end{tabular}%
\end{center}%
\caption{Fields of PajeCreateContainer event}
\label{f:pajecreatecontainer}
\end{figure}

Figure~\ref{f:createcontainerexample} shows the events necessary to create 
the containers ``Thread Testing
Program'' of type ``Program'' and ``Thread 1'' and ``Thread 2'' of
type ``Thread'', contained by ``Thread Testing Program''.%, from the example in section~\ref{s:generic}.

\begin{figure}[htbp]
\begin{center}
\begin{tabular}{|LL|}
\hline
\textbf{Field Name} & \textbf{Field Value} \\
\hline
PajeEventName & PajeCreateContainer \\
Time          & 0\\
Name          & "Thread Testing Program"\\
Container     & /\\
Type          & P\\
Alias         & TTP\\
\hline
\end{tabular}%

%\quad%\quad
\begin{tabular}{|LL|}
\hline
\textbf{Field Name} & \textbf{Field Value} \\
\hline
PajeEventName & PajeCreateContainer \\
Time          & 0.986789\\
Name          & "Thread 1"\\
Container     & TTP\\
Type          & T\\
Alias         & T1\\
\hline
\end{tabular}%
\quad%\quad
\begin{tabular}{|LL|}
\hline
\textbf{Field Name} & \textbf{Field Value} \\
\hline
PajeEventName & PajeCreateContainer \\
Time          & 1.012332\\
Name          & "Thread 2"\\
Container     & TTP\\
Type          & T\\
Alias         & T2\\
\hline
\end{tabular}%
\end{center}%
\caption{Examples of PajeCreateContainer events}
\label{f:createcontainerexample}
\end{figure}


\subsubsection*{PajeDestroyContainer\index{PajeDestroyContainer}}

Containers can be destroyed using the event named
``PajeDestroyContainer'' with fields ``Time'', ``Name'' and ``Type'' (see
figure~\ref{f:pajedestroycontainer}.
After simulating this event, the container named (or aliased) ``Name'' of type
``Type'' will be marked as being destroyed at time ``Time''.

\begin{figure}[htbp]
\begin{center}
\begin{tabular}{|LLL|}
\hline
\multicolumn{3}{|T|}{\textsf{\textbf{PajeDestroyContainer}}}\\\hline
\textbf{Field Name} & \textbf{Field Type} & \textbf{Description}\\
\hline
Time          & date              & Time of destruction of container \\
Name          & string or integer & Name of container \\
Type          & string or integer & Type of container \\
\hline
\end{tabular}%
\end{center}%
\caption{Fields of PajeDestroyContainer event}
\label{f:pajedestroycontainer}
\end{figure}

For example, if ``Thread 1'' finishes execution at time 4.34565. it can be
represented by the event in figure~\ref{f:destroycontainerexample}.


\begin{figure}[htbp]
\begin{center}
\begin{tabular}{|LL|}
\hline
\textbf{Field Name} & \textbf{Field Value} \\
\hline
PajeEventName & PajeDestroyContainer \\
Time          & 4.34565\\
Name          & "Thread 1"\\
Type          & T\\
\hline
\end{tabular}%
\end{center}%
\caption{Example of PajeDestroyContainer event}
\label{f:destroycontainerexample}
\end{figure}



\subsection{Definitions of types of entities}
\label{sec:entypedef}

Entities are the leaves of the type hierarchy tree of a Paj�
specialization. There exist four types of entities:
\begin{itemize}
\item \textbf{events}\index{event}, used to represent an event that happened in a
certain point in time, usually displayed as triangles in Paj�'s space-time
diagram;
\item \textbf{states}\index{state}, used to represent the fact that a certain
container was in a determined state during a certain amount of time, usually
displayed as rectangles in Paj�'s space-time diagram;
\item \textbf{links}\index{link}, used to represent a relation between two
containers that started in a certain time and finished in a possibly different
time (for example, a communication between two nodes), usually displayed as arrows; and
\item \textbf{variables}\index{variable}, used to represent the evolution in
time of a
certain value associated to a container, displayed as
graphs in the space-time diagram.
\end{itemize}

An event of type event, state or link can have a value associated with it, and
all possible values must be defined before an event with this value can be
created.
There are four different events to create an entity type in Paj�,
``PajeDefineEventType'', ``PajeDefineStateType'', ``PajeDefineLinkType'' and
``PajeDefineVariableType'' and one event to define a possible value of an
entity, ``PajeDefineEntityValue''.

\subsubsection*{PajeDefineEventType\index{PajeDefineEventType}}

Entities of this new type represent a remarkable type of event
recorded during the visualized executions and are displayed as
triangles in the space-time diagram.  Event types are defined with the
"PajeDefineEventType" event.
This event (see figure~\ref{f:pajedefineevent}) contains the
fields ``Name'' and ``ContainerType''.  It defines a
new event entity type called ``Name'', subtype of the previously defined
container type ``ContainerType''. Optionally it can have a field named ``Alias''
to have an alternative way to identify this type of entity.

\begin{figure}[htbp]
\begin{center}
\begin{tabular}{|LLL|}
\hline
\multicolumn{3}{|T|}{\textsf{\textbf{PajeDefineEventType}}}\\\hline
\textbf{Field Name} & \textbf{Field Type} & \textbf{Description}\\
\hline
Name          & string or integer & Name of new entity type \\
ContainerType & string or integer & Type of container of entity\\
\hline
Alias         & string or integer & Alternative name of new entity type \\
Shape         & string            & Name of shape used to represent entities\\
Height        & integer           & Height of shape, in points\\
Width         & integer           & Width of shape, in points\\
\hline
\end{tabular}%
\end{center}%
\caption{Fields of PajeDefineEventType event}
\label{f:pajedefineevent}
\end{figure}

\subsubsection*{PajeDefineStateType\index{PajeDefineStateType}}

Entities of this new type will represent ``states'', and are displayed
as rectangles in the space-time diagram.  The definition contains (see
figure~\ref{f:pajedefinestate}) the
fields ``Name'' and ``ContainerType''. Optionally, it can have a field
``Alias''.

\begin{figure}[htbp]
\begin{center}
\begin{tabular}{|LLL|}
\hline
\multicolumn{3}{|T|}{\textsf{\textbf{PajeDefineStateType}}}\\\hline
\textbf{Field Name} & \textbf{Field Type} & \textbf{Description}\\
\hline
Name          & string or integer & Name of new entity type \\
ContainerType & string or integer & Type of container of entity\\
\hline
Alias         & string or integer & Alternative name of new entity type \\
Shape         & string            & Name of shape used to represent entities\\
Height        & integer           & Height of shape, in points\\
\hline
\end{tabular}%
\end{center}%
\caption{Fields of PajeDefineStateType event}
\label{f:pajedefinestate}
\end{figure}

In the example of \S\ref{sec:instant}, the event in
figure~\ref{f:definestateexample} could be used to define the entity type that will represent the states of
the threads of the program.

\begin{figure}[htbp]
\begin{center}
\begin{tabular}{|LL|}
\hline
\textbf{Field Name} & \textbf{Field Value} \\
\hline
PajeEventName & PajeDefineStateType \\
Name          & "Thread State"\\
Alias         & S\\
ContainerType & Thread\\
\hline
\end{tabular}%
\end{center}%
\caption{Example of PajeDefineStateType event}
\label{f:definestateexample}
\end{figure}


\subsubsection*{PajeDefineVariableType\index{PajeDefineVariableType}}

Entities of this new type represent variables, whose evolutions are to
be visualized as graphs during the execution of parallel programs.
Variables are created with the \texttt{PajeDefineVariableType} event (see
figure~\ref{f:pajedefinevariable}), containing fields ``Name'',
``ContainerType'' and optionally ``Alias''.
Their value represent an attribute of a container, whose value (a
double) is set by the \texttt{PajeSetVariable} event.

\begin{figure}[htbp]
\begin{center}
\begin{tabular}{|LLL|}
\hline
\multicolumn{3}{|T|}{\textsf{\textbf{PajeDefineVariableType}}}\\\hline
\textbf{Field Name} & \textbf{Field Type} & \textbf{Description}\\
\hline
Name          & string or integer & Name of new entity type \\
ContainerType & string or integer & Type of container of entity\\
\hline
Alias         & string or integer & Alternative name of new entity type \\
Height        & integer           & Height of shape, in points\\
\hline
\end{tabular}%
\end{center}%
\caption{Fields of PajeDefineVariableType event}
\label{f:pajedefinevariable}
\end{figure}

\subsubsection*{PajeDefineLinkType\index{PajeDefineLinkType}}

Links are used to display a directed link between two containers such
as a communication or the identification of a reaction in a container
corresponding to an action on another one.  The source and destination
containers must have a common ancestral in the container hierarchy
(identified by ``Container'' in the events below). Links are usually displayed
as arrows.

\begin{figure}[htbp]
\begin{center}
\begin{tabular}{|LLL|}
\hline
\multicolumn{3}{|T|}{\textsf{\textbf{PajeDefineLinkType}}}\\\hline
\textbf{Field Name} & \textbf{Field Type} & \textbf{Description}\\
\hline
Name          & string or integer & Name of new link type \\
ContainerType & string or integer & Type of common ancestral container \\
SourceContainerType & string or integer & Type of source container of link\\
DestContainerType & string or integer & Type of destination container of link\\
\hline
Alias         & string or integer & Alternative name of new link type \\
Shape         & string            & Name of shape used to represent entities\\
\hline
\end{tabular}%
\end{center}%
\caption{Fields of PajeDefineLinkType event}
\label{f:pajedefinelink}
\end{figure}



\subsubsection*{PajeDefineEntityValue\index{PajeDefineEntityValue}}
\label{sec:entvaldef}

Contains fields ``Name'', ``EntityType'' and optionally ``Alias''.  Used to give
names to the possible values of an entity type.  ``Alias''
will represent the value ``Name'' that entities of type ``EntityType''
can have.  

\begin{figure}[htbp]
\begin{center}
\begin{tabular}{|LLL|}
\hline
\multicolumn{3}{|T|}{\textsf{\textbf{PajeDefineEntityValue}}}\\\hline
\textbf{Field Name} & \textbf{Field Type} & \textbf{Description}\\
\hline
Name          & string or integer & Value of entity \\
EntityType    & string or integer & Type of entity that can have this value \\
\hline
Alias         & string or integer & Alternative name of new value \\
Color         & color             & Color of entities of this value\\
\hline
\end{tabular}%
\end{center}%
\caption{Fields of PajeDefineEntityValue event}
\label{f:pajedefinevalue}
\end{figure}

In the example started in \S\ref{sec:instant}, ``Thread State''s can be
``Executing'' or ``Blocked'', as shown in figure~\ref{f:definevalueexample}.

\begin{figure}[htbp]
\begin{center}
\begin{tabular}{|LL|}
\hline
\textbf{Field Name} & \textbf{Field Value} \\
\hline
PajeEventName & PajeDefineEntityValue \\
Name          & Executing\\
Alias         & E\\
EntityType    & S\\
Color         & "0 1 0"\\
\hline
\end{tabular}%
\quad\begin{tabular}{|LL|}
\hline
\textbf{Field Name} & \textbf{Field Value} \\
\hline
PajeEventName & PajeDefineEntityValue \\
Name          & Blocked\\
Alias         & B\\
EntityType    & S\\
Color         & "0.9 0 0.1"\\
\hline
\end{tabular}%
\end{center}%
\caption{Example of PajeDefineEntityValue event}
\label{f:definevalueexample}
\end{figure}

\subsection{Creation of visualizable entities}
\label{sec:creation}

There are different events to create entities of each possible type
(states, events, variables or links).
In events that create entities, the optional fields named "FileName" and "LineNumber" can be used to relate the created event to a position in a file, that can be obtained in Paj� during the inspection of the entity.
These events can also have the optional field named "RelationKey", to group entities that are somehow related to each other. All entities with the same key are highlighted in the space-time diagram when the mouse cursor is over one of them.

\subsubsection{States}

There are events to change a state
("PajeSetState")\index{PajeSetState}, to push a state, saving the old
state ("PajePushState")\index{PajePushState}, and to pop the
previously saved state ("PajePopState")\index{PajePopState}.

\begin{figure}[htbp]
\begin{center}
\begin{tabular}{|LLL|}
\hline
\multicolumn{3}{|T|}{\textsf{\textbf{PajeSetState}}}\\\hline
\textbf{Field Name} & \textbf{Field Type} & \textbf{Description}\\
\hline
Time          & date              & Time the state changed \\
Type          & string or integer & Type of state \\
Container     & string or integer & Container whose state changed \\
Value         & string or integer & Value of new state of container \\
\hline
\end{tabular}

\begin{tabular}{|LLL|}
\hline
\multicolumn{3}{|T|}{\textsf{\textbf{PajePushState}}}\\\hline
\textbf{Field Name} & \textbf{Field Type} & \textbf{Description}\\
\hline
Time          & date              & Time the state changed \\
Type          & string or integer & Type of state \\
Container     & string or integer & Container whose state changed \\
Value         & string or integer & Value of new state of container \\
\hline
\end{tabular}

\begin{tabular}{|LLL|}
\hline
\multicolumn{3}{|T|}{\textsf{\textbf{PajePopState}}}\\\hline
\textbf{Field Name} & \textbf{Field Type} & \textbf{Description}\\
\hline
Time          & date              & Time the state changed \\
Type          & string or integer & Type of state \\
Container     & string or integer & Container whose state changed \\
\hline
\end{tabular}%
\end{center}%
\caption{Fields of state changing events}
\label{f:pajesetstate}
\end{figure}

For example, if "Thread 1" blocks at time 2.34567 and unblocks at time
2.456789, the trace file could contain the events shown in
figure~\ref{f:setstateexample}.

\begin{figure}[htbp]
\begin{center}
\begin{tabular}{|LL|}
\hline
\textbf{Field Name} & \textbf{Field Value} \\
\hline
PajeEventName & PajeSetState \\
Time          & 2.34567\\
Type          & "Thread State" \\
Container     & "Thread 1"\\
Value         & Blocked \\
\hline
\end{tabular}%
\quad\begin{tabular}{|LL|}
\hline
\textbf{Field Name} & \textbf{Field Value} \\
\hline
PajeEventName & PajeSetState \\
Time          & 2.456789\\
Type          & S \\
Container     & T1\\
Value         & E \\
\hline
\end{tabular}%
\end{center}%
\caption{Example of PajeSetState event}
\label{f:setstateexample}
\end{figure}

\subsubsection{Events}

Events are created with the event named ``PajeNewEvent''\index{PajeNewEvent}.
Just like states, the values of events must be previously
defined by ``PajeDefineEntityValue''.

\begin{figure}[htbp]
\begin{center}
\begin{tabular}{|LLL|}
\hline
\multicolumn{3}{|T|}{\textsf{\textbf{PajeNewEvent}}}\\\hline
\textbf{Field Name} & \textbf{Field Type} & \textbf{Description}\\
\hline
Time          & date              & Time the event happened \\
Type          & string or integer & Type of event \\
Container     & string or integer & Container that produced event \\
Value         & string or integer & Value of new event \\
\hline
\end{tabular}%
\end{center}%
\caption{Fields of PajeNewEvent event}
\label{f:pajenewevent}
\end{figure}


\subsubsection{Variables}

There exist several events to set, add or subtract a value to/from a
variable\index{PajeSetVariable}\index{PajeAddVariable}\index{PajeSubVariable}.

\begin{figure}[htbp]
\begin{center}
\begin{tabular}{|LLL|}
\hline
\multicolumn{3}{|T|}{\textsf{\textbf{PajeSetVariable}}}\\\hline
\textbf{Field Name} & \textbf{Field Type} & \textbf{Description}\\
\hline
Time          & date              & Time the variable changed value\\
Type          & string or integer & Type of variable \\
Container     & string or integer & Container whose value changed \\
Value         & double            & New value of variable \\
\hline
\end{tabular}

\begin{tabular}{|LLL|}
\hline
\multicolumn{3}{|T|}{\textsf{\textbf{PajeAddVariable}}}\\\hline
\textbf{Field Name} & \textbf{Field Type} & \textbf{Description}\\
\hline
Time          & date              & Time the variable changed value\\
Type          & string or integer & Type of variable \\
Container     & string or integer & Container whose value changed \\
Value         & double            & Value to be added to variable \\
\hline
\end{tabular}

\begin{tabular}{|LLL|}
\hline
\multicolumn{3}{|T|}{\textsf{\textbf{PajeSubVariable}}}\\\hline
\textbf{Field Name} & \textbf{Field Type} & \textbf{Description}\\
\hline
Time          & date              & Time the variable changed value\\
Type          & string or integer & Type of variable \\
Container     & string or integer & Container whose value changed \\
Value         & double            & Value to be subtracted from variable \\
\hline
\end{tabular}%
\end{center}%
\caption{Fields of events that change value of variables}
\label{f:pajesetvalue}
\end{figure}


\subsubsection{Links}

A link is defined by two events, a
``PajeStartLink''\index{PajeStartLink} and a
``PajeEndLink''\index{PajeEndLink}.  These two events are matched and
considered to form a link when their respective ``Container'',
``Value'' and ``Key'' fields are the same.

\begin{figure}[htbp]
\begin{center}
\begin{tabular}{|LLL|}
\hline
\multicolumn{3}{|T|}{\textsf{\textbf{PajeStartLink}}}\\\hline
\textbf{Field Name} & \textbf{Field Type} & \textbf{Description}\\
\hline
Time          & date              & Time the link started\\
Type          & string or integer & Type of link \\
Container     & string or integer & Container that has the link \\
SourceContainer&string or integer & Container where link started \\
Value         & string or integer & Value of link \\
Key           & string or integer & Used to match to link end \\
\hline
\end{tabular}

\begin{tabular}{|LLL|}
\hline
\multicolumn{3}{|T|}{\textsf{\textbf{PajeEndLink}}}\\\hline
\textbf{Field Name} & \textbf{Field Type} & \textbf{Description}\\
\hline
Time          & date              & Time the link started\\
Type          & string or integer & Type of link \\
Container     & string or integer & Container that has the link \\
DestContainer & string or integer & Container where link ended \\
Value         & string or integer & Value of link \\
Key           & string or integer & Used to match to link start \\
\hline
\end{tabular}%
\end{center}%
\caption{Fields of events that create links}
\label{f:pajelink}
\end{figure}


\section{Example}
\label{sec:example}

The whole trace file of the example would be:

\begin{verbatim}

%EventDef       PajeDefineContainerType 1
%       Alias           string
%       ContainerType   string
%       Name            string
%EndEventDef
%EventDef       PajeDefineStateType     3
%       Alias           string
%       ContainerType   string
%       Name            string
%EndEventDef
%EventDef       PajeDefineEntityValue   6
%       Alias           string
%       EntityType      string
%       Name            string
%EndEventDef
%EventDef       PajeCreateContainer     7
%       Time            date
%       Alias           string
%       Type            string
%       Container       string
%       Name            string
%EndEventDef
%EventDef       PajeDestroyContainer    8
%       Time            date
%       Name            string
%       Type            string
%EndEventDef
%EventDef       PajeSetState           10
%       Time            date
%       Type            string
%       Container       string
%       Value           string
%EndEventDef
1 P 0 Program
1 T P Thread
3 S T "Thread State"
6 E S Executing
6 B S Blocked
7 0 TTP P 0 "Thread Testing Program"
7 0.986789 T1 T TTP "Thread 1"
10 0.986789 S T1 E
7 1.012332 T2 T TTP "Thread 2"
10 1.012332 S T2 E
10 2.34567 S T1 B
10 2.405678 S T2 B
10 2.456789 S T1 E
10 4.001543 S T2 E
8 4.295677 T2 T
8 4.34565 T1 T
8 4.3498 TTP P

\end{verbatim}


\section{Visualisation of the activity of the processors of a cluster}
\label{sec:admin}

The genericity of Paj� made it possible to visualize the system
activity of the processors of a large-sized (200 PEs) cluster of
personal computers \cite{GuilloudCAS:2001}. Several visualizations
were built from the system information available in the \texttt{/proc}
pseudo-directory of each PE. It was thus possible to represent the
processor and memory load of each PE, the most time consuming process
of each PE., etc. Paj� was also used to visualize the reservations of
PEs by the users of the cluster, the reservations being done using the
PBS system \cite{PBS} (see figure~\ref{fig:pbs}). The main bits of the
Paj� trace file analyzed to produce this figure are given below.

\begin{figure}[ht]
\epsfxsize=\linewidth
%\epsfysize=5cm
%\centerline{\epsfbox{FIG/pbs5.eps}}
\caption{Scheduling of jobs on a large-sized cluster}
\label{fig:pbs}
\end{figure}

\begin{verbatim}
%EventDef       PajeDefineContainerType 1
%       Alias           string
%       ContainerType   string
%       Name            string
%EndEventDef
%EventDef       PajeDefineEventType     2
%       Alias           string
%       ContainerType   string
%       Name            string
%EndEventDef
%EventDef       PajeDefineStateType     3
%       Name            string
%       ContainerType   string
%EndEventDef
%EventDef       PajeDefineEntityValue   6
%       Name            string
%       EntityType      string
%EndEventDef
%EventDef       PajeCreateContainer     7
%       Time            date
%       Alias           string
%       Type            string
%       Container       string
%       Name            string
%EndEventDef
%EventDef       PajeDestroyContainer    8
%       Time            date
%       Name            string
%       Type            string
%EndEventDef
%EventDef       PajeSetState           10
%       Time            date
%       Type            string
%       Container       string
%       Value           string
%EndEventDef
1 MG        0      M-Grappe
1 G         MG     Grappe
1 M         G      Machine
1 CPU       M      Processeur
3 pbs-task  CPU
7 7 MG1     MG     0   M-grappe_1
7 7 G1      G      MG1 Grappe_1
6 nobody pbs-task
6 chapron pbs-task
6 charao pbs-task
6 fchaussum pbs-task
6 feliot pbs-task
6 guilloud pbs-task
6 gustavo pbs-task
6 leblanc pbs-task
6 maillard pbs-task
6 mpillon pbs-task
6 paugerat pbs-task
6 plumejea pbs-task
6 romagnol pbs-task
6 sderr pbs-task         
7 8 M_icluster11  M  G1 M_icluster11
7 8 P_icluster11  CPU  M_icluster11 P_icluster11
7 8 M_icluster21  M  G1 M_icluster21
7 8 P_icluster21  CPU  M_icluster21 P_icluster21
7 8 M_icluster31  M  G1 M_icluster31
7 8 P_icluster31  CPU  M_icluster31 P_icluster31
7 8 M_icluster41  M  G1 M_icluster41

[...]

10 1 pbs-task P_icluster5 nobody
10 4273 pbs-task P_icluster5 nobody
10 5323 pbs-task P_icluster5 plumejea
10 7893 pbs-task P_icluster5 nobody
10 8277 pbs-task P_icluster5 feliot
10 8611 pbs-task P_icluster5 nobody
10 8633 pbs-task P_icluster5 feliot
10 8804 pbs-task P_icluster5 nobody
10 8836 pbs-task P_icluster5 feliot
10 9655 pbs-task P_icluster5 nobody
10 10038 pbs-task P_icluster5 feliot
10 10899 pbs-task P_icluster5 nobody
10 10930 pbs-task P_icluster5 feliot
10 10944 pbs-task P_icluster5 nobody

[...]

10 438224 pbs-task P_icluster100 feliot
10 438278 pbs-task P_icluster100 nobody
10 438339 pbs-task P_icluster100 feliot
10 438713 pbs-task P_icluster100 nobody
10 665686 pbs-task P_icluster100 sderr
10 665727 pbs-task P_icluster100 nobody
8 1465976  P_icluster100 P
8 1465976  M_icluster100 M
8 1465976 G1 G
8 1465976 MG1 MG


\end{verbatim}


%%%%%%%%%%%%%%%%%%%%%%%%%%%%%%%%%%%%%%%%%%%%%%%%%%%%%%%%%%%%%%%%%%%%%%%%%%%%%
%\chapter{Conclusion}
%%%%%%%%%%%%%%%%%%%%%%%%%%%%%%%%%%%%%%%%%%%%%%%%%%%%%%%%%%%%%%%%%%%%%%%%%%%%%
%%%%%%%%%%%%%%%%%%%%%%%%%%%%%%%%%%%%%%%%%%%%%%%%%%%%%%%%%%%%%%%%%%%%%%%%%%%%%%
%\chapter{Conclusion}
%%%%%%%%%%%%%%%%%%%%%%%%%%%%%%%%%%%%%%%%%%%%%%%%%%%%%%%%%%%%%%%%%%%%%%%%%%%%%

Paj� is a versatile visualization tool which can be used in a large
variety of contexts. This report describes the data format used by
Paj�. Paj� being trace-based, the data actually used for the
visualisation is to be presented as a set of execution events. In
addition, a description of the type hierarchy of the visual objects
needs to be included in the data (trace) file. Both the formats of the
type hierarchy description and of the events being self defined, there
also need to be a definition of these formats in the input data
(trace) file used by Paj�.

The versatility property has been used so far to visualize a
distributed Java application and the activity of the nodes of a
large-sized cluster of PCs. 

To enlarge the applicability of Paj�, a translator from traces
produced by Tau \cite{ShendeMCLBK:1998} into the Paj� format is
currently being implemented.




\bibliography{lang-paje}
\bibliographystyle{abbrv}

\addcontentsline{toc}{chapter}{Index}
\IfFileExists{lang-paje.ind}{%
  \documentclass[11pt,twoside]{report}
\usepackage{fullpage}
\usepackage{epsfig}
\usepackage{xspace}
\usepackage{alltt}

% \usepackage[francais]{babel}        % Pour Linux
\usepackage[latin1]{inputenc}
\usepackage[T1]{fontenc}

\setcounter{secnumdepth}{3}  %% pour num�roter les subsubsections
\setcounter{tocdepth}{3}     %% profondeur dans la table des mati�res

\usepackage{times}

\include{definitions} % D�finitions de la th�se de Benhur

\makeindex

\title{Paj� trace file format}

\author{B. de Oliveira Stein\\ 
Departamento de Eletr\^onica e Computa\c{c}\~ao\\
Universidade Federal de Santa Maria - RS, Brazil.\\
Email: benhur@inf.UFSM.br\\
\and
J. Chassin de Kergommeaux\\
Laboratoire Informatique et Distribution (ID-IMAG)\\
ENSIMAG - antenne de Montbonnot,\\ ZIRST, 51, avenue Jean Kuntzmann\\
F-38330 Montbonnot Saint Martin, France \\ 
Email:Jacques.Chassin-de-Kergommeaux@imag.fr\\
http://www-apache.imag.fr/\~\/chassin
}

\begin{document}

\maketitle

\begin{abstract}
  
  Paj� is an interactive and scalable trace-based visualization tool
  which can be used for a large variety of visualizations including
  performance monitoring of parallel applications, monitoring the
  execution of processors in a large scale PC cluster or representing
  the behavior of distributed applications. Users of Paj� can tailor
  the visualization to their needs, without having to know any insight
  nor to modify any component of Paj�. This can be done by defining
  the type hierarchy of objects to be visualized as well as how these
  objects should be visualized. This feature allows the use of Paj�
  for a wide variety of visualizations such as the use of resources by
  applications in a large-size cluster or the behavior of distributed
  Java applications.  This report describes the trace format used by
  Paj�. Traces include three different kind of informations:
  definition of the formats of the event, definition of the type
  hierarchy of the objects to be visualized, definition of the formats
  of the events of the trace and a set of recorded events, complying
  with the format definition, to be used to build visualizations
  according to the type hierarchy.

 \textbf{Keywords:} performance debugging, visualization, MPI, pthread, 
parallel programming, self defined data format.

  
\end{abstract}

\tableofcontents

%%%%%%%%%%%%%%%%%%%%%%%%%%%%%%%%%%%%%%%%%%%%%%%%%%%%%%%%%%%%%%%%%%%%%%%%%%%%%
%\chapter{Introduction}
%%%%%%%%%%%%%%%%%%%%%%%%%%%%%%%%%%%%%%%%%%%%%%%%%%%%%%%%%%%%%%%%%%%%%%%%%%%%%
%\input{intro.tex}

%%%%%%%%%%%%%%%%%%%%%%%%%%%%%%%%%%%%%%%%%%%%%%%%%%%%%%%%%%%%%%%%%%%%%%%%%%%%%
%\chapter{Extensibility of Paj�}
%\label{chap:paje}
%%%%%%%%%%%%%%%%%%%%%%%%%%%%%%%%%%%%%%%%%%%%%%%%%%%%%%%%%%%%%%%%%%%%%%%%%%%%%
%\input{extensibility.tex}

%%%%%%%%%%%%%%%%%%%%%%%%%%%%%%%%%%%%%%%%%%%%%%%%%%%%%%%%%%%%%%%%%%%%%%%%%%%%%
\chapter{Definition of type hierarchies and trace event formats}
\label{chap:format}
%%%%%%%%%%%%%%%%%%%%%%%%%%%%%%%%%%%%%%%%%%%%%%%%%%%%%%%%%%%%%%%%%%%%%%%%%%%%%
\input{typeevent.tex}

%%%%%%%%%%%%%%%%%%%%%%%%%%%%%%%%%%%%%%%%%%%%%%%%%%%%%%%%%%%%%%%%%%%%%%%%%%%%%
%\chapter{Conclusion}
%%%%%%%%%%%%%%%%%%%%%%%%%%%%%%%%%%%%%%%%%%%%%%%%%%%%%%%%%%%%%%%%%%%%%%%%%%%%%
%\input{conclusion.tex}

\bibliography{lang-paje}
\bibliographystyle{abbrv}

\addcontentsline{toc}{chapter}{Index}
\IfFileExists{lang-paje.ind}{%
  \input{lang-paje.ind}
}

\end{document}

}

\end{document}

}

\end{document}

}

\end{document}
