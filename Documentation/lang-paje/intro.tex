%%%%%%%%%%%%%%%%%%%%%%%%%%%%%%%%%%%%%%%%%%%%%%%%%%%%%%%%%%%%%%%%%%%%%%%%%%%%%
% \chapter{Introduction}
%%%%%%%%%%%%%%%%%%%%%%%%%%%%%%%%%%%%%%%%%%%%%%%%%%%%%%%%%%%%%%%%%%%%%%%%%%%%%

This report defines the input data format used by the Pajé
visualization tool. Pajé is a versatile trace-based visualization tool
designed to help performance debugging of large-sized parallel
applications. From trace files, recorded during the execution of
parallel programs, Pajé builds a graphical representation of the
behavior of these programs, to help programmers identify their
``performance errors''. The main novelty of Pajé is an original
combination of three of the most desirable properties of visualisation
tools for parallel programs: extensibility, interactivity and
scalability. 

Scalability is the ability to represent the execution of
parallel programs executing during long periods on large-sized
systems; it is provided in Pajé by zooming and filtering
functionalities, both in space --- ability to synthesize the
information originating from several nodes of the system or to zoom in
one of these nodes --- and in time --- possibility to display  period
of time at various levels of detail. Interactivity is the ability to
interrogate visual objects --- events, thread states, communications,
etc. --- to obtain more details or check the source code whose
execution produced a given event; it is also the ability to move back
and forth in time or to zoom from a synthetic representation to a
detailed one or vice versa or to set or reset a filter. Extensibility
is the possibility ot extend the tool with new functionalities ---
visual representations, filters, etc. --- or to display new
programming models. Several characteristics of Pajé contribute to its
extensibility: careful modular design, independence of the
visualization modules from the programming model.

Key to the ability to build a visual representation of the behavior of
parallel programs, developed with various programming models, is the
\textit{genericity} of Pajé: ability to parameterize the tools with a
description of \textit{what} is to be represented and \textit{how}.
This description is provided in the trace file as a hierarchy of the
types of objects appearing in the visualization. The format of this
description as well as the format of the events of the trace are also
described in the trace file. The trace files\index{trace file} used as
input by Pajé thus contain four categories of data:
\begin{enumerate}
\item Description of the format of the generic instructions.
\item Generic instructions, describing the hierarchy of the types of
  objects appearing in the visualization.
\item Description of the format of the events recorded during the
  execution of the visualized program.
\item Events recorded during the execution of the program to be
  visualized.
\end{enumerate}

The aim of this technical report is to describe the Pajé trace data
format. The organization of the report is the following. After this
introduction, the extensibility and genericity of Pajé are described
in detail. The following section defines the Pajé data format and
gives an example of use before the conclusion.


%Pajé is an interactive visualization tool originally designed for
%displaying the execution of parallel applications where a
%(potentially) large number of communicating threads of various
%life-times execute on each node of a distributed memory parallel
%system.   To be easier
%to extend, Pajé was designed as a data-flow graph of modular
%components, most of them being independent of the semantics of the
%parallel programming model of the visualized parallel programs. In
%addition, application programmers can tailor the visualization to
%their needs, without having to know any insight nor to modify any
%component of Pajé. This can be done by defining the type hierarchy of
%objects to be visualized as well as how these objects should be
%visualized.
